%--------|---------|---------|---------|---------|---------|---------|---------|
%       10        20        30        40        50        60        70        80


\cleardoublepage

\phantomsection\addcontentsline{toc}{chapter}{Campaign}
\chapter*{Campaign}
\chaptermark{campaign}


The campaign aspects of the hactac project were the most fun, so we'll keep going with that. The death toll was lower than I expected through all the main campaign threads we've played so far. There is a surprisingly strong effect of Heroes being able to retreat and regroup when things were getting dire. Not intentionally, but good, and excellent for campaign gaming. Surviving Heroes generally make for better long form stories. 

For campaign gaming you'll need a group of players that is somewhat consistent and that can schedule sessions. The players should probably build their own heroes, perhaps starting from scratch with 100xp newbies. And you'll need a campaign arc spanning several adventures. There are already released campaigns available, and more is on the way. Ping the author if you want the notes from the as of yet unreleased adventures and campaigns. We have hundreds of hours worth of adventures available spanning several campaigns

\

The game is primarily designed for short sessions with fast gaming, and we play mainly by remote with a virtual tabletop (\emph{as of late 2019, still maptools and mumble, still working well even for the sometimes enormous battles of the Goblin Destiny campaign.}). We'll continue with the usual disconnected reality clichés, either by rules, circumstance, or magic, so that it is trivial to inject and extract characters through the sessions. I expect people to be connecting/disconnecting due to family, scheduling, net/bugs, etc. The campaign style and encounter design must accommodate for issues like that.




%===============================================================================
%                    C R E A T I N G   C H A R A C T E R S
%                    -------------------------------------

\phantomsection\addcontentsline{toc}{section}{create a character}
\section*{Creating characters}
%-----------------------------
\begin{enumerate}
\item Roll/select/build your brand new Hero-to-become character. Decide with your group and GM how the characters should be rolled up. Suggestions below.
\item Buy skills that makes sense for the style and role you want your character to have. Tailor to the game style your group is looking for: more role play, more fun nonsense, more tactical challenges?
\item Select some suitable shitty noob equipment together with the GM. Spend some of your hard earned and long saved coin to upgrade some critical equipment perhaps?
\item \emph{If} this is your first time, play through the \texttt{Dungeon of Testing} solo with your GM to see if your character has enough survival potential to leave his childhood homestead and venture into the World Outside for Danger and Adventure in Far Away Lands, or the village next by.
\end{enumerate}
% edit below (200205): hmm, fun, but gives the wrong impression actually, mortality has been shown to be lower than originally expected. Most characters survive well past 0.3k maturity. Approximated mortality is <20% up to 0.5k
%Don't spend too much time creating your first character, it will probably die soon anyway, and you'll have to make another, and another, and yet another one again. 
%It might even be a good idea to create a bunch of them in the first place so that you have the queue already waiting by the mouth of the dungeon, ready to charge into the darkness in search of a quick and painful death.

Now, all depending on game style: For combat oriented gaming there are a few things you should consider. It might be worth it to focus the character on a single role when creating a 100xp newbie. He will not have enough points to get good at several different tasks in the very beginning. Skills you expect to use regularly should have level 5 or better, and the main combat skill at lvl 7 or more is a good start. If possible, try to spread out some basic support or campaign skills over the Heroes in the party.

Early on it's good to have a strong defence. Shields and two handed staffs are good defensive options since their bonuses makes them more reliable in the early adventures. E.g: The cost to get an effective 8 in parrying defence action varies greatly. For sword it's 64xp, for 2h staff it's 28xp, and for a normal shield it's 17xp. The strongest defensive action is generally \emph{avoid}, but that costs 51xp for lvl 8.

A standard fighter should probably have a main weapon and a shield. An alternative if he has good mobility is a two handed high damage weapon and good avoid capacity. A ranged fighter can be fast to stay out of trouble by positioning, or also carry a defensive melee weapon such as a staff or shield. The same goes for spell casters. Close combat tends to come for everyone quite often. Archers should probably also be quick at changing between the bow and the melee weapon.

You can find some example Heroes at various experience levels in the \hyperref[cpt:characters]{characters section}, page \pageref{cpt:characters}.


\subsection*{Rolling character stats}
%------------------------------------
Further below are lists of different races and their specific stat rolls. The GM together with the player group should decide how to roll up characters. Straight up one roll? Rerolls? How many? Best of 3? Roll then distribute? Points scheme? How many points?

I recommend using the \verb|rollchars.py| script to roll up a set of characters that the player can choose from. Easy to tweak how many characters of the different races should be in a fresh rolled noob set.

%Note that you may use the "hero" version of the rolls if you want, and that you may roll movement instead of choosing the race standard if so desired.

%The GM should decide how many character sets a player is allowed to re-roll when creating characters. However, I suggest allowing the players to roll more humans than dwarves, elves, and orcs. For example, allow the player to roll three humans or goblins, but only two dwarves, elves, orcs, halflings. Then select the one he wants to play. The humans and goblins are less powerful than some of the other races. By allowing the player to roll more alternatives this power deficiency is balanced a bit.

%To make things simple I suggest using the \verb|rollchars.py| script. Just change the number of total characters to choose from and run the script. It will roll a selection of characters from all races.


\subsection*{Buying skills}
%--------------------------
The cost of buying skills is the "skill cost factor" (scf) times the square of the skill level, round down. \verb|cost = scf * sl^2|. However, a skill will always cost at least 1xp. \\
E.g: skill cost factor 1.3 to level 4 costs 1.3*16=20.8 round down ~20. \\
E.g: raise same skill from 4 to 6: 1.3*(36-16)=26. \\
Note that it can be a little bit cheaper to raise skills a little bit at a time instead of a lot in one go. However, if you want to do this you have to go adventuring between every skill purchase.

Sometimes characters start with some free skills rolled up with the character. Those are free above the rolled up starting xp value. Save your original pristine rolled up noob version of your character, as is, before spending any xp.


%\subsubsection*{some skills to consider}
%%---------------------------------------
%There are some basic skills that are worth considering in the early stage of your character development.
%gossip, find, track, \\
%avoid, jump, climb, swim, balance, ride, travel, \\
%literate, counting, (language), lock&traps, sneak, \\
%histography, monsterology, dungeoneering, first aid, \\




%NOTE: childhoods and professions disabled, this also disbles ability:munchkin
%
%\subsubsection*{Buy a childhood}
%%----------------------------
%\subsubsection*{Buy a profession}
%%--------------------------------
%\subsubsection*{Buy a childhood / Buy a profession}
%%--------------------------------------------------
%Childhoods and professions have been removed.
%They became obvious easy choices and made it
%too simplistic to opt for specific targets.
%Hence the character creation became less interesting.
%That was of course not the intent. So, gone for now, until fixed...
%
%%Still available in \emph{childhood-and-profession.tex} file.


\subsection*{The character's background}
%---------------------------------------
Once the character has a full set of traits, you go ahead and make up a suitably funny background. Also, get some starting gear that makes sense for the character, nothing too flashy and useful. He is a newbie after all. If it is expensive, it should be generally useless for combat and hard to sell.


%\subsection*{The inbred rule}
%----------------------------
%The shittier the character, the richer he may be. Nobility is rich, and inbred. Therefore, crappy stats \emph{might} mean that he is a dim witted offspring of some nobleman or other. It might also mean that he is a total looser, and poor on top of that.
%
%A general rule could be that the character may have non-combat equipment to a total value of 1 gold per total point under average in his primary stats (str-cha).


\subsection*{Selecting equipment}
%.-------------------------------
Any aspiring adventurer can get a "starter kit" suitable for his background, all for free! However, the starter kit equipment is generally of really poor quality. Not 2nd hand, but probably 13th hand. It is generally unsellable stuff. The shitty start equipment will have strange quirks and failure modes, and worse stats and penalties than the standard equipment.

See the "pregen newbies" in the character file for some examples. Some of them have spent money to improve their free starter kit gear.

%The inbred rule can be applied here to give a more expensive, (but generally non combat) gear to chars with lousy stats.

Improving the starer kit can be done by spending money, or by purchasing a newbie starter kit (see below) Look through the equipment list and see what is available. Newbies probably don't have access to exotic equipment, and perhaps only 50\% chance to special equipment.

Some interesting equipment to purchase can be: \\
light source: torch, lamp, candles, etc. \\
fire tools: flint and steel, coal box, etc. \\
food and wine, and a waterskin or two. \\
sacks or bags for loot. \\
backpack for easy carrying of equipment. \\
rope to help people climb up out of trap pits. \\
grappling hook if no-one is left outside the pit. \\
pick axe for getting out of the collapsed tunnel...

Some character backgrounds would allow them to start out rich, and with some expensive equipment, but then it should come with a serious down side. Perhaps the princelings manservant is an evil backstabber, plotting with the local highway robbers to steal everything and kill the little snot in the process?


\subsection*{Equipment newbie starter kits}
%------------------------------------------
Starting out in the Hero Adventuring business can be difficult, and it's so useful to have a small knapsack of stuff with you when you leave the old homestead. Character creation is probably also the only time when your Hero can buy equipment for xp instead of silver.

This equipment is of decent quality and not the kind of hand me down rusty old fail-at-the-worst-possible-moment standard starter fare that the less prepared newbie adventurers tote around.

\

\goodbreak \small \begin{samepage} \begin{verbatim}
Woodsman's backpack (cost 5xp):
    backpack 5enc, knife, spear or bow with 10 arrows,
    water bottle, food for 3+1d3 days, fishing gear, snares,
    flint & steel, tinder, 1d3 torches,
    water proof hooded cape, 5+D10 copper.
\end{verbatim}\goodbreak\begin{verbatim}
Uncle Smith's well wishes (cost 8xp):
    sword, shield, warm cape, water bottle, 20+D10 copper.
\end{verbatim}\goodbreak\begin{verbatim}
Farmer Johns goodbye (cost 4xp):
    knife, staff / axe, shoulder sack, water bottle, food for 5+D10 days,
    flint \& steel, warm blanket, 10+D10 copper.
\end{verbatim}\goodbreak\begin{verbatim}
Auntie's present (cost 5xp):
    3x healing salve, first aid kit, warm sweater, 5+D10 copper.
\end{verbatim} \end{samepage} \normalsize

\

\noindent Make up more suitable starting kits to suit your adventures and campaign setting.


\subsection*{rolling a character, or selecting one}
%--------------------------------------------------
For most campaigns I'll keep using the method of pre-rolling a set of potential Heroes for the player to choose from. The simple \verb|rollchars.py| script rolls up sets of basic characters. It's easy to tweak the selection set of races for a specific campaign or adventure setting.
For one-off or later added Heroes the player can instead roll a small set and choose a character from there.


\subsection*{Alternative movement rolls}
%---------------------------------------
Instead of taking the base line race movements you can roll the character's movement stats. You can either roll one d10 and apply the result to calculate all the movements, or you can roll for each movement stat.


\subsection*{Race traits}
%------------------------
Consider, what's the main focus of the game for yourself and your group? Is it tactical tabletop battles, more role playing and character identity, or just silly fun? Tailor the character creation process to support the goal of the game.

\

If ignoring combat power balance for the moment, different races have typical personality characteristics associated with them. Not that all dwarves are the same, just that there are clear cultural and racial tendencies. Change to suit your flavour of game world of course.

\begin{description}
\item[Humans] are the normal folk of most of the land. Found just about anywhere and in all occupations. Individually and culturally highly varied.
\item[Dwarves] are rough, stern, determined, disciplined, with a lovely sense of humour that few outsiders understand. They have strong collective responsibility.
\item[Elves] few and far between, artistic, aloof, or austere, all in all quite a bit strange. 
\item[Halflings] are usually happy with a great apetite for life, merriment, good food and drink.
\item[Orcs] have grit, nothing is too heavy or too dangerous. But don't get them angry.
\item[Goblins] have difficulties: lazy, impulsive, short sighted, wildly swingin between agressive and meek, cowardly and blood thirsty, usually hungry, and with a nasty inkling towards schadenfreude and dangerous ill thought through pranks. Loads of fun to play!
\end{description}

\noindent All races have their own general strengths and weaknesses. Some make better fighters, others better mages, yet others make great cannon fodder and compost. Some are just for fun.
There is no strict balance between how \emph{good} or \emph{powerful} the various character races are when making a new Hero. Instead I've chosen to apply some balance based on the number of characters that are rolled for selection. I recommend the following spread:

\

When choosing race before rolling, roll a small set of one race only:\\
3 human, 2 dwarf, 2 elf, 3 halfling, 2 orc, 4 goblin.\\
The characters created this way will be less powerful than the ones rolled as a larger set of any race.

\

When choosing one character regardless of race, roll:\\
2 human, 1 dwarf, 1 elf, 2 halfling, 1 orc, 3 goblin.\\
Odds are high that this will yield some overpowered newbie characters, but it will give a fun mix of races.

\

For my campaigns I've started using the second option and simply use \verb|rollchars.py| to roll small set of characters, of all races, for each player to choose a character from.

\

\emph{Please note:} that the values below are for rolling Heroes and important NPCs. Every Tom, Dick and Harry don't have cool rolled stats. For most NPCs and general uninteresting cannon fodder and back drop characters use the standard race values, and perhaps +/- a few points here and there for variation. The npc standard race averages are listed further below, and in the \verb|races| summary compilation.




%                 d10/10 1-9                 10     0,1  9,1
%                 d10/9  1-8               9-10     0,1  8,2
%                 d10/8  1-7               8-10     0,1  7,3
%                 d10/7  1-6               7-10     0,1  6,4
%                 d10/5  1-4      5-9        10     0-2  4,5,1
% below are the symmetric splits:
%                 d10/6  1-5               6-10     0,1  5-5
%                 d10/4  1-3      4-7      8-10     0-2  3-4-3
%                 d10/3  1-2   3-5   6-8   9-10     0-3  2-3-3-2
%                 d10/2  1  2-3 4-5 6-7 8-9  10     0-5  1-2-2-2-2-1




\

Here is the basic character sheet layout. It has already been explained above, in the \hyperref[sec:charsheet]{character sheet section}, page \pageref{sec:charsheet}.

\

\goodbreak
\begin{samepage} \begin{verbatim}
===================================
name                        (token)
-----------------------------------
str          hp abs
dex          m w r d
con          stamina
int          vision arc
psy          mana
per          ap
cha          xp
----------
skills
----------
spells
----------
equipment
money ( g s c )
===================================
\end{verbatim} \end{samepage}

\

\goodbreak 
\begin{samepage}
\noindent The basic stat acronyms:
\begin{verbatim}
basic character traits
----------------------
(long)        (eng) (old Swedish)
strength       str   sty
dexterity      dex   smi
constitution   con   fys
intelligence   int   int
psyche         psy   psy
perception     per   itf (attention att)
charisma       cha   kar/per

hitpoints      hp    kp
absorption     abs   abs
movement (maneuver walk run dash)   m/w/r/d
stamina        sta
vision         vis (distance) arc (angle)
magic power    mana
action points  ap
experience     xp

money          gold silver copper (g s c)
\end{verbatim} \end{samepage}

\


% =============================================================================
\TODO The race Hero roll listings below are not updated yet! Refer to, and use \verb|rollchars.py| as up to date. And see \verb|races| listing for summary.
% -----------------------------------------------------------------------------

\


\small

\goodbreak \begin{samepage} \begin{verbatim}
===================================
human                       (token)
-----------------------------------
str  2d5     hp 2d10 abs 0         (alt hp 2d6+8 for heroes [10-20])
dex  2d5     m1 w3 r6 d9           (alt sta 1d7+3 for heroes [4-10])
con  2d5     stamina 2d5           (alt movements (all round down))
int  2d5     vision 15+1d10         m = 1+d10/10
psy  2d5     mana 2d10              w = 2+d10/4
per  2d5     cap 7+1d6              r = 5+d10/4
cha  2d5     xp 100+1d20            d = 8+d10/3
vision arc 180+1d90 deg
yield bonus is 2+d10/4 round down
money 1d4 gold 1d8 silver 1d20 copper
\end{verbatim} \end{samepage}

\

\goodbreak \begin{samepage} \begin{verbatim}
===================================
dwarf                       (token)
-----------------------------------
str  2d5+2   hp 2d10+2 abs 0       (alt hp 2d6+10 for heroes [12-22])
dex  2d5-1   m1 w2 r4 d6           (alt movement (all round down))
con  2d5+2   stamina 2d5+5          m = 1+d10/8
int  2d5     vision 15+1d5 infra*   w = 2+d10/6
psy  2d5     mana 2d10              r = 3+d10/4
per  2d5     cap 8+1d4              d = 5+d10/4
cha  2d5-1   xp 110+1d20
vision arc 120+1d90 deg
yield bonus is 2+d10/4 round down
con+3 against poisons
money 1d10 gold 1d20 silver 1d20 copper + gem stone (5+1d6 gold)
* infra vision is dusk vision until good implementation exist in maptools
haggle bonus +3
dwarves without any gems, or with less than 5 gold total coin
suffers psy-1 mod until wealthy again
dwarves with 50+ gold in coin and gems gains psy+1 mod while wealthy.
\end{verbatim} \end{samepage}

\

\goodbreak \begin{samepage} \begin{verbatim}
===================================
elf                         (token)
-----------------------------------
str  2d5-2   hp 2d9 abs 0          (alt hp 2d6+6 for heroes [8-18])
dex  2d5+1   m2 w4 r8 d12          (alt movement (all round down))
con  2d5     stamina 1d5+7          m = 2+d10/8
int  2d5+1   vision 20+1d10 night   w = 3+d10/4
psy  2d5     mana 2d10+4            r = 7+d10/4
per  2d5+2   cap 9+1d4              d = 11+d10/4
cha  2d5+2   xp 120+1d20
vision arc 220+1d90 deg
yield bonus is 2+d10/3 round down
immune to poisons
money, what for?, Well I have 1d4 silver and 1d10 copper somewhere.
haggle bonus -3
elves who are staying in a city or cave without access to nature
suffer psy-1 mod per week to max -3. Immediately restored to mod-0
when returning to nature.
\end{verbatim} \end{samepage}

\

\goodbreak \begin{samepage} \begin{verbatim}
===================================
halfling                    (token)
-----------------------------------
str  2d5-3   hp 2d8 abs 0          (alt hp 2d5+6 for heroes [8-16])
dex  2d5+2   m2 w3 r6 d8           (alt sta 1d6+2 for heroes [3-8])
con  2d5-1   stamina 1d5+2         (alt movement (all round down))
int  2d5     vision 15+1d10         m = 2+d10/10
psy  2d5     mana 2d10              w = 2+d10/4
per  2d5+2   cap 8+1d6              r = 4+d10/3
cha  2d5+1   xp 100+1d20            d = 6+d10/3
vision arc 270+1d90 deg
yield bonus is 2+d10/3 round down
sneak bonus +3
find bonus +3
money 1d4 gold 1d8 silver 1d20 copper
\end{verbatim} \end{samepage}

\

\goodbreak \begin{samepage} \begin{verbatim}
===================================
orc                         (token)
-----------------------------------
str  2d5+3   hp 2d10+5 abs 0       (alt 2d8+9 for heroes [11-25])
dex  2d5     m2 w4 r6 d8           (alt movements (all round down))
con  2d5+3   stamina 2d5+5          m = 1+d10/4
int  2d5-2   vision 10+1d10 dusk    w = 3+d10/4
psy  2d5-2   mana 2d10-5            r = 5+d10/4
per  2d5     cap 7+1d5              d = 6+d10/3
cha  2d5-3   xp 80+d20
vision arc 120+1d60 deg
yield bonus is 2+d10/4 round down
veteran bonus +3
brawling bonus +3
attack: bite 1d4 dam 2+d2 4ap scf 0.75
double con against poisons
orcs without any war trophies suffer psy-1 mod.
money 1d6 silver 1d10 copper 1d4 large teeth/claws
\end{verbatim} \end{samepage}

\

\goodbreak \begin{samepage} \begin{verbatim}
===================================
goblin                      (token)
-----------------------------------
str  2d4-2   hp 2d6 abs 0          (alt 2d5+4 for heroes [6-14])
dex  2d5     m1 w3 r5 d7           (alt movement (all round down))
con  2d3     stamina 2d5-2          m = 1+d10/10
int  2d4     vision 15+1d10 dusk    w = 2+d10/4
psy  2d4     mana 2d8-4             r = 3+d10/3
per  2d5     cap 6+1d6              d = 5+d10/3
cha  2d4-2   xp 60+1d20
vision arc 180+1d90 deg
yield bonus is 2+d10/3 round down
sneak bonus +3
hide bonus +3
disengage bonus +3
attack: bite 1d4 dam 1d3 3ap (2ap if both hands free) scf 0.75
attack: claw 1d4 dam 1 2ap (1ap if both hands free) scf 0.66
money 1 silver 1d6 copper 1d4 teeth 1d3 stones 1d2 feathers 1d3 glass beads
\end{verbatim} \end{samepage}

\

\goodbreak \begin{samepage} \begin{verbatim}
Goblin runts should be crated from regular goblin rolls:
str/2, dex+1, con/2,
hp/2, ap+1, xp-10,
yield+1, avoid+2, sneak+2,
common-2, svartlingo-1
round up for all half values.

As the runt grows up to be a grown goblin he will snap into the full 
rolled values and loose his runt bonuses.
\end{verbatim} \end{samepage}

\normalsize





%===============================================================================
%E X A M P L E   N E W B I E   C H A R A C T E R S
%-------------------------------------------------

\subsection*{Example characters}
%-------------------------------

A set of example newbie characters can be found in the 
\hyperref[cpt:characters]{characters section}, 
page \pageref{cpt:characters}, 
amongst those: \\
 \\
Morten Flaff, an all round fighter neophyte. \\
MistaMuhda and GrabbaKill, two violent young goblins. \\
Parry Hotter the Fizzler, a young wizard's apprentice. \\
KrijgRauch EckStein, a young dwarf fighter fresh out of the mountains. \\
Ein von Dääken, a noble fop who would probably do better writing sonnets. \\
Pyttelina, a halfling thief ready to take (on) the world. \\
Lars LongShot, a young hunter who is nowadays hunting for gold instead of geese. \\
Newton the Happy Farmboy, a poor farm hand turned wannabe fighter. \\

\

\todo update and clean up the various remaining example characters

\

\noindent
Manny the Mage Trainee, another spellcaster in the making. \\
 \\
A few examples of more experienced characters can also be found after the newbie characters.










%===============================================================================
% A D V E N T U R E S   A N D   C A M P A I G N   G A M I N G
%------------------------------------------------------------

\phantomsection\addcontentsline{toc}{section}{adventures and campaigns}
\section*{Adventures and campaign gaming}
%----------------------------------------
This is of course the main purpose of the exercise, to be able to play interesting adventures and string them together into longer running campaigns. Several adventures were available for download from the rptools / maptool gallery when that service was still up. I'll see if I can put them together again and push them somewhere. In the meantime all adventures (in whatever state of completion or decrepitude) are available from the author.

This project is designed mainly for short evening sessions, two to three hours of gaming on weekday evenings. The guidelines below reflect our experience from this kind of game scheduling.

Balancing character progression has been shown to be difficult for many game designers and is often poorly done in many well known adventure and table top games. By now we have over 150 game sessions under the belt, playing around 25 different characters varying between 100-1000xp. The general quadratic cost of skill progression has been shown to work well for giving balanced mid and late game play. We have not yet played much with characters over 1000xp, so there is not enough data for very experienced heroes.

A good general idea is to try to design adventures and encounters to give a reward of around 5-10xp per character per session with another 10-30xp bonus for mission completion or for specific side tasks. This has been shown to give a suitable progression curve as well as the occasional nice payday feeling for the players after special accomplishments.
Most adventures are designed to take three to six sessions, i.e. 10-20h to play. The longer adventures should give partial completion rewards in between.

Differentiate between "opposition xp" and "mission xp". Give opposition xp after each session, but only mission xp when specific tasks have been completed or milestones reached.
Opposition xp should be divided between the characters and henchmen. If they are many, then they get less per person. Mission xp on the other hand should be handed out at the full amount to each participant.
Consider giving full or partial opposition xp also for monsters that were avoided, scared away, tricked, negotiated with, or otherwise managed even if they were not killed.

It could also be a good idea to set a minimum base line of ca 5xp or so per session regardless of accomplishments. That way there is always a slow progression ticking away, but nothing game breaking. I generally reduce the mission xp by the already handed out session xp.

XP is of course not the only reward. Money and equipment is also common. It is usually a good idea to keep the heroes poor most of the time with short bursts where they can roll in gold. A Quaff or mana potion should always be an expensive investment, and not just stuff you buy to pad the equipment list to the next encumbrance limit.

Silver coins close to XP reward is a decent balance for everyday adventuring, with more lavish pay-outs from some adventures. A lot of "civilised" monsters carry some coin, and their equipment is usually worth a little bit even if it is in really poor shape or of orcish origin. This should be taken into account when calculating rewards if the players are known to scavenge.

Be very careful when handing out magical or master crafted items. Those are worth a lot of money, and they are very powerful in game. Even a well crafted sword with mod+1 is a significant boost for a normal character, and that bonus is worth a lot of xp in the hands of an experienced character. A high abs sword or shield is in practice close to unbreakable and will never need replacing. Saves a lot of money.
I once did the \emph{huge} mistake of placing a (fast+1) 2ap 2h sword in a campaign. I'll never do that again :(


\subsection*{Death, Replacement, Inheritance}
%--------------------------------------------
Heroes will die occasionally, it's an occupational hazard. When that happens let the player roll a new one and have it inherit some of the XP from it's predecessor. E.g: First Fred dies after gaining 50xp from the first few adventures. Second Sam can then be built with +30xp inherited? Discuss it with the players to balance the loss against keeping pace with the rest of the party. Whatever works with the campaign difficulty scaling and keeps the game fun.

Generally, keep the party somewhat in power parity. Weak newbie characters cannot do much if surrounded by 1k ultra murder machines, and it quickly gets boring for the players.


\subsection*{Available campaings and adventures}
%-----------------------------------------------
%200204:
There are a bunch of adventures and campaigns already available, in various states of polish. Just grab and play. Contact the author if you want the notes from the yet not released adventures.

\begin{description}
    \item[00 Dungeon of Testing] is the first little mini adventure every new player should go through with their first character, solo.
    \item[00 Return of Uchly Namen] is the first campaign in the game and world setting. Intended for newbie characters and containing a bunch of diverse interconnected adventures through a cohesive campaign arc with a few off-shoots.
    \item[01 Dark Clan, Deep Cave, Rescue Nurensachs] are three linked adventures forming a mini campaign of it's own, intended for more experienced characters.
    \item[02a Overlord Orvar] is a stand alone adventure, somewhat classical in nature. Easy to insert as a significant side quest in an existing campaign.
    \item[02b Edwin the Chromophobe] is a stand alone adventure, a large dungeon hack with impact on the regional world map. 
    \item[03 Ottokar's Test Dungeon lvl 1] is for experienced Heroes who want to get a proper license for taking Hero Adventure jobs.
    \item[04 Destroy the Altar] is a Hero Adventurers' job for experienced characters and players. It small but has some challenging oddities.
    \item[05 Turn off the Light] is another Hero Adventurers' job for very experienced and tough Heroes and skilled players. The starting challenge is tactically very different from the previous experiences.
    \item[06 Goblin Destiny] is a long stand alone campaign for a large bunch of newbie goblins Heroes. More fun than serious, with unique tactical and role playing challenges. A large set of smaller adventures and events, spread over time, around the life of a goblin bandit clan.
\end{description}


