%--------|---------|---------|---------|---------|---------|---------|---------|
%       10        20        30        40        50        60        70        80


\cleardoublepage

\phantomsection\addcontentsline{toc}{chapter}{equipment}
\chapter*{Equipment}
\chaptermark{equipment}

\begin{readoutloud}
It's very brave to journey up the mountain, in the dead dark of winter, to slay the three headed beast which dwells in the cave, wearing nothing but your nightgown ... but why would you? A nice warm padded brigandine armour? Perhaps a sword and a spear? You'd be glad you brought extra when the first one breaks. Don't forget a shiny shield. Why not have your family crest on it? You don't have a crest? I'll have a herald sort that out for you while you wait. A warm cup of coffee afterwards? And a smoulder box to quickly start the fire up there in the icy, dark, stormy winds. Don't forget the lantern, and some extra oil. A few torches to place at strategic corners. Some rope and a hook? A few sacks if the monster has loot stashed away? Almost forgot the water skin. Can't go into battle thirsty. And some rations if the climb takes longer than expected. At this point you'll need a large backpack, but that's heavy. I have this great friendly donkey and two pack bags. He's a great climber up steep terrain. And some oats for the donkey of course, and a few larger water skins as well. And with the donkey to carry things you can easily bring this nice tent, and a comfy bed roll, and this brazier to warm you. A cook pot and some fresh spices perhaps? Goes great with the rations, right? A bottle of wine? But don't drink before taking on the monster.

And while we're at it, I also have this wonderful ...
\end{readoutloud}






%---------    equipment.tex   all done changing verbatim listings



%\
%
%%\goodbreak \small \begin{samepage} \begin{verbatim}
%%\end{verbatim} \blocklistgap \begin{verbatim}
%%\end{verbatim} \end{samepage} \normalsize \goodbreak
%%
%%\
%%
%\TODO continue blocklist replacement here
%
%
%\small \begin{samepage} \begin{verbatim}
%\end{verbatim} \blocklistgap \begin{verbatim}
%\end{verbatim} \end{samepage} \normalsize
%
%\
%



%---------






%-------------------------------------------------------------------------------
% W E A P O N S  &  A R M O U R
%------------------------------

\phantomsection\addcontentsline{toc}{section}{weapons and armour}
\section*{Weapons and armour}

Weapons vary greatly in how they behave in game. Damage, absorption, penetration, speed, reach, modifications to attacks or parries, or how difficult they are to parry or avoid, special versions of attacks, and so on.

The weapons listed below are the common weapon types and sizes. Custom made weapons are treated further down.

There is no realism here. Weapon and equipment names, general and relative statistics, etc, have nothing to do with the real world. Everything is chosen simply to sound interesting and convey some colour to general players who have not spent time studying historical weapons, armour, and equipment. Lots of names like longsword, baselard, estoc, etc, have been heretically misappropriated for fun and flair. These are all fantasy weapons, imagine them like fairytale and fantasy story illustrations, not like actually really useful items.

\

Damage stats are supposed to be read as 1dX. E.g: the sword does 1d6 damage not always 6. Weapon penetration is however always the max value and is not rolled. E.g: A hackaxe with dam 5 pen 2 does 1D5 damage but ignores up to 2 armour absorption. So, against plate armour damage absorption 3: Roll 1d5 get 4, pen 2 ignore 2 armour, does 3 damage through the plate. And against leather abs 1: Roll 1d5 get 2, pen 2 ignore the 1 abs, it does 2 damage through the leather.


\begin{description}

\item[dam] is the maximum damage of the weapon from a normal attack. The attacker rolls for damage in the range [1,dam], where various skills affect how that is done. E.g: dam 7 means damage 1d7, not 1d6+1. For physical tabletop games, dice sets from 2-20 in increments of 1 can be purchased, or simply use your smartphone dice / rng app. On VTT the dice roll or rng scripts support any "dice" rolls.

\item[pen] is the penetration of the weapon, how many points of armour the weapon ignores when applying damage. Penetration never converts to damage, and it's always the full value, never rolled. E.g: dam 3 pen 2 cannot do more than 3 damage, and always ignore 2 abs armour.

\item[abs] is how much incoming damage the weapon can absorb without taking any damage when e.g. it's being used to parry an incoming blow. If the abs is exceeded by the incoming damage the excess damage is applied to the wielder and there is a risk that the weapon either takes damage or breaks. Armour also has abs but doesn't break by excess damage.

\item[reach] is the range of a melee weapon. Normally a melee weapon can only be used to strike targets adjacent to the wielder, but with reach it is possible to attack targets further away. Using the reach distance usually carries a modifier.

\item[mod] is a blanket modifier that makes the weapon more difficult, or easier, to use regardless if it's for an attack, parry, deflect, or whatnot.

\item[attack] modifier applied when using the weapon to attack.

\item[defend] modifier applied when defending with the weapon, like parry or deflect.

\item[parry] modifier applied only when parrying or deflecting with the weapon.

\item[todefend] modifier applied to the target for any defensive action against the incoming attack, including avoid, parry, deflect, etc.

\item[toparry] modifier applied to the target for a defensive parry or deflect action against the incoming attack.

\item[toavoid] modifier applied to the target for a defensive avoid action against the incoming attack.

\item[finesse] is an inherent limitation of the weapon on how extra difficult the attacks can be made for the target to defend against when the attacker is using the fancy attacks skill or tricky attack maneuvers.

E.g: Fancy Fred has fancy attacks 5 but swinging an axe which has finesse 3 and toparry-1 toavoid+1. FF cannot make the attacks more difficult than toparry-4 and toavoid-2, since the weapon finesse is the limiting factor. With a proper sword with finesse 6 FF could use his whole fancy attacks 5 giving his attacks a todefend-5 difficulty.

The same limitation applies with tricky maneuvers like roundabout smack. The maneuver gives a maximum todefend-6 but if the attacker is wielding a broadsword with finesse 4 the attack is only todefend-4.

\item[str] is the strength requirement of the weapon. Any lacking strength from the wielder gives mod-1 dam-1 when using the weapon.

\item[str bonus cap] are limits on how much strength bonus can be used with the weapon. E.g. a max str dam 1 pen 2 limits the strength bonus to damage to 1 with another 2 giving penetration, to a combined maximum of 3.

\item[stamina free attacks] some weapons allow for free attacks that does not cost stamina. First attack in a round always cost stamina, but it can be that e.g. every second or third attack cost no stamina.

\item[range] of bows, bottles, boulders, etc. The normal range is given and usually the weapon then also has short range at normal/2, long range at normal*1.5, very long range at normal*2, extreme range at normal*3. Where short is mod+1, long is mod-3 dam-1, vlong mod-6 dam-2, extreme mod-9 dam-3. Different weapons have different behaviour to varied ranges.

\item[rate of fire] is how fast the ranged weapon can be fired. Firing faster or slower carries modifiers in various ways depending on the weapon.

\item[maneuvers] are alternative actions that can be trained with weapons above the normal attack and parry actions. The maneuvers modify the weapon stats, e.g:
\begin{verbatim}
2h axe    dam 12, abs 11, parry-3, toparry-1, toavoid+1, finesse-3
          str 6
          swing: slow 1 mod-1 dam+3 toavoid+2
\end{verbatim}
Where the swing attack maneuver of that 2h axe would have total stats as follows:
\begin{verbatim}
2h axe swing    slow 1, mod-1, dam 15, toparry-1, toavoid+3
\end{verbatim}
Note: that toparry-1 is unmodified and the toavoid+3 is the sum of the weapon's default +1 and the swing's +2.

Optional weapon maneuvers like poke and swing usually cost 5xp to train.

\end{description}

\

\subsection*{Strength requirements}
%----------------------------------
Most regular one handed (1h) weapons are tailored for a minimum strength baseline of 4, with a smaller version for str 2, and a larger for str 6. Two handed (2h) weapons are intended for a strength baseline of 6, with a smaller version for str 4 and a larger version for str 8.

More unusual or speciality weapons can have drastically different strength requirements. They are more often custom made and can then be tailored for the specific wielder.

Military issue weapons tend to require a baseline 5, with possible alternate versions for str 3 and/or 7.

%\

% typical: relative base, 1h/2h  dam , extras                          str range
% knife    +1   +1/--                         fast, finesse               [1-4]
% sword    +2   +2/+4                         finesse 6                   [2-8]
% axe      +3   +3/+6                         parry-3 toparry-1 toavoid+1 [2-12]
% staff    -1   -1/+1 parry +1/+2             fast, reach                 [2-6]
% spear    +1   +1/+2 pen 0/1 parry -1/+1     reach                       [2-6]
% club     =0
% brawl    -3                                 (at str 6)
%   fist   -4                                 fast, deflect
%   kick   -2                                 deflect
%
% bow      +1   --/=0 pen 0-2                 range, pen, 1/2r
% crossbow +3   --/++ pen 1-3                 range, pen, 1/3r
%
% hammer   +1                                 slow, knockback



\

\subsection*{Melee weapons}
%melee weapons
%-------------
\raggedbottom
\goodbreak \small \begin{samepage} \begin{verbatim}
Regular Swords

small sword         dam 4, abs 6,
                    str 2 (max +1 str bonus), finesse-6
                    poke: mod-1 dam-1 pen+1
                    swing: slow 1 mod-1 dam+1 todefend+2
\end{verbatim} \blocklistgap \begin{verbatim}
sword               dam 6, abs 10,
                    str 4 (max +2 str bonus), finesse-6
                    poke: mod-1 dam-1 pen+1
                    swing: slow 1 mod-1 dam+2 todefend+2
\end{verbatim} \blocklistgap \begin{verbatim}
large sword         dam 8, abs 14,
                    str 6 (max +3 str bonus), finesse-6
                    poke: mod-1 dam-1 pen+1
                    swing: slow 1 mod-1 dam+2 todefend+2
\end{verbatim} \blocklistgap \begin{verbatim}
small 2h sword      dam 8, abs 12,
                    str 4 (max +2 str bonus), finesse-6
                    reach 1 mod-7
                    poke: mod-1 dam-1 pen+1
                    swing: slow 1 mod-1 dam+2 todefend+2
\end{verbatim} \blocklistgap \begin{verbatim}
2h sword            dam 10, abs 15,
                    str 6 (max +3 str bonus), finesse-6
                    reach 1 mod-6
                    poke: mod-1 dam-1 pen+1
                    swing: slow 1 mod-1 dam+2 todefend+2
\end{verbatim} \blocklistgap \begin{verbatim}
large 2h sword      dam 12, abs 18,
                    str 8 (max +4 str bonus), finesse-6
                    reach 1 mod-5
                    poke: mod-1 dam-1 pen+1
                    swing: slow 1 mod-1 dam+3 todefend+2
\end{verbatim} \end{samepage} \normalsize \goodbreak

\

\goodbreak \small \begin{samepage} \begin{verbatim}
Infantry Swords

shortsword          dam 5, abs 10,
                    str 3 (max +1 str bonus), finesse-4
                    poke: mod-1 dam-1 pen+1
                    swing: slow 1 mod-1 dam+1 todefend+2
\end{verbatim} \blocklistgap \begin{verbatim}
broadsword          dam 7, abs 14,
                    str 5 (max +2 str bonus), finesse-4
                    poke: mod-1 dam-1 pen+1
                    swing: slow 1 mod-1 dam+2 todefend+2
\end{verbatim} \blocklistgap \begin{verbatim}
longsword           dam 9, abs 18,
                    str 7 (max +3 str bonus), finesse-4
                    poke: mod-1 dam-1 pen+1
                    swing: slow 1 mod-1 dam+3 todefend+2
\end{verbatim} \end{samepage} \normalsize \goodbreak

\

\goodbreak \small \begin{samepage} \begin{verbatim}
Knives and fast blades

knife               dam 2, abs 2, parry-2, toparry-1, toavoid-2, finesse-9
                    str 1 (no str bonus)
                    fast 1 if str 3 and dex 4
                    every second attack in a round costs no stamina
                    poke: mod-1 dam-1 pen+1
                    gap: mod-abs dam-1 penetrating
\end{verbatim} \blocklistgap \begin{verbatim}
dagger              dam 3, abs 3, parry-1, toparry-1, toavoid-2, finesse-9
                    str 2 (no str bonus)
                    fast 1 if str 4 and dex 5
                    every second attack in a round costs no stamina
                    poke: mod-1 dam-1 pen+1
                    gap: mod-abs dam-1 penetrating
\end{verbatim} \blocklistgap \begin{verbatim}
rapier              dam 4, abs 4, toparry-1, toavoid-2, finesse-9
                    str 3 (no str bonus)
                    fast 1 if str 5 and dex 6
                    every second attack in a round costs no stamina
                    poke: mod-1 dam-1 pen+1
                    gap: mod-abs dam-1 penetrating
\end{verbatim} \blocklistgap \begin{verbatim}
large rapier        dam 5, abs 5, toparry-1, toavoid-2, finesse-9
                    str 4 (no str bonus)
                    fast 1 if str 6 and dex 7
                    every second attack in a round costs no stamina
                    poke: mod-1 dam-1 pen+1
                    gap: mod-abs dam-1 penetrating
\end{verbatim} \end{samepage} \normalsize \goodbreak

\

\goodbreak \small \begin{samepage} \begin{verbatim}
Speciality Blades   costs 5xp to learn speciality, or suffer mod-1

duelling sword      dam 6, abs 6, toparry-1, toavoid-2, finesse-9
                    str 5, (no str bonus)
                    fast 1 if str 7 and dex 9
                    every second attack in a round costs no stamina
                    poke: mod-1 dam-1 pen+1
                    swing: slow 1 mod-1 dam+1 todefend+2
\end{verbatim} \blocklistgap \begin{verbatim}
large duell. sword  dam 7, abs 7, toparry-1, toavoid-2, finesse-9
                    str 6, (no str bonus)
                    fast 1 if str 8 and dex 10
                    every second attack in a round costs no stamina
                    poke: mod-1 dam-1 pen+1
                    swing: slow 1 mod-1 dam+1 todefend+2
\end{verbatim} \blocklistgap \begin{verbatim}
parrying dagger     dam 3, abs 8, attack-1, parry+1, finesse-5
                    str 2 (no str bonus)
                    every second attack in a round costs no stamina
                    fast 1 if str 3 and dex 5
                    poke: mod-1 dam-1 pen+1
\end{verbatim} \blocklistgap \begin{verbatim}
baselard            dam 2, pen 1, abs 3, parry-3, toparry-1, finesse-6
                    str 2 (no str bonus)
                    fast 1 if str 3 and dex 5
                    every second attack in a round costs no stamina
                    poke: mod-1 dam=1 pen+2
\end{verbatim} \blocklistgap \begin{verbatim}
bastard sword       dam 7/9, abs 12, finesse-5/-6
(1h/2h)             str 5/5 (max +2/+3 str bonus)
                    poke: mod-1 dam-1 pen+1
                    swing: slow 1 mod-1 dam+2 todefend+2
\end{verbatim} \blocklistgap \begin{verbatim}
great sword         dam 10, abs 20, toavoid+1, finesse-4
1h                  4ap: str 6 (max dam+4 str bonus)
                    3ap: str 9 dex 6
                    poke: mod-1 dam-1 pen+1
                    swing: slow 1 mod-1 dam+3 todefend+2
\end{verbatim} \blocklistgap \begin{verbatim}
claymore            dam 14, abs 28, toavoid+1, finesse-4
2h                  4ap: str 8 (max dam+4 str bonus)
                    3ap: str 11 dex 8
                    reach 1 mod-6
                    poke: mod-1 dam-1 pen+1
                    swing: slow 1 mod-1 dam+3 todefend+2
\end{verbatim} \blocklistgap \begin{verbatim}
estoc               dam 7, abs 8, toparry-1, toavoid-1, finesse-9
2h                  3ap: str 5 (no str damage bonus, max +1 penetrating bonus)
                    2ap: str 7 and dex 9
                    every second attack in a round costs no stamina
                    poke: mod-1 dam-2 pen+2
                    swing: slow 1 mod-1 dam+2 todefend+2
\end{verbatim} \blocklistgap \begin{verbatim}
grande estoc        dam 9, abs 10, toparry-1, toavoid-1, finesse-9
2h                  3ap: str 7 (no str damage bonus, max +1 penetrating bonus)
                    2ap: str 9 and dex 11
                    every second attack in a round costs no stamina
                    poke: mod-1 dam-2 pen+2
                    swing: slow 1 mod-1 dam+2 todefend+2
\end{verbatim} \end{samepage} \normalsize \goodbreak

\

\goodbreak \small \begin{samepage} \begin{verbatim}
Regular Axes

small 1h axe        dam 5, abs 6, parry-3, toparry-1, toavoid+1, finesse-3
                    str 2 (max dam+2 str bonus)
                    swing: slow 1 mod-1 dam+2 toavoid+2
\end{verbatim} \blocklistgap \begin{verbatim}
1h axe              dam 7, abs 8, parry-3, toparry-1, toavoid+1, finesse-3
                    str 4 (max dam+4 str bonus)
                    swing: slow 1 mod-1 dam+2 toavoid+2
\end{verbatim} \blocklistgap \begin{verbatim}
large 1h axe        dam 9, abs 10, parry-3, toparry-1, toavoid+1, finesse-3
                    str 6 (max dam+6 str bonus)
                    swing: slow 1 mod-1 dam+3 toavoid+2
\end{verbatim} \blocklistgap \begin{verbatim}
small 2h axe        dam 10, abs 9, parry-3, toparry-1, toavoid+1, finesse-3
                    str 4 (max dam+6 str bonus)
                    swing: slow 1 mod-1 dam+3 toavoid+2
\end{verbatim} \blocklistgap \begin{verbatim}
2h axe              dam 12, abs 11, parry-3, toparry-1, toavoid+1, finesse-3
                    str 6 (max dam+8 str bonus)
                    swing: slow 1 mod-1 dam+3 toavoid+2
\end{verbatim} \blocklistgap \begin{verbatim}
large 2h axe        dam 14, abs 13, parry-3, toparry-1, toavoid+1, finesse-3
                    str 8 (max dam+10 str bonus)
                    swing: slow 1 mod-1 dam+4 toavoid+2
\end{verbatim} \end{samepage} \normalsize \goodbreak

\

\goodbreak \small \begin{samepage} \begin{verbatim}
Speciality Axes     costs 5xp to learn speciality, or suffer mod-1

battle axe          dam 8, abs 10, parry-2, toparry-2, finesse-4
1h                  str 5 (max dam+5 str bonus)
                    swing: slow 1 mod-1 dam+2 toavoid+2
                    poke: mod-2 dam-3 pen+2
\end{verbatim} \blocklistgap \begin{verbatim}
great axe           dam 12, abs 14, parry-4, toparry-3, toavoid+2, finesse-2
1h                  4ap: str 6 (max dam+6 str bonus)
                    3ap: str 9 dex 6
                    swing: slow 1 mod-1 dam+4 toavoid+2
\end{verbatim} \blocklistgap \begin{verbatim}
lochaber            dam 16, abs 20, parry-4, toparry-3, toavoid+2, finesse-2
2h                  4ap: str 8  (max dam+12 str bonus)
                    3ap: str 11 dex 8
                    swing: slow 1 mod-1 dam+4 toavoid+2
\end{verbatim} \blocklistgap \begin{verbatim}
pick axe            dam 4, pen 4, parry-3, toavoid+2, finesse-2
2h                  4ap: str 4  (max dam+2 pen+4 str bonus)
                    double damage against stone, doors, etc
                    swing: slow 1 mod-1 dam+1, pen+1, toavoid+2
\end{verbatim} \blocklistgap \begin{verbatim}
large pick axe      dam 6, pen 4, parry-3, toavoid+2, finesse-2
2h                  4ap: str 6 (max dam+3 pen+5 str bonus)
                    double damage against stone, doors, etc
                    swing: slow 1 mod-1 dam+1, pen+1, toavoid+2
\end{verbatim} \end{samepage} \normalsize \goodbreak

\

\goodbreak \small \begin{samepage} \begin{verbatim}
Staffs and Spears
spear slash maneuver costs 5xp and does slashing instead of piercing damage,
but with dam-1 modifier.

Spears, pikes and pole arms with enough reach can be used to attack over
the shoulder or shield of a friend in a two layer formation. This requires
phalanx of both characters. A three layer formation is possible if the
first front line is kneeling or formed of smaller characters.

Heavy versions of staffs, spears, and pole arms have
dam+2 abs+4 finesse-1 for str+3.
OR add additional slow 1 (ap+1) without changing str requirements.
Also raises max str bonus by +1 for staffs,
and and max str and pen bonus by +1 for spears.
Heavy staffs also raises the fast 1 requirements for with additional
str+3 and dex+6 (total str+6 dex+6).

narrow tip spears are dam-1 pen+1 mod-1
narrow tip spears can have all str bonus as penetrating
broad tip spears are dam+1 mod-1

small staff         dam 1/2, abs 4, parry+1/+2, reach 1/0 mod-4, finesse-3/-5
1h/2h               str 2/1 (max +1/+2 str bonus)
                    2h: fast 1 if str 4 and dex 4
\end{verbatim} \blocklistgap \begin{verbatim}
staff               dam 3/4, abs 6, parry+1/+2, reach 1/0 mod-4, finesse-3/-5
1h/2h               str 4/3 (max +1/+2 str bonus)
                    2h: fast 1 if str 6 and dex 6
\end{verbatim} \blocklistgap \begin{verbatim}
large staff         dam 5/6, abs 8, parry+1/+2, reach 1/0 mod-4, finesse-3/-5
1h/2h               str 6/5 (max +1/+2 str bonus)
                    2h: fast 1 if str 8 and dex 8
\end{verbatim} \blocklistgap \begin{verbatim}
small spear         dam 3/3, pen 0/1, abs 4, parry-1/+1, reach 1 mod-3
1h/2h               str 2/1 (str bonus max +1 dam then max +1 pen)
                    2h: fast 1 if str 6 and dex 6
                    slash: mod-2, dam-1
                    finesse-4/-6
\end{verbatim} \blocklistgap \begin{verbatim}
spear               dam 5/5, pen 0/1, abs 4, parry-1/+1, reach 1 mod-3
1h/2h               str 4/3 (str bonus max +1 dam then max +2 pen)
                    slash: mod-2, dam-1
                    finesse-4/-6
\end{verbatim} \blocklistgap \begin{verbatim}
large spear         dam 7/7, pen 0/1, abs 6, parry-1/+1, reach 1 mod-3
1h/2h               str 6/5 (str bonus max +1 dam then max +3 pen)
                    slash: mod-2, dam-1
                    finesse-4/-6
\end{verbatim} \blocklistgap \begin{verbatim}
long spear          dam 5, pen 1, abs 5, parry-3, reach 1 mod-0
2h                  str 4  (str bonus max +1 dam then max +2 pen)
                    finesse-3
\end{verbatim} \blocklistgap \begin{verbatim}
large long spear    dam 7, pen 1, abs 7, parry-3, reach 1 mod-0
2h                  str 6 (str bonus max +1 dam then max +3 pen)
                    finesse-3
\end{verbatim} \blocklistgap \begin{verbatim}
infantry spear      dam 6/7, pen 0/1, abs 9, parry-1/+1, reach 1 mod-3
1h/2h               str 5 (str bonus max +1 dam then max +3 pen)
                    slash: mod-2, dam-1
                    finesse-3/-5
\end{verbatim} \end{samepage} \normalsize \goodbreak

\

\goodbreak \small \begin{samepage} \begin{verbatim}
Speciality Pole Arms costs 5xp to learn speciality, or suffer mod-1
They are usually military infantry issue and thus tailored for str 5.

pike                dam 6, pen 2, abs 8, parry-6, slow 1
2h                  reach 0 mod-3, reach 1 mod-0, reach 2 mod-3
                    str 5 (max dam+2 str bonus, rest is penetrating)
                    finesse-2
\end{verbatim} \blocklistgap \begin{verbatim}
halberd             dam 11, abs 8, parry-6, slow 2 (5ap), toavoid+2
2h                  reach 0 mod-2, reach 1 mod-0, reach 2 mod-5,
                    str 5 (max dam+5 str bonus)
                    finesse-2
                    5ap normal attack (slashing)
                    swing: 6ap mod-1 dam 14 toavoid++2 (slashing)
                    poke: 4ap mod-1 dam 6 pen 2 (piercing), reach 2 mod-4
                    Halberds can be used either with axe or spear skill.
\end{verbatim} \blocklistgap \begin{verbatim}
glave               dam 9, abs 8, parry-3, slow 1, toavoid+1
2h                  reach 0 mod-1, reach 1 mod-0, reach 2 mod-6,
                    str 5 (max dam+4 str bonus)
                    finesse-4
                    Glaves can be used either with sword or spear skill.
\end{verbatim} \end{samepage} \normalsize \goodbreak

\

\goodbreak \small \begin{samepage} \begin{verbatim}
Clubs, Hammers, Flails
attack maneuver swing costs 5xp

1h hammers and 2h clubs have "knock down" attacks at mod-2 instead of mod-3.
2h hammers and heavy 2h clubs have "knock down" attacks at mod-1.

Flails have "trip up" attacks at mod-2 instead of mod-3
Long chain flails have "trip up" at mod-1.

small club          dam 2/4, abs 6, parry-1, toavoid+1, finesse-2
1h/2h               str 2 (max +2/+3 str bonus)
                    1h: str vs str for knockback 1
                    2h: str+3 vs str for knockback 1
                    swing: slow 1 mod-1 dam+2 toavoid+2 knockback roll +3
\end{verbatim} \blocklistgap \begin{verbatim}
club                dam 4/6, abs 10, parry-1, toavoid+1, finesse-2
1h/2h               str 4 (max +3/+4 str bonus)
                    1h: str vs str for knockback 1
                    2h: str+3 vs str for knockback 1
                    swing: slow 1 mod-1 dam+2 toavoid+2 knockback roll +3
\end{verbatim} \blocklistgap \begin{verbatim}
large club          dam 6/8, abs 14, parry-1, toavoid+1, finesse-2
1h/2h               str 6 (max +3/+4 str bonus)
                    1h: str vs str for knockback 1
                    2h: str+3 vs str for knockback 1
                    swing: slow 1 mod-1 dam+2 toavoid+2 knockback roll +3
\end{verbatim} \blocklistgap \begin{verbatim}
very large club     dam 8/10, abs 18, parry-1, toavoid+1, finesse-2
1h/2h               str 8  (max +3/+4 str bonus)
                    1h: str vs str for knockback 1
                    2h: str+3 vs str for knockback 1
                    swing: slow 1 mod-1 dam+3 toavoid+2 knockback roll +3
\end{verbatim} \blocklistgap \begin{verbatim}
monstrous club      dam 10/12, abs 22, parry-1, toavoid+1, finesse-2
1h/2h               str 10  (max +3/+4 str bonus)
                    1h: str vs str for knockback 1
                    2h: str+3 vs str for knockback 1
                    swing: slow 1 mod-1 dam+3 toavoid+2 knockback roll +3
\end{verbatim} \blocklistgap \begin{verbatim}
heavy clubs         dam+2, abs+50%, parry-3, toavoid+2, finesse-1, slow 1
                    str requirement +3
                    knockback roll +3
\end{verbatim} \end{samepage} \normalsize \goodbreak

\

\goodbreak \small \begin{samepage} \begin{verbatim}
1h hammer           dam 6, abs 6, parry-3, toavoid+1, finesse-2, slow 1
                    str 4 (max +4 str bonus)
                    str+3 vs str for knockback 1
                    swing: slow 1 mod-1 dam+2 toavoid+2 knockback+1 (on success)
\end{verbatim} \blocklistgap \begin{verbatim}
large 1h hammer     dam 8, abs 8, parry-3, toavoid+1, finesse-2, slow 1
                    str 6 (max +6 str bonus)
                    str+3 vs str for knockback 1
                    swing: slow 1 mod-1 dam+2 toavoid+2 knockback+1 (on success)
\end{verbatim} \blocklistgap \begin{verbatim}
2h hammer           dam 8, abs 16, parry-3, toavoid+1, finesse-2, slow 1
                    str 6
                    knockback 1
                    swing: slow 1 mod-1 dam+3 toavoid+2 knockback+1
\end{verbatim} \blocklistgap \begin{verbatim}
large 2h hammer     dam 10, abs 20, parry-3, toavoid+1, finesse-2, slow 1
                    str 8
                    knockback 1
                    swing: slow 1 mod-1 dam+3 toavoid+2 knockback+1
\end{verbatim} \end{samepage} \normalsize \goodbreak

\

\goodbreak \small \begin{samepage} \begin{verbatim}
1h flail            dam 4, abs 8, parry-6, toparry-3, finesse-6
                    str 4 (max +2 str bonus), slow 1
                    reach 1 mod-3, snag-3
                    swing: slow 1 mod-1 dam+1 toavoid+2
\end{verbatim} \blocklistgap \begin{verbatim}
large 1h flail      dam 6, abs 12, parry-6, toparry-3, finesse-6
                    str 6 (max +4 str bonus), slow 1
                    reach 1 mod-3, snag-3
                    swing: slow 1 mod-1 dam+2 toavoid+2
\end{verbatim} \blocklistgap \begin{verbatim}
2h flail            dam 8, abs 16, parry-6, toparry-3, finesse-6
                    str 6, slow 1
                    reach 1 mod-3, snag-3
                    swing: slow 1 mod-1 dam+3 toavoid+2
\end{verbatim} \blocklistgap \begin{verbatim}
large 2h flail      dam 10, abs 20, parry-6, toparry-3, finesse-6
                    str 8, slow 1
                    reach 1 mod-3, snag-3
                    swing: slow 1 mod-1 dam+3 toavoid+2
\end{verbatim} \end{samepage} \normalsize \goodbreak

\

\goodbreak \small \begin{samepage} \begin{verbatim}
Speciality Bludgeoning Weapons   costs 5xp to learn speciality, or suffer mod-1

sledge hammer       dam 10, abs 20, parry-6, toavoid+3
1h                  str 7, slow 2
                    knockback 2
                    swing: slow 1 mod-1 dam+2 toavoid+2 knockback+1
\end{verbatim} \blocklistgap \begin{verbatim}
maul                dam 14, abs 25, parry-6, toavoid+3
2h                  str 10, slow 2
                    knock back 2
                    slow-3 (6ap): knockback 3 costs one extra stamina
                    swing: slow 1 mod-1 dam+3 toavoid+2 knockback+1
\end{verbatim} \blocklistgap \begin{verbatim}
morning star        dam 6, pen 1, abs 6, parry-3, toavoid+1, finesse-3
                    str 5
                    str vs str for knock back 1
                    swing: slow 1 mod-1 dam+2 toavoid+2 knockback roll +3
\end{verbatim} \blocklistgap \begin{verbatim}
2h morning star     dam 9, pen 2, abs 9, parry-3, toavoid+1, finesse-3
                    str 7
                    str+3 vs str for knock back 1
                    swing: slow 1 mod-1 dam+3 toavoid+2 knockback roll +3
\end{verbatim} \blocklistgap \begin{verbatim}
long chain flail    dam 7, abs 7, parry-6, toparry-6, toavoid-3, finesse-9
                    str 5, slow 2
                    reach 1 mod-0, snag-0
                    swing: slow 1 mod-1 dam+2 toavoid+2
\end{verbatim} \blocklistgap \begin{verbatim}
2h long chain flail dam 10, abs 10, parry-6, toparry-6, toavoid-3, finesse-9
                    str 7, slow 2
                    reach 1 mod-0, snag-0
                    swing: slow 1 mod-1 dam+3 toavoid+2
\end{verbatim} \end{samepage} \normalsize \goodbreak

\


\subsection*{Shields and armour}
%shields and armour
%------------------

\small \begin{samepage} \begin{verbatim}
Shields can be carried on one arm, on the back, or on a backpack.

Heavy version of shields have abs+50% but are either
slow 1 (4ap) or require str+3 and equal dex to retain normal speed

Plated heavy shields have double abs but are either
slow 2 (5ap) or require str+3 and equal dex to be slow 1 (4ap)
or str+6 and equal dex to be normal speed (3ap).

2h shields have -2 strength requirements.

Small characters (halflings, goblins) have extra mod-1 to incoming ranged
attacks when shield is "in the way" or when they are "hiding behind"
the shield.

buckler             abs 6, parry+1
                    str 0
                    fast 1 if str 4 and dex 6 and fast shield 1
                    Ranged attacks mod-0 when in the way.
                    Hiding behind it (3ap) ranged mod-1
\end{verbatim} \blocklistgap \begin{verbatim}
small shield        abs 8, parry+2,
                    str 2
                    fast 1 if str 6 and dex 9 and fast shield 2
                    Ranged attacks mod-1 when in the way.
                    Hiding behind it (3ap) ranged mod-2
\end{verbatim} \blocklistgap \begin{verbatim}
shield              abs 10, parry+3,
                    str 4, or slow 1 str 1
                    fast 1 if str 8 and dex 12 and fast shield 3
                    Ranged attacks mod-2 when in the way.
                    Hiding behind it (3ap) ranged mod-4
                    tackle mod+1
\end{verbatim} \blocklistgap \begin{verbatim}
large shield        abs 12, parry+4,
                    str 6, or slow 1 str 3
                    Ranged attacks mod-3 when in the way.
                    Hiding behind it (3ap) ranged mod-6
                    tackle mod+2
\end{verbatim} \blocklistgap \begin{verbatim}
tower shield        abs 14, parry+5,
                    str 8, or slow 1 str 5
                    Ranged attacks mod-4 when in the way.
                    Hiding behind it (3ap) ranged mod-8
                    tackle mod+3
\end{verbatim} \blocklistgap \begin{verbatim}
infantry shield     abs 13, parry+3,
                    str 5
                    Ranged attacks mod-2 when in the way.
                    Hiding behind it (3ap) ranged mod-4
                    tackle mod+1
\end{verbatim} \end{samepage} \normalsize \goodbreak

\

\goodbreak \small \begin{samepage} \begin{verbatim}
Armour movement penalties will leave effective character movement
at least:      Maneuver=1, Walk=2, Run=3, Dash=4

leather armour abs 1 mostly like thick clothing made to take a punch
thick cloth    str 1 (str penalties affect all actions)
               max stamina -1
               dash-1
               acrobatics mod-1
               martial arts mod-1
               sneak mod-1
               spellcasting mod-1
               swim mod-1
               takes 2 rounds to put on or take off
\end{verbatim} \blocklistgap \begin{verbatim}
chain mail     abs 2 ring, scale, brigantine, etc
scale mail     str 3 (str penalties affect all actions)
brigandine     max stamina -2
               run-1, dash-2
               dex-1
               per-1
               acrobatics mod-3
               martial arts mod-2
               spellcasting mod-2
               sneak mod-3
               swim mod-3
               takes 3 rounds to put on or take off
\end{verbatim} \blocklistgap \begin{verbatim}
plate mail or  abs 3 plate armour of various sorts, incl helmet
heavy scale,   str 5 (str penalties affect all actions)
h brigandine   max stamina -3
               run-1, dash-3
               dex-2, yield bonus -1
               per-2, vision-20%, cone vision -20% max 270 deg
               acrobatics mod-6, climb-1, jump-1
               martial arts mod-3
               spellcasting mod-3
               sneak mod-4
               swim mod-4
               avoid mod-1
               Turning as separate actions cost more ap:
               45deg=0ap, 90deg=1ap, 135deg=2ap, 180deg=3ap
               (hex: 60deg=1ap, 120deg=2ap, 180deg=3ap)
               takes 4 rounds to put on or take off
\end{verbatim} \blocklistgap \begin{verbatim}
full plate     abs 4 full plate armour with full helmet
               str 7 (str penalties affect all actions)
               max stamina -4
               walk-1, run-2, dash-4
               dex-3, yield bonus -2
               per-3, vision-30%, cone vision -30% max 180 deg
               acrobatics mod-9, climb-3, jump-3
               martial arts mod-6
               tackle mod+1
               spellcasting mod-4
               sneak mod-5
               swim mod-5
               avoid mod-2
               Turning as separate action cost more ap:
               45deg=1ap, 90deg=2ap, 135deg=3ap, 180deg=3ap
               (hex: 60deg=1ap, 120deg=3ap, 180deg=3ap)
               takes 6 rounds to put on or take off
\end{verbatim} \blocklistgap \begin{verbatim}
heavy plate    abs 5 very heavy full plate armour with all the trimmings.
               str 9 (str penalties affect all actions)
               max stamina -5
               maneuver-1, walk-2, run-3, dash-5
               dex-4, yield bonus -3
               per-4, vision-40%, cone vision -40% max 120 deg
               acrobatics mod-12, climb-6, jump-6
               martial arts mod-9
               tackle mod+2
               spellcasting mod-6
               sneak mod-6
               swim mod-6
               avoid mod-3
               Turning as separate action cost more ap:
               45deg=1ap, 90deg=2ap, 135deg=3ap, 180deg=4ap
               (hex: 60deg=1ap, 120deg=3ap, 180deg=4ap)
               takes 10 rounds to put on or take off
\end{verbatim} \end{samepage} \normalsize \goodbreak

\

Custom armour can have different stats, but keep in mind that an extra +1 abs is a very potent and heavy customisation. Better to start at the given abs and look at what penalty mods you want to change for the custom armour. E.g: an expensive darkened matte black leather armour without the sneak mod, for a much higher price.


\subsection*{Ranged weapons}
%---------------------------
Are the monsters a little too far away for your sword? Are you just a bit too lazy to walk up and say hi with your huge axe, or perhaps a tad shy? Then why not send a friendly murder greeting with a crossbow or throw them a heart felt javelin?

\

Regarding range: Do not think of range as "how long you can shoot or throw", but how far you can \emph{accurately} shoot or throw.

%\TODO:
\todo: Change "range" stat to "accuracy", and actual range and damage / penetration reduction as the multiples of accuracy the weapon can be used at, or separate damage related range factor.

\

\noindent Thrown weapons include anything from knives and javelins to improvised missiles like beer bottles, rocks, and mistuned musical instruments.

Thrown weapons can also be used as poor versions of similar melee weapons in a pinch and with some mods. Regular melee weapons can often also be used as thrown weapons but with penalty mods and reduced damage. When a thrown weapon is used as a temporary melee weapon it uses the melee weapon class skill, and when a melee weapon is used as a thrown weapon it uses the throw skill.

\

%\todo rewrite throwing weapons  -  210130 sort of ok now
\goodbreak \small \begin{samepage} \begin{verbatim}
Throwing an already held weapon is a regular 3ap action. Drawing and throwing
in one go is a 1r action. Faster throwing gives mods which can be mitigated by
quick shot. Manually drawing with quickdraw then throwing as already held might
be better if the user is skilled in quickdraw.

Draw and throw:
normal  1r  mod=0
aimed   2r  mod+1
quick   3ap mod-3, 2ap mod-6, 1ap mod-9

Throwing already drawn weapon:
normal  3ap     mod=0
aimed   6ap/1r  mod+1
quick   1ap     mod-3
\end{verbatim} \blocklistgap \begin{verbatim}
Thrown weapons have normal range tohit and damage mods:
short       mod+1          0.5 * normal range
long        mod-3 dam-1    1.5 * normal range
very long   mod-6 dam-2    2.0 * normal range
extreme     mod-9 dam-3    3.0 * normal range
\end{verbatim} \blocklistgap \begin{verbatim}
throwing dart     dam 1 pen 2 : OR : mod-(target abs) dam 1 penetrating
throwing star     range 8 + max(str,dex)/3
                  long range: pen-1, vlong pen-2, extreme range not possible
                  OR if mod by abs: double range mod
                  str 2, no str bonus
                  every second attack require no stamina
\end{verbatim} \blocklistgap \begin{verbatim}
throwing knife    dam 2 : OR : mod-(target abs) dam 2 penetrating
                  range 6 + max(str,dex)/3
                  long range: dam-1, vlong dam-2, extreme range not possible
                  OR if mod by abs: double range mod
                  str 2, no str bonus
                  every second attack require no stamina
                  melee: knife mod-1, dam 2, abs 2, finesse-3
                         fast 1 if str 2 and dex 4
                         every second attack require no stamina
\end{verbatim} \blocklistgap \begin{verbatim}
throwing dagger   dam 3 : OR : mod-(target abs) dam 3 penetrating
                  range 6 + max(str,dex)/3
                  long range: dam-1, vlong dam-2, extreme range not possible
                  OR if mod by abs: double range mod
                  str 3, no str bonus
                  every second attack require no stamina
                  melee: knife mod-1, dam 3, abs 3, finesse-3
                         fast 1 if str 2 and dex 4
                         every second attack require no stamina
\end{verbatim} \blocklistgap \begin{verbatim}
small throwing    dam 4
axe               range 5 + str/3
                  str 2 (str bonus: dam+1)
                  melee: mod-1, dam 4, abs 3, parry-4, toparry-1, toavoid+1,
                         finesse-2
                         str 2 (str bonus: dam+1)
                         swing: slow 1 mod-1 dam+2 toavoid+2
\end{verbatim} \blocklistgap \begin{verbatim}
throwing axe      dam 6
                  range 5 + str/3
                  str 4 (str bonus: dam+2)
                  melee: mod-1, dam 6, abs 5, parry-4, toparry-1, toavoid+1,
                         finesse-2
                         str 4 (str bonus: dam+2)
                         swing: slow 1 mod-1 dam+2 toavoid+2
\end{verbatim} \blocklistgap \begin{verbatim}
large throwing    dam 8
axe               range 5 + str/3
                  str 6 (str bonus: dam+3)
                  melee: mod-1, dam 8, abs 6, parry-4, toparry-1, toavoid+1,
                         finesse-2
                         str 6 (str bonus: dam+3)
                         swing: slow 1 mod-1 dam+2 toavoid+2
\end{verbatim} \blocklistgap \begin{verbatim}
javelin           dam 3, penetrating 1
                  range 8 + str/2
                  str 4 (str bonus: dam+1, pen+1)
                  melee: spear mod-1, dam 3, abs 3, reach 1 mod-5, finesse-3
                         str 4 (str bonus: dam+1, pen+1)
\end{verbatim} \blocklistgap \begin{verbatim}
large javelin     dam 5, penetrating 1,
                  range 8 + str/2
                  str 6 (str bonus: dam+1, pen+2)
                  melee: spear mod-1, dam 5, abs 5, reach 1 mod-5, finesse-3
                         str 6 (str bonus: dam+1, pen+2)
\end{verbatim} \blocklistgap \begin{verbatim}
improvised:

stones            dam = 1+str/3, range = 4 + str/3
                  select stones to trade between damage and range:
                  trade dam-1 for range+3
\end{verbatim} \blocklistgap \begin{verbatim}
bottle / keg / flask / stool / chair / kettle / skillet
                  mod-1 to mod-3 depending on how unwieldy
                  dam = 1+str/3, range = 3 + str/3
                  larger objects force trade range for damage: range-2 dam+1
\end{verbatim} \blocklistgap \begin{verbatim}
examples of throwing regular melee weapons, still roll for throw:
knife: mod-1, dam-1, range 4 + max(str,dex)/3
sword: mod-3, dam-3, range 2 + str/3
axe:   mod-2, dam-2, range 4 + str/3
spear: mod-1, dam-1, range 6 + str/2
club:  mod-2, dam-2, range 3 + str/3
\end{verbatim} \end{samepage} \normalsize \goodbreak

\


\noindent Bows, crossbows, arbalests, and ballistas use arrows, bolts, darts, javelins, spears, or stones as ammunition. All rate of fire info below assumes that arrows are available from a readied easy to reach quiver, stuck in the ground, or similar. The ammo container does not have to be in a quickdraw slot for the ammo to be drawn quickly without rolls.

If you choose the optional rule that characters need to keep track of their ammunition instead of just assuming "enough arrows" it might be worth investing some xp in arrow recovery.

Bows and crossbows have one less damage reduction on long ranges: Long dam-0, vlong dam-1, extreme dam-2.

\

\goodbreak \small \begin{samepage} \begin{verbatim}
normal rate of fire for bows:
this includes reload from accessible quiver.
normal mod=0 2r
aimed  mod+1 3r
quick  mod-3 1r / 9ap

snap / rapid fire for bows:
Quick shot is 1r/9ap at mod-3. For every ap reduced below quick shot the
archer takes another mod-1. E.g: rapid fire bow at 4ap thus takes mod-8,
since quick is mod-3 and another mod-5 to reduce 9ap to 4ap.
An archer cannot bring the rapid fire ap cost below 3ap, mod-9.
Rapid fire from bows include reload from accessible quiver as normal,
arrows stuck in the ground, or similar ammo arrangement. It does not need
quick draw
\end{verbatim} \blocklistgap \begin{verbatim}
small bow         dam 3,
                  range 12
                  str 2 (no str bonus)
\end{verbatim} \blocklistgap \begin{verbatim}
bow               dam 4, pen 1,
                  range 14
                  str 4 (no str bonus)
\end{verbatim} \blocklistgap \begin{verbatim}
large bow         dam 5, pen 2,
                  range 16
                  str 6 (no str bonus)
\end{verbatim} \blocklistgap \begin{verbatim}
The longbows and shortbows are military infantry or cavalry issue.
Tailored for str 5 baseline and focus on range instead of damage,
though often used with broadhead or piercing arrows to increase the damage or
penetration based on the composition of the enemy forces.

shortbow          dam 3
                  range 15
                  str 3 (no str bonus)
                  mods when rapid fire improved by 1
                  mod limitations from ride improved by 1
\end{verbatim} \blocklistgap \begin{verbatim}
longbow           dam 4, pen 1,
                  range 20
                  str 5 (no str bonus)
\end{verbatim} \blocklistgap \begin{verbatim}
large longbow     dam 5, pen 2,
                  range 25
                  str 7 (no str bonus)
\end{verbatim} \blocklistgap \begin{verbatim}
Bows allow the use of different kinds of arrows:
regular arrows    no mods
cheap arrows      mod-1 dam-1, 50% reusable
piercing arrows   mod-1 dam-1 pen+1
broadhead arrows  mod-1 dam+1 pen-1
fine arrows       mod+1
barbed arrows     dam+1
sharp arrows      pen+1
\end{verbatim} \blocklistgap \begin{verbatim}
fire arrows       mod-2 dam-1 range-25%
                  oil splash dam 3/r for 5r
                  light source as fire, burns 5r
                  ignite 1a with fire source
torch arrows      mod-2 dam-2 range-25%
                  light source as torch, burns 10r
                  ignite 1a with fire source
candle arrows     mod-2 dam-2 range-25%
                  light source as candle, burns 20r
                  ignite 1a with fire source
\end{verbatim} \blocklistgap \begin{verbatim}
Most regular arrows can be recovered and reused of they strike a soft target.
Speciality arrows often cannot be reused.
Special bows sometimes require special arrows.

Heavy bows tend to have str+3, dam+1, pen+1, cannot use aimed shot
with special heavy hardened arrows they get another pen+1
with normal arrows they take mod-1
\end{verbatim} \end{samepage} \normalsize \goodbreak

\




%===============================================================================
\vspace{20mm}
\tmpsepline
%TODO:
\TODO: switch to named style bows:
\small \begin{verbatim}
archery bow   precision   low str dam
hunting bow   regular    normal and large
war bow       heavy     normal and large  no aim

cavalry bow     shortbow style   ride+3 cap
longbow         mods per sq movement declared  long range accuracy
field bow       mod-3 infinite range, rapid BARRAGE, heavy damage no aim no move
\end{verbatim} \normalsize
\tmpsepline
\vspace{20mm}
%===============================================================================




\

\goodbreak \small \begin{samepage} \begin{verbatim}
Crossbows tend to be military infantry issue, and tailored for a str 5 baseline.

Attacking with a crossbow requires no stamina, but reloading costs 1 stamina per
round. E.g: reloading a crossbow requires a total of 3 stamina, 1 per round for
three rounds. But only 2 stamina if the loader is strong enough for a 2r fast
reload.
With quickdraw it's possible to effectively support a good shooter with a
couple of strong loaders and a bunch of crossbows. Handing the spent weapons
over for loading and receiving ready to fire ones back in return.

Rate of fire for crossbows and arbalest:
Split between reload and fire actions.
Shooting a loaded crossbow is reasonably fast. Reloading takes a while.
The reload times assume readily available bolts in a quiver, stuck in ground,
or similar.

fire actions:
normal  mod=0 1r
aimed   mod+1 2r
quick   mod-3 3ap, mod-6 2ap, mod-9 1ap

crossbows have no str bonus
\end{verbatim} \blocklistgap \begin{verbatim}
small crossbow    dam 5, penetrating 1,
                  range 12
                  carried: str 3
                  mounted: str 1 reload +1r
                  reload 2r (str 9 dex 6 : 1r)
\end{verbatim} \blocklistgap \begin{verbatim}
crossbow          dam 6, penetrating 2,
                  range 15
                  carried: str 5
                  mounted: str 3 reload+1r
                  reload 3r (str 12 dex 8 : 2r)
\end{verbatim} \blocklistgap \begin{verbatim}
large crossbow    dam 7, penetrating 3,
                  range 18
                  carried: str 7
                  mounted: str 5 reload+1r
                  reload 4r (str 15 dex 10 : 3r)
\end{verbatim} \blocklistgap \begin{verbatim}
arbalest          broadhead: dam 12
(special)         spikehead: dam 9 pen 5
                  blunthead: dam 10, knockback 1
                  range 20
                  carried: str 9
                  mounted: str 5 reload+1r
                  reload 5r (str 18 dex 12 : reload 4r if custom built)
                  quick shot is +1ap, above normal crossbow mods
\end{verbatim} \blocklistgap \begin{verbatim}
mounted arbalest  broadhead: dam 18
(special)         spikehead: dam 14 penetrating
                  blunthead: dam 15 knockback 2
                  range 22
                  mounted: two man team, str 5, reload 6r
                  mounted: single man, str 8, reload 8r
                  carried: single man, str 12, reload 7r
                      (str 21 dex 14 : reload 6r if custom built)
                  quick shot is +2ap, above normal crossbow mods
The mounted arbalest is 8 enc: 4 for the weapon and 4 for the tripod.
They are not meant to be carried around. Usually used by a two man team.
\end{verbatim} \blocklistgap \begin{verbatim}
double type       as the regular crossbows except that they have two first
crossbows         shots, which can be used at the same time or separately.
(exotic)          They also require str+2, and take 1r longer to reload.
                  If both shots are fired at the same time (same action)
                  you should still roll each bolt separately.
\end{verbatim} \blocklistgap \begin{verbatim}
repeater type     as the regular crossbows except that they take one round
crossbows         less to reload, but at str+1, and have a bolt magazine
(exotic)          which when empty must be reloaded manually taking
                  2r + 1r / bolt in the magazine.
                  These machines are exceptionally rare and expensive,
                  manufactured only by legendary mechanismiths and
                  enginartificers.
\end{verbatim} \blocklistgap \begin{verbatim}
Some crossbows allow the use of speciality bolts:
fine bolts        mod+1
barbed bolts      dam+1
sharp bolts       pen+1
piercing bolts    mod-1 dam-1 pen+1
broadhead bolts   mod-1 dam+1 pen-1
\end{verbatim} \end{samepage} \normalsize \goodbreak

\

\goodbreak \small \begin{samepage} \begin{verbatim}
rate of fire for ballistas:
Ballistas take very long to reload and require at least two people.
Aiming and firing the ballista is still a cumbersome thing to do.

Firing loaded weapon:
normal  3r mod=0
aimed   4r mod+1
quick   2r mod-3
snap    1r mod-6
blind   3ap mod-9
\end{verbatim} \blocklistgap \begin{verbatim}
Ballistas are designed to shoot at large stationary objects far away
and have different range tohit and damage mods:
short       0.5 mod+1 dam+10%
long        2.0 mod-3 dam-10%
very long   4.0 mod-6 dam-15%
extreme     8.0 mod-9 dam-25%
\end{verbatim} \blocklistgap \begin{verbatim}
light ballista    dam 20, penetrating, knockback 1
                  range 20
                  mounted: str 3, carried: str 20, no str bonus
                  reload 10r, tot str 15, max 3ppl (only 1 person +5r)
\end{verbatim} \blocklistgap \begin{verbatim}
ballista          dam 30, penetrating, knockback 2
                  range 30
                  mounted: str 5, carried: str 50, no str bonus
                  reload 15r, tot str 20, max 4ppl (only 1 person +5r)
\end{verbatim} \blocklistgap \begin{verbatim}
heavy ballista    dam 40, penetrating, knockback 3
                  range 40
                  mounted: str 7, carried: str 100, no str bonus
                  reload 20r, tot str 25, max 5ppl (only 1 person +5r)
\end{verbatim} \end{samepage} \normalsize \goodbreak

\

\subsection*{custom weapons}
%---------------------------
It's always possible to have a craftsman build customised weapons that are suited specifically to the user's strength, dexterity, abs requirements, etc. These weapons cost a lot more, usually several times the normal list price, and require a skilled craftsman and sometimes some special materials.

The general guidelines for weapon classes look something like this:\\
\small \begin{samepage} \begin{verbatim}
1h blade:
    dam str+2, abs 2*dam-2
2h blade:
    dam str+4, abs 2*dam-2
fast blade:
    (3ap:) dam str+1, abs dam+0
    (2ap:) fast 1 if str>=dam+0 dex>=dam+2
\end{verbatim} \blocklistgap \begin{verbatim}
1h axe:
    dam str+3, abs dam+1
2h axe:
    dam str+6, abs dam
\end{verbatim} \blocklistgap \begin{verbatim}
staff (1h/2h):
    dam str-1/str+1, abs 2h dam+2
spear (1h/2h):
    dam str+0/str+1 and pen 1/2, abs 2h dam
\end{verbatim} \blocklistgap \begin{verbatim}
club (1h/2h):
    dam str+0/+2, abs 2*str+2, knockback 1 str-vs-str
\end{verbatim} \end{samepage} \normalsize \goodbreak

\

Heavy weapons are common customisations. They usually mean a slow 1 with damage and other stats as though a str+2 was applied but with unchanged str requirement, OR, increasing requiring str+3. Usually with significant increase in abs, +30-50\%. So heavy weapons loose 1 str damage efficiency for the benefit of very high abs.

Customised weapons should at least be treated as special equipment, with the reduced availability. Significant customisations might even push them to exotic. They usually need to be ordered at least a week in advance. Common customisations might be available off the shelf in large well stocked city shops.

Typical vanity items like the back seat general's heavy longsword, with flame decorations, have a reasonable chance of being available off the shelf even in Trade Town.

\


\subsection*{monster weapons}
%----------------------------
Some races, as NPCs, will sometimes use special versions of weapons which are different than those human craftsmen produce.

\

\noindent Goblins make shitty but evil weaponry, usually spears, knives, and axes. Low abs, poor durability, low str but high dex requirements to keep speed up. Market price is non existent, and goblin stuff requires some getting used to.

The poor durability gives low abs and perhaps a 10\% roll to just break at some point during the fight. Suggestion: every 10+1d10r roll for durability. If fail then deduct 1 abs. When the weapon reaches abs 0 it will break at next attack/parry, or just fall apart by itself.

All goblin weapons require speciality "gobliny" 5xp or suffer mod-1.
All npc goblins have this speciality innately.
Hero goblins who start the game with goblin weaponry also get it for free.
The speciality generally give two attack options for gobliny weapons:
a slow attack at higher damage, or a faster attack at normal damage.
At very high dex the the fast attack also get the higher damage.

\

\small \begin{samepage} \begin{verbatim}
goblin short sticka  4ap dam 4/5, pen 0/1, abs 2, parry-1/+1, finesse-3/-5
1h/2h                str 4/4 (max dam+1 pen+1)
spear skill          dex 7: optional 3ap dam-1
                     dex 10: 3ap no dam mod
                     reach 1 mod-6
\end{verbatim} \blocklistgap \begin{verbatim}
goblin longa sticka  4ap dam 4/5, pen 0/1, abs 2, parry-3, finesse-2
2h                   str 4/4 (max dam+1 pen+1)
spear skill          dex 7: optional 3ap dam-1
                     dex 10: 3ap no dam mod
                     reach 1 mod-3
\end{verbatim} \blocklistgap \begin{verbatim}
goblin hakka   4ap dam 6, abs 2, parry-4, toparry-1, toavoid+1, finesse-2
1h             str 4, max dam+1
axe skill      dex 7: optional 3ap dam-1
               dex 10: 3ap no dam mod
               swing: slow 1 mod-1 dam+2 toavoid+2
\end{verbatim} \blocklistgap \begin{verbatim}
goblin klyvva  4ap dam 8, abs 3, parry-4, toparry-1, toavoid+1, finesse-2
2h             str 4, max dam+2
axe skill      dex 7: optional 3ap dam-2
               dex 10: 3ap no dam mod
               swing: slow 1 mod-1 dam+2 toavoid+2
\end{verbatim} \blocklistgap \begin{verbatim}
goblin swisha  4ap dam 5, abs 3, finesse-4
1h             str 4, max dam+1
knife skill
\end{verbatim} \end{samepage} \normalsize \goodbreak

\


\goodbreak
\noindent Orcs produce simple and heavy weapons. They have higher abs, lower finesse, higher str requirements. The market price is -50\% of human made weapons and their encumbrance is +33\%. When taking damage to abs they loose double abs.

Wielding orc weapons also require the speciality orcish, cost 5xp, or suffer mod-1.
All npc orcs have it innately, and Hero orcs starting with orc weapons get it for free.

\

\small \begin{samepage} \begin{verbatim}
orc war axe         dam 10, abs 10, parry-3, toparry-1, toavoid+1, finesse-2
1h                  str 9 (max +6 str bonus)
                    swing: slow 1 mod-1 dam+3 toavoid+2
\end{verbatim} \blocklistgap \begin{verbatim}
orc war spear       dam 7/8, pen 1/2, abs 8, parry-1/+1, reach 1 mod-3
1h/2h               str 9/9 (max +3 str damage bonus, max+3 penetrating bonus)
                    finesse-2/-4
\end{verbatim} \blocklistgap \begin{verbatim}
orc war shield      abs 13, parry+3,
                    str 9, or slow 1 str 6
                    Ranged attacks mod-2 when in the way.
                    Hiding behind it (3ap) ranged mod-5
                    tackle mod+3
\end{verbatim} \blocklistgap \begin{verbatim}
orc cleaver         dam 11, abs 12, parry-4/-3, toparry-2, toavoid+2,
1h/2h               reach 1 mod-4, finesse-2
(speciality 10xp)   str 11/9, or slow 1 str 8/6
                    swing: slow 1 mod-1 dam+3 toparry-1 toavoid+2
                    can be used with either axe, sword, or spear skill
                    bastardised orc version of a short glave-ish thingy
\end{verbatim} \end{samepage} \normalsize \goodbreak

\


\goodbreak
\noindent Dwarves make their own personal battle axes and crossbows, customised for the owner and each one has their own personality and special stats, making them difficult to use. The dwarves takes great care of their weapons and they often stay with their owner through their entire life, possibly hundreds of years. The Dwarven clan smiths very rarely make a weapon for someone outside their clan.

Encumbrance is similar as the human standard weapons, but the durability is much better. Price is also significantly higher, at least 5x human made. Start around 3+1d3g +5x base price. Remember that it's all custom made items. Dwarven weapons cost at least 5g.

Dwarven weapons are sturdy and generally have 33\% higher abs than human made weapons, and take 1 less damage to abs after rolling for weapon breakage.

\

\small \begin{samepage} \begin{verbatim}
dwarven 1h axe      dam 8, abs 12, parry-3, toparry-1, toavoid+1, finesse-3
                    str 5 (max +4 str bonus)
                    swing: slow 1 mod-1 dam+2 toavoid+2
\end{verbatim} \blocklistgap \begin{verbatim}
dwarven battle axe  dam=str+3, abs=dam*2, parry-2, toparry-2, finesse-5
1h                  custom made: str 6-10
(speciality 5xp)    swing: slow 1 mod-1 dam+2 toavoid+1   (5xp)
                    poke: mod-2 dam-3 pen+2               (5xp)
                    Each axe requires its own speciality 5xp.
                    Each axe has 1 points to improve either:
                    parry / toparry / toavoid / finesse
\end{verbatim} \blocklistgap \begin{verbatim}
dwarven breaker     abs 10+str*2, parry+4,
shield              custom made: str 6-10
(speciality 5xp)    Ranged attacks mod-4 when in the way.
                    Hiding behind it (3ap) ranged mod-8
                    tackle mod+3
                    Each shield requires its own speciality 5xp.
\end{verbatim} \blocklistgap \begin{verbatim}
dwarven clan axe    dam=str+6, abs=dam*2, parry-2, toparry-2, finesse-5
2h                  custom made: str 6-10
(speciality 10xp)   swing: slow 1 mod-1 dam+3 toavoid+1   (5xp)
                    fast parry: 2ap mod-3 (dex=str+2)     (10xp)
                    Each axe requires its own speciality 10xp.
                    Each axe has 2 points to improve either:
                    parry / toparry / toavoid / finesse
                    with max one point on any one stat.
\end{verbatim} \blocklistgap \begin{verbatim}
dwarven crossbow    dam=str, pen 3
(speciality 10xp)   range 15+dex, no str bonus
                    custom made: str 6-10, dex 3-10
                    reload 3r (2r str+3, dex+3 : str & dex 3 higher than custom)
                    Each crossbow requires its own speciality 10xp.
                    Each crossbow has 1 point to improve either:
                    dam or pen.
                    Requires special dwarven made bolts, stats as usual bolts.
\end{verbatim} \end{samepage} \normalsize

\


\goodbreak
\noindent Elves produce slender weapons of very high quality. They will last longer and when rolling for weapon breakage they have half the chance of breaking (round down) and their permanent abs damage should be rounded down instead of rounded up. Elven made weapons are very rare and difficult to find outside of elven cities. Their encumbrance is -30\% of the human version, and the price is at least 10x higher than a human made. Start at 5+1d5g + 10x base price. Elven weapons usually cost at least 10g. Most are custom made for their owner. Consider the list below as a baseline.

Training the "elvish" speciality, at 10xp cost, will grant a mod+1 when wielding any high quality elven weapon. The rare poor quality elven weapons do not get mod+1.
The slenderblade, illspets, and truebow are still speciality weapons though, and will require their own specialisation skill, cost 10xp, or suffer mod-1. The "elvish" speciality and the weapon specialisation skills stack for a total of mod+1.

\

\small \begin{samepage} \begin{verbatim}
elven sword         dam 6, abs 12,
                    str 4 (max +2 str bonus), finesse-7
                    poke: mod-1 dam-1 pen+1
                    swing: slow 1 mod-1 dam+2 todefend+2
\end{verbatim} \blocklistgap \begin{verbatim}
elven slenderblade  dam 6, abs 14,
1h                  str 6 (max +2 str bonus), finesse-9
(speciality 10xp)   dex 8 fast 2ap
                    poke: mod-1 dam-1 pen+1
                    swing: slow 1 mod-1 dam+2 todefend+2
                    deflect is mod-1 instead of mod-3
\end{verbatim} \blocklistgap \begin{verbatim}
elven illspets      dam 4, pen 4, abs 12, parry+1
1h                  str 6 (no str dam bonus, max +2 pen str bonus)
(speciality 10xp)   dex 8 fast 2ap
                    poke: mod-1 dam-1 pen+2
                    slash: mod-1 dam+1 pen-2
                    deflect is mod-1 instead of mod-3
\end{verbatim} \blocklistgap \begin{verbatim}
elven truebow       dam 5, penetrating 2,
(speciality 10xp)   range 24, short 12 mod+1, long 36 mod-3,
                    vlong 48 mod-6 dam-1, extreme 70 mod-9 dam-2,
                    str 6 (no str bonus)
                    quick shot mods are +1 if dex 8
                    with elven arrows the bow is normally mod+1
\end{verbatim} \end{samepage} \normalsize

\

The elven illspets is used as a fast light defensive weapon which can also be used to attack heavily armoured targets.










%-------------------------------------------------------------------------------
% E Q U I P M E N T
%------------------

\phantomsection\addcontentsline{toc}{section}{basic equipment}
\section*{Basic equipment}

There is a plethora of useful stuff an adventuring Hero can make good use of in his travels. Perhaps some coffee before the cold dungeon? Or a health wafer or two with the evening meal after leaving the dungeon with loot and wounds?

Most basic equipment is mundane and not explained below. Just look at the \hyperref[sec:basicequipmentpricelist]{price list}, page~\pageref{sec:basicequipmentpricelist}.

\

\small \begin{samepage} \begin{verbatim}
sheath/scabbard/straps/etc  A good place to keep your weapon is critical if you
               want to be sure you can draw it reliably when needed.
               A good scabbard will always ensure the weapon can be drawn and
               sheathed without having to roll and risk a failure.
\end{verbatim} \blocklistgap \begin{verbatim}
rope           mod+6 to climb
\end{verbatim} \blocklistgap \begin{verbatim}
grappling hook roll dex + throw to hook it, 1r action.
               A careful 3r action gives a mod+3 to hook it
\end{verbatim} \blocklistgap \begin{verbatim}
antidote       counters the effects of a poison, weaker than the antidote's str.
               Very general antidotes you have here...
\end{verbatim} \blocklistgap \begin{verbatim}
pain killer    small white pills, red pills, liquids, etc, with different
               strengths and side effects.
               eliminates some pain points equal to strength
               usually takes a few rounds to have effect.
\end{verbatim} \blocklistgap \begin{verbatim}
alcohol        5r to take effect.
               reduces pain by -1 per dose,
               adds an inebriation mod-1 per dose above con/3 (round down),
               OR counters one dose of coffee.
               Effect remains until next day, or until countered.
\end{verbatim} \blocklistgap \begin{verbatim}
coffee         5r to take effect, 2r if hot (use a thermos perhaps?)
               increases current and max stamina by one for each dose (max+3),
               OR if inebriated, then cancels all effects of one dose of
               alcohol per dose of coffee.
               Above con/3 doses the hero is jittery and must pass a psy roll
               each round he tries to rest. Effect remains until next day.
\end{verbatim} \blocklistgap \begin{verbatim}
health elixir  heals 3hp, 1hp/3r, and draws 3mana and 3stamina immediately
               when quaffed. It has no effect if the mana or stamina draw rolls
               fail.
\end{verbatim} \blocklistgap \begin{verbatim}
health wafer   heals 2hp, 1hp/10r, and modifies max stam -2
               until the Hero has slept.
\end{verbatim} \blocklistgap \begin{verbatim}
healing salve  heals 1hp in 100r, max 1 application per wound per day.
               gives +1 pain per use until the target has slept.
               applying more than 1+con/3 per day gives risk of poisoning:
               roll con-(nr applications) or suffer mod-3 whole day
               until successful, cumulative mod+1/d
\end{verbatim} \blocklistgap \begin{verbatim}
health tea     takes a fire and 100r to brew, must be served hot,
               heals 1d3 hp in next 100r at cost of 1 mana,
               no effect if mana draw fails.
               The mana can be paid by the target drinking,
               or by the one preparing the tea.
               Drinking more than con/3 per day gives risk of poisoning:
               roll con-(nr doses) or suffer mod-3 whole day
               until successful, cumulative mod+1/d
\end{verbatim} \blocklistgap \begin{verbatim}
lock picks     mod+X to "pick lock" skill.
               X depends on the quality of the picks.
\end{verbatim} \blocklistgap \begin{verbatim}
tool box       mod+X to "traps" and "McGyverism" skills.
               X depends on the quantity and quality of the tools.
\end{verbatim} \end{samepage} \normalsize

\








%-------------------------------------------------------------------------------
% S P E C I A L   E Q U I P M E N T
%----------------------------------


\phantomsection\addcontentsline{toc}{section}{special equipment}
\section*{Special equipment}

Special items are not so easy to come by and their prices vary a bit.

\

\small \begin{samepage} \begin{verbatim}
cave cart      A small (1sq) cart that can be dragged (str 5 walk) into the
               narrow spaces of the caves where monsters and treasure hide.
               It can load 100 enc. Rummaging on the cart is same as in a normal
               container, except effective item number is items/10 round up.
\end{verbatim} \blocklistgap \begin{verbatim}
Rucksack with  Regular backpack but with a few compartments, which makes it
compartments   faster when rummaging through it for the required item.
               Divide the amount of items by 2-4 when calculating rummage time.
\end{verbatim} \blocklistgap \begin{verbatim}
tunnel plug    A small (1sq) cart with a very large and sturdy shield that can
               be flipped into position to block up to three square wide so
               pursuing monsters cannot pass or attack through it.
               The cart can be dragged (str 3 walk) into the narrow spaces of
               caves, then anchored to the ground when flipping up.
               Takes 3r to flip up and anchor.
               plug shield: abs 5, 50hp.
\end{verbatim} \blocklistgap \begin{verbatim}
tunnel plug    A medium (2x2) cart with a very large and sturdy shield that can
large          be flipped into position to block up to four squares wide so
               pursuing monsters cannot pass or attack through it.
               The cart can be dragged (str 5 walk) into the narrow spaces of
               caves, then anchored to the ground when flipping up.
               Takes 5r to flip up and anchor.
               plug shield abs 6, 100hp.
\end{verbatim} \blocklistgap \begin{verbatim}
portadoor      A portable (enc 5.0) door that can be set up in 3r (full round
               actions). It can block one square so that pursuing monsters
               cannot follow or attack through it.
               abs 4, hp 40.
\end{verbatim} \blocklistgap \begin{verbatim}
portagate      A portable (3x enc 5.0) gate that can be set up in 5r (full
               round actions). It can block one or two squares so that pursuing
               monsters cannot follow or attack through it.
               abs 5, hp 50
\end{verbatim} \blocklistgap \begin{verbatim}
fightlight     A large brazier with a polished metal mirror behind it, creating
               a very bright 90deg cone of light, 40sq radius.
               Just put it on the ground and light it. The expensive models even
               have a smoulder box or a spark lever for fast lighting.
               enc 10.0
\end{verbatim} \blocklistgap \begin{verbatim}
fightlight     As the larger fightlight, but weaker. enc 5.0
small          90deg cone, 20sq radius
\end{verbatim} \blocklistgap \begin{verbatim}
magnifying     Gives mod+3 to find, but takes one round extra for each find roll
lens
\end{verbatim} \end{samepage} \normalsize

\









%-------------------------------------------------------------------------------
% E X O T I C   E Q U I P M E N T
%--------------------------------

%Exotic items are difficult to find and their prices vary greatly.


\phantomsection\addcontentsline{toc}{section}{exotic equipment}
\section*{Exotic equipment}

These items are usually custom made, unique or in small series, at great expense and by the best master craftsmen. A few examples of the rare and expensive stuff people sometimes buy:

\

\small \begin{samepage} \begin{verbatim}
todo: list stuff here
\end{verbatim} \blocklistgap \begin{verbatim}
todo: list stuff here
\end{verbatim} \end{samepage} \normalsize

\









%-------------------------------------------------------------------------------
% T R A P S
%----------


\phantomsection\addcontentsline{toc}{section}{traps}
\section*{Traps}

Traps are usually the kind of things that you fall into, put spikes into your belly, poke your eyes out, or that prick your finger injecting horribly painful poison. But they can also be your friends. If you have the time and money a few well placed traps can slow down or decimate your enemies as they advance to kill you dead.

All traps have a few different properties such as price and encumbrance when you buy and transport them. Then size, trigger type, damage, time to set up and arm, difficulty to hide, etc. Some traps require that you dig a hole, or hack out some empty space in a wall.

\

\small \begin{samepage} \begin{verbatim}
small snare: size 1sq 33% chance of triggering.
    Caught target can break free in one action on str vs 5 roll
    or disentangle in two rounds after successful int vs 2 roll
    or loop cut in one round if carrying small blade or similar.
    Takes 3r to set and must be anchored to something within 5sq.
    They are easy to hide at mod+3.
\end{verbatim} \blocklistgap \begin{verbatim}
large snare: size 1sq 66% chance of triggering.
    Caught target can break free in one action on str vs 8 roll
    or disentangle in two rounds after successful int vs 2 roll
    or loop cut in one round if carrying small blade or similar.
    Takes 5r to set and must be anchored to something within 8sq.
    They are easy to hide at mod+3.
\end{verbatim} \blocklistgap \begin{verbatim}
trigger snare: size 1sq 90% chance of triggering.
    It's a large snare with an extra trigger tension spring.
The common way to anchor snares is around tree trunks or other protruding solid
objects. Other ways is to use dirt wedges to anchor it in the ground. A dirt
wedge can be hammered down using heavy blunt object in 2r and can be retrieved
in 1d5 rounds using similar objects.
Snares can also be anchored in some stone walls using rock wedges. A rock wedge
can be inserted in a crevice using a hammer, butt of an axe, stone, or similar,
in 2r. It can be retrieved using similar tools in 1d5r.
Dirt and rock wedges hold a certain strength, usually 5-20, depending on object
and environment.
\end{verbatim} \blocklistgap \begin{verbatim}
spike arm:
        Pre-made: cost 5s in materials, enc 5+1/dam.
        Half foraged: cost 3s in materials, enc 3 + 1/2dam + foraging.
        Completely foraged: cost nothing but requires a knife. not portable.
        Spike arm traps are the things that usually put a slab of pointy objects
        into the stomach of people. Excellent to rig just behind a corner or a
        tree, or perhaps directly in the ground covered by old leaves or snow.
        A skilled woodsman can forage for all the materials, but that takes 100r
        per dam extra. A smart man brings some stuff with him, and forages only
        for the scaffold and such, which takes 5r/dam. When in dungeons etc it
        might be a good idea to bring all of the materials in a bundle.
        A pre-made bundle is set in 10+dam rounds.
        A half foraged trap is rigged in forage time + 15+2*dam rounds.
        A completely foraged trap is constructed in forage time + 20+4*dam time.
        Hiding a spike arm trap is mod-0.
\end{verbatim} \blocklistgap \begin{verbatim}
pit trap: cost nothing, requires perhaps a canvas, and some shovels to dig with.
        Pit traps are a cheap and common trap for when you have enough time.
        1) Dig a pit. In dirt outdoors this takes 100r/sq if the digger has
        a spade or shovel. Otherwise it takes four times as long. As many
        people can help as there are connecting side squares of the trap.
        2) Put down spikes or other nastiness, takes a few rounds per sq.
        3) Cover with suitably weak material, another few rounds per sq.
        4) Camouflage the trap, another couple of rounds per sq.
        A regular pit does 3 dam, a double deep pit does 6 dam, a triple deep
        pit does 10 dam. With spikes the damage is double, and the spikes get
        one point of penetrating per depth of the pit, including the first.
        Deep pits take longer to dig of course.
        Hiding a pit trap is mod+3.
\end{verbatim} \end{samepage} \normalsize










%-------------------------------------------------------------------------------
% T R A I N E D   A N I M A L S
%------------------------------

\phantomsection\addcontentsline{toc}{section}{trained animals}
\section*{Trained animals}

Companion animals can be trained to do simple tasks. Each animal has a set of specific commands it knows and can perform. Animals also have a command modifier, depending on how well they are trained, which affect how easy it is to command them. The skill "animal command" is used to handle animals.
The GM has ultimate control over the details of the animal's actions, but it will follow the given command as it interprets it, and according to situation. The practical movement on the map, rolls, etc is generally handled by the commanding player.

Some smarter trained animals can have skills and xp.
It can for example be useful for an archer to train their attack hawk or war dog as a "target pointer".



\openitemslist

\eqitem{Horse:}
An average horse can carry a rider plus a 20enc pack load in saddle bags without problems, at a daily 10sq distance. Without rider it can take 60enc in sacks. Horses vary greatly in training, movement, travel, carrying capacity, etc.
\small \begin{samepage} \begin{verbatim}
===================================
average horse               (token)  set token size large (2x2)
-----------------------------------  position rider token at top rear square
str 30    hp 40 abs 0
dex  5    m3 w6 r12 d24
per  8    initiative 8
rear kick   5 dam8 pen2 knockback 2
front kick  6 dam4 pen1 knockback 1
bite        4 dam2
-----------------------------------
\end{verbatim} \end{samepage} \normalsize


\eqitem{Donkey:}
Normal donkeys are not trained for riding, but to carry load or drag a small cart. It carries 50enc in pack sacks for 10sq per day.
\small \begin{samepage} \begin{verbatim}
===================================
donkey                      (token)  set token size normal
-----------------------------------
str 20    hp 25 abs 0
dex  5    m2 w4 r8 d16
per  8    initiative 8
rear kick   6 dam6 pen1 knockback 1
front kick  6 dam3
bite        7 dam2
-----------------------------------
\end{verbatim} \end{samepage} \normalsize


\eqitem{War Dog:}
Big and dangerous breed of dogs.
\small \begin{samepage} \begin{verbatim}
===================================
war dog                     (token)  set token size small
-----------------------------------
str  4    hp 5 abs 0
dex  8    m2 w4 r8 d16
per 10    initiative 12
charge 6
balance 3
avoid 4 yield+4 always yield when possible
bite 5 dam 5
claw 5 dam 2 fast 1
-----------------------------------
\end{verbatim} \end{samepage} \normalsize
Command: attack target pointed at: Will charge the target and bite and claw. \\
Command: return to owner \\
Command: guard spot or small area: Will attack intruders that get within 5 range from the area. \\
Command: guard person or group: Will attack intruders that get within 5 range from the person or group. This is the default behaviour in combat situations. \\
Command: follow: Will follow owner and keep calm unless obvious enemy comes within 1d4sq. This is the default non-combat behaviour. \\
Command: fetch arrow: Will run to a corpse and fetch an arrow. Better trained dogs might fetch more than one. This is not a standard command and costs extra.


\eqitem{Attack Hawk:}
Well trained hawks or other large fast bird of prey. \\
hp 1, to hit -6 (incl fighting speed), movement fly 20, \\
claw and beak: 5 damage 1 \\
avoid  6 (incl always yield) \\
Command: attack target pointed at: \\
Will fly to the target and distract it by clawing at it's face. This gives the target a mod-3 to all actions, and must pass an int roll to not try to target the hawk. \\
Command: return to owner. \\
Command: follow, will follow the owner or sit on his shoulder.


\eqitem{Darkwing:}
Large bats, not very clever. \\
hp 1, to hit -8 (incl speed), movement fly 15 \\
avoid 5 (incl always yield) \\
Command: scout \\
Will fly away in the indicated direction. \\
Return immediately and chatter if found anything that might be dangerous. \\
Return after a while calmly if it didn't find anything interesting. \\
Command: follow \\
Will follow the owner, or cling to his back, and stay out of trouble.


\eqitem{Fetching Ferret:}
hp 1, to hit -6 (incl speed), movement r4 d12 \\
encumbrance limit 1.0, cannot carry more than that. \\
Command: fetch \\
Pick up indicated object and bring it back to owner. \\
Might not always pick up and bring what you want. \\
Command: port \\
Pickup an object from the first indicated person and give to the second. \\
Might not be what you want unless actually given the item. \\
Command: return to owner. \\
Comes back and clings to the owners shoulders or such.


\eqitem{Tripping Traccoon:}
hp 2, to hit-3 (incl speed), movement r4 d12 \\
encumbrance limit 2.0, cannot carry more than that. \\
Command: fetch (see ferret) but only stuff that glitters or smells interesting \\
Command: port (see ferret) \\
Command: return \\
returns to owner and follows or clings. \\
gives track+1


\eqitem{Wolverine:}
Very persistent, inexhaustible endurance, never gives up. With con99 it's just silly, but make it die or limp away when taken suitable amounts of damage for the situation. The maptool implementation means stamina and con must both be set. Also fun that it will always seem "rested" under most normal situations.
\goodbreak \small \begin{samepage} \begin{verbatim}
===================================
wolverine                   (token)  set token size small
-----------------------------------
str  4    hp 5 abs 0
dex  8    m2 w4 r6 d12
con 99    stamina 99 (makes it near impossible to actually kill)
per 10    initiative 10
pain threshold 5
black knight 2
balance 6
charge 3
avoid  4 yield+3 always yields when possible \\
avoid 4 yield+4 always yield when possible
claw: 5 dam 3 fast 1
bite: 6 dam 4 pen 1
hold: 9 dam 2 pen 2 only after successful bite (prev round ok, no movement)
                    target has mod-3 to all actions, persists over rounds
-----------------------------------
\end{verbatim} \end{samepage} \normalsize
Command: attack: Attacks indicated target and follows until called back, or target is dead. \\
Command: return: Returns to owner and follows, does not start fights unless provoked or commanded.


\eqitem{Explorat:}
Small, although large for a rat, curious, can gnaw it's way through most anything given time. \\
Can be trained to search for one specific thing, like monsters, gold, food, water, etc.\\
hp 1, to-hit-6, movement 4 \\
Release and it will return to find you sooner or later, hopefully indicating that it has found what it is trained for. \\
A few reports have come in that some breeds of explorats have a tendency to spontaneously and violently combust, or explode. This has never been proven and should under no circumstances deter you from purchasing one of this small but very helpful pets.


\eqitem{Bright Beetle:}
Is actually a rather large beetle. It is barely trainable insect, but it's calm and unafraid, and will go where it is commanded. \\
Command: Go \\
The beetle will travel to the indicated location, then stay there. \\
Command: Return \\
The beetle will return to the owner, then follow. \\
Command: Target \\
The beetle will travel to a target then stay a few squares away, following it.\\
hp 1, to hit-3 (melee) to hit -6 (ranged) \\
movement walk 3, run 6, fly 9 \\
It is too small to take up a square and cannot block movement. When killed it explodes doing dam 3 to the square it is on and dam 1 to radius 1.


\eqitem{Empowering Brightwing:}
A medium size very rare and exotic bird. It glows with a yellow light (light source range 5), and empowers those in contact with it with extra energy. A brightwing restores 1 stamina each full round it is sitting on a character's shoulder. It will not land, or stay on a shoulder when the character is moving faster than walk, or when he is in melee combat. When not sitting on its owner's shoulder it tends to stay in the vicinity. \\
Immune to fire. \\
hp 3, to hit-6 (incl movement), movement fly 10 \\
avoid 5 (incl always yield) \\
Command: go to \\
Will fly to indicated character and land on his shoulder. \\
Command: return \\
Will fly back to the owner and land on his shoulder.


\eqitem{Medicinal Serpent:}
A very rare and exotic snake, about 1-2m long. It is green and oozes slightly of a glowing greenish aura (light source range 2). It rests on a staff or other long stick. For each consecutive two full rounds the snake is coiled around a target the target heals 1hp. The target cannot take any actions, and not move faster than maneuver. The owner of the snake may move at walk speed and take minor actions that do not require much movement. \\
hp 5, to hit-3 (incl movement), movement slither 4. \\
Command: engage \\
The snake will slither onto the indicated target and coil around its abdomen.
Command: disengage \\
The snake will uncoil from the target and return to its resting place


\eqitem{Glow Fly:}
Various small critters that glow in the dark. They don't give off much light, but it can be enough to scout some areas, especially for Heroes with dusk vision or night vision. The glow bugs are very small, hp 1, mod-6 to hit them with melee weapons, and mod-9 with ranged weapons. Various bugs have different speeds, from crawl 1 to fly 10. They are impossible to train, and have to be commanded by use of special food pastes. The owner throws a dollop of paste at a location or target and the bug goes there to eat. It will stay for a few rounds depending on the size of the dollop. Roll for animal command, and apply the fail diff as random +/- rounds to the intended time. After it has finished eating it will return to the owner if he is within 10-30 sq, depending on how keen eyes, ears, or nose the bug has. \\
It is recommended to have the gob dispenser as quickdraw, then the "command" of throwing a dollop takes two actions, one for select size and one for throw. Otherwise it takes two rounds. Hitting the intended area with a dollop is a throw roll mod+3. Hitting an intended target is a regular throw roll. Dollop base range is dex+str, but minimum 3sq.


\eqitem{Jumping Spider:}
Aggressive spiders that can barely be trained. Unless directly controlled by animal command the spiders will always attack the nearest living non-spider it can find. This includes allies. They jump onto the target then stays attached,
making one "bite" attack each round until called off or the target dies.
\small \begin{samepage} \begin{verbatim}
jumping spider: hp 1, move 6, jump 2, all terrain (no water)
                Tiny target: melee mod-3 ranged mod-6
jump-attach attack 7 (once per round)
        The jump attack (2sq) does not count against movement.
        The spider stays attached until cleaned off or target dies.
        Each attached spider gives the target mod-1 to all actions.
bite 8 (mod-1 per abs) (once per round, after attached)
        dam 1 penetrating
command: attack (point and whistle, 3ap) max range 6
        The spider will attack the target until it's dead.
        Then it will move to attack the closest living non-spider
        it can find, including allies.
command: return (whistle, 3ap) max range 15
        Spiders will disengage and return to handler.
It takes an action (3ap) and a dex roll to clean off a spider.
The spider is placed on an adjacent square chosen by the target.
Others have dex+3 when helping the target.
Attacking a spider which is attached to a target require success+3
otherwise rest of the damage hits the target. Fail-3 or worse will
hit the target.
\end{verbatim} \end{samepage} \normalsize
Spiders can be housed in a "spider box", or "hive box". A box normally holds max ca 3 spiders per enc. E.g: spider box enc 4 has "ammo" 10+1d4 spiders. A spider box can normally spawn a spider each round.


\closeitemslist









%-------------------------------------------------------------------------------
% C O M P A N I O N S
%--------------------

\phantomsection\addcontentsline{toc}{section}{companions}
\section*{Companions}

The GM has ultimate control over the details of the companion's actions, but it will follow the given command as interpreted according to situation. The practical movement on the map, rolls, etc is generally handled by the commanding player. The character that handles companions need the companion command skill.

In general, most companions will take or require their fair part of XP and loot.


\openitemslist

\eqitem{Henchman:}
A henchman can be anything from a goblin to an elf, he can be rented for the day or a trusted old friend. Henchmen have a mind of their own and full stats and skills according to a regular character sheet.
Hiring henchmen is expensive. They generally require both a payment up front and a cut of the treasure. Of course, if he cannot count... But most henchmen can both count and haggle. Annoying, I know.

The henchman will also take a full part of the XP from the adventure, just like a Hero. Hiring a bunch of henchmen will make the adventures safer, but reduce the experience gained.


\eqitem{Demons, spirits, etc:}
Some heroes can summon strange beings to do their bidding. Some of these odd critters can stay around for a long time, and they might even be difficult to get rid of. Some will require some sort of payment or compensation, be it in gold, blood, xp, special equipment items, or left ears of the righteous.


\eqitem{Trained animals, etc:}
Covered above. Trained animals, or monsters, are excellent companions but often with lower versatility.

\closeitemslist


For calculating division of plunder when one or both of the parties fail their counting rolls see the "counting" skill description. Some loot requires proper valuation. This is why henchmen often have some levels in the appraise skill.

Each henchman or companion critter have their own personality and goals. It may be nothing special, simple greed, or it may be something which has special meaning in the current adventure or campaign plot arc.

When playing henchmen, be stringent with actually letting them behave in a reasonable way. They don't have magical foresight, perfect sense of tactics, unlimited bravery, or other feats which would be required to act as the players often wish them to act. Henchmen require clear orders, and they will generally adhere to those orders as long as it makes sense and does not send them into unreasonable danger.

Situations often apply where the controlling player might want the henchman to take other actions than he has been ordered to because it is "clearly better". First, would it be better for the henchman? Is it evident for the henchman? Is he smart enough to figure it out (roll int)? Secondly, does the henchman often have enough free reign to ignore orders? Is he often strongly ordered to do dangerous stuff? Does he have the presence of mind to think clearly (roll psy)? And so on.

Most henchmen will try to save their own skin when they start to get hurt. They might disobey orders, hesitate when ordered to attack, be more defensive, etc.











%-------------------------------------------------------------------------------
% L E G E D A R Y
%----------------

\phantomsection\addcontentsline{toc}{section}{legendary items}
\section*{Legendary items}


These items are almost always unique and custom made, at great expense, by renowned craftsmen. A few examples of the outrageously expensive items few people have managed to acquire:

\

\small \begin{samepage} \begin{verbatim}
Master BronzeBolt's Twin Whistler
heavy crossbow      dam 6, penetrating 2, mod+1
double bow          range 15
repeater            str 7 (no str bonus)
(speciality 10xp)   reload 2r (str 10 dex 13 : 1r)
                    magazine 12 (8r manual reload, 4r if dex 10)
                    requires special bolts: mod+1, pen+1, range+10
A truly remarkable piece of legendary artisanal mechanickery.
It can fire twin whistling arrows in a single action, or separately, and sings
beautifully when the reload mechanism is pulled.
It is rumoured that BronzeBolt paid over 300 gold for the machine and has to
pay 1 gold per bolt. Only one craftsman can make the ammunition and it takes
him several days to make a bundle of bolts.
\end{verbatim} \end{samepage} \normalsize

\

\small \begin{samepage} \begin{verbatim}
bla bla bla
\end{verbatim} \end{samepage} \normalsize




