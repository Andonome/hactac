%--------|---------|---------|---------|---------|---------|---------|---------|
%       10        20        30        40        50        60        70        80
%                                                                               12345

%===============================================================================
%
% rules
%
%===============================================================================


% force start on right side page
\cleardoublepage
\flushbottom

% set fancyhdr heading
\chaptermark{rules}

% manually fix the table of contents and no numbering or "chapter" heading
\phantomsection\addcontentsline{toc}{chapter}{Rules}
\chapter*{Rules}
%---------------
Never let the rules get in the way of fun. If something is more entertaining but contrary to the rules, go for fun. That said, a functional and consistent implementation of rules is important as it allows players to develop a better understanding of the world, a gut feeling for the risks and possibilities, take greater calculated chances, find that cool edge, discover a fun loophole, powerful combo, etc.

The rule set is here to \emph{help} create a fun game experience, not to kill it by a thousand cuts and struts. Strike a balance that suits your group.

\emph{Make it your own!} Your game, not mine. All of the stuff here are suggestions. Pick the stuff you like, skip the rest, add and rewrite to suit yourself and your players. If it works well, push it upstream.


\phantomsection\addcontentsline{toc}{section}{introduction}
\section*{fundamental introduction: what, why, how}
%--------------------------------------------------
The idea is to create more tactical elements in tabletop/digital rpg-hacknslash. Inspired by some concepts from classic tabletop games like space hulk, descent, etc, but striving for more interesting battles, and tying it together with role playing game style adventures and campaigns.

This game is not for everyone, but it is designed to be easy to scale and tweak over a wide range of preferences: In it's simplest form it works well as a small and fast framework for quick gaming with young kids, on a table or across the living room. At full complexity it is a very challenging online tactical tabletop game wrapped in a lighthearted role playing game environment.
E.g: At the very simple end see
\hyperref[sec:youngkids]{\texttt{Bitter Candy}}, page~\pageref{sec:youngkids} and
\hyperref[sec:basicenough]{\texttt{Ominous Crown}}, page~\pageref{sec:basicenough},
and for really complex examples look at the adventure
\texttt{Turn off the Lights} and
the Lundqvist force in the \texttt{Goblin Destiny}
campaign.

It's primarily designed for fast, online, tactically deep, tabletop and rpg campaign gaming. And specifically to make it easy to schedule game sessions for busy people and still have interesting gameplay that requires some thinking.
%\TODO: My players are mostly engineers and scientists, so the difficulty level can get quite high.

Cooperation and coordination, movement, positioning, and clever use of character skills is much more important than commonly found in monster hack games of today and yesteryear.

The \textbf{Gold and Glory} rule set, game setting, and adventure campaigns are not created with any form of \emph{realism} in mind. The idea is to make it \emph{fun} and \emph{interesting} to play, nothing else.

Please, when reading and playing, keep you eyes open and head in gear. If the rules don't make sense, seem unbalanced, or should be more fun in a different way, \emph{please tell me about it}. Ping me an email or make a change and send a pull request. My handle is \emph{netjiro} on gmail and github.


%\

\cleartoleftpage
\textbf{Game targets:}
\begin{itemize}
    \item Schedulable: Interesting even in short online game sessions.
    \item Allow GM and players to select suitable level of complexity.
    \item Flexibility and options, with consequences and tradeoffs.
    \item Dynamic, flowing fights, but still fast to play.
    \item Tactical depth, movement and actions matter, no dice rally.
    \item Terrain and layout matters, positioning, movement, timing matters.
    \item Many different viable combat styles and character types.
    \item Require player coordination for combat success.
    \item Initiative favours the fast through higher flexibility.
    \item Numerical advantage is synergistic, not just linear dps.
    \item Action coordination is highly advantageous.
    \item Specialities of characters to have significant influence.
    \item Different skills and equipment should play and \emph{feel} distinct.
    \item Success and failure should not be binary all or nothing, but gradual.
    \item Freedom in character building, no classes or similar limitations.
\end{itemize}

These are lofty goals. Things I want out of gaming. Most tabletop games I have tried fall short.

%\

%\cleartorightpage
\clearpage
\textbf{How to get there:}
\begin{itemize}
    \item Simple core rules that create inherent depth.
    \item Optional addons for variety.
    \item Effects and options stack together and can be combined.
    \item The degree of success or failure have real effects, not just success/fail.
    \item Trade realism for interesting options and effects.
    \item Doing more things in a round makes it harder to succeed with actions.
    \item Faster movement makes it harder to succeed with actions.
    \item Characters with high initiative can decide when to act in a round.
    \item Relative positioning, facing, direction gives mods to actions.
    \item Attack, defence, movement, damage, etc are multidimensional, not single axis.
    \item Actually care about how probability works.
    \item Keep the rounds short and quick. More rounds, more options.
\end{itemize}

\

Just charging into the opponents and hacking away will lead to disaster and gruesome death. Instead there are tactical considerations that enhance survival and make the battles more interesting. Ganging up on an opponent to overcome defences, coordinate actions and positioning to create openings, or distracting him to allow the others to get in better hits, are all very powerful tactics. Moving and positioning to get the upper hand in a battle will make a world of difference, especially when fighting many on many, or in complex terrain and map layout. Always analyse the terrain when planning your next segment of the fight. Be ready to reassess the combat plan as the circumstances change. Fights are fluid and variable, not static line dancing with sharp sticks. The rules set the scene for flexible shuffling of offensive or defensive focus and tactical initiative. Coordination of timing and ordering of actions between characters has a large impact on any battle. Smart positioning, timing, and coordination of actions can be devastating to the opposition if done right.

\

Normal quick skirmishes with around 10 combatants take perhaps a millifortnight or two and play out similar to this: \\
Heroes 2-5 attack monsters 2-10: \\
round 1-5: Initial positional movement for tactical advantage. \\
round 3-8: Contact: a few attacks, parries, duels. \\
round 6-15: Repositioning, movement, attacks, parries and counter attacks. \\
round 10-20: heavy violence and tactical movement to exploit weaknesses. \\
round 15-25: flights, chases, mopping up the stragglers, or flee to fight another day.

Larger or more complex fights of course take longer to complete. In hot areas several small skirmishes will meld into long rolling fights with positioning for control and advantage, contact, rallies, restructuring, support or withdraw, defensive short rest periods, and so on, as the Heroes try to accomplish their goals.
Really large fights with 50+ combatants can run upwards of 10-15min per round. This should be avoided. It creates a lot of dead time for most of the participants.

\

There are more rounds per battle than in most common games, but each round is short and quick. Exploring and clearing a small dungeon can creep into 50+ rounds, but can be done in an hour or two even when explaining the rules to a first timer. This has been demonstrated many times in the dreaded \texttt{Dungeon of Testing} scenario, used to introduce new players to the game.
%demonstrated in the introductory screencast

Standard encounters for a few players and similar number of regular opposition take perhaps 20-30 rounds and 20-40min. More complex fights take longer. In the advanced \texttt{Goblin Destiny} campaign playthrough we had six players control two goblins each, plus sidekicks and allies, and there were usually 20-40 fighters on the board in most battles. Then the rounds can sometimes creep upwards of 10min. This is not recommended. Keep battles small and fast, as that is more enjoyable for the players.
% I consider the time consumption a design failure on my part of that specific campaign. I should have done it differently. But the players loved the campaign and reported it as one of the best so I have not done any significant changes to it.

Several times we've had dungeon crawl adventures stretch to hundreds of rounds with skirmishes and lulls. As long as each round is fast this does not take much game time. In some cases the GM can call for things like "take 3r" since he knows nothing significant can happen for a little bit. Or fast forward even quicker. I've found that 3r movement increments are good when exploring through fog of war, movement, facing, lighting, and vision. It works for the players moving and talking, and for me as GM moving and reacting with the opposition under fog of war.

\

One design choice that slows down the gameplay is the action -- reaction dynamic and the possibility to mix movement and actions. This will take longer since it switches between players very often. I belive the benefit is worth it. It gives the reacting player options on how to tackle problems imposed on the Hero and can give rise to interesting reaction chains allowing to better collaboration and coordination between Heroes.

For the alternative design, see e.g. Space Hulk where the target of an action has no options for reactions, and the target defense is baked into the attack roll. But it is faster since the one player runs through a whole round of all his units before switching to the other player. Note how much slower it is to advance against overwatch.


\section*{simplified or complex}
%-------------------------------
The fundamental design idea is to allow the GM and players to choose which level of complexity they want for their game. The simplest basic rule set is very fast and small enough to keep in the head or with a small paper. Easily assisted with some markers or counters, and simple tokens or minis. Good for quick physical tabletop gaming sessions, and quite friendly to small kids as long as they can count to 10, add and subtract.

As shown in the
\hyperref[sec:basicenough]{\texttt{Ominous Crown} adventure example}, page~\pageref{sec:basicenough}, and the
\hyperref[sec:youngkids]{\texttt{Bitter Candy} recounting}, page~\pageref{sec:youngkids},
a very small set of rules and skills, with very simple implementation, is enough to deliver a fun, fast, and varied game experience. A set of fewer than 10 skills and spells gives enough variability and agency for the players to feel they are in control, and their choices matter. \emph{Really don't need much.}

Then the group can pick and choose the rules, skills, etc which they feel add interesting gameplay, increasing variation and complexity. In total this book has over a hundred optional skills, spells, maneuvers, and abilities, many of which drastically change how the game plays.

At some point it's better to have automation assistance by simple scripting in digital / virtual tabletop software. I have a maptool implementation that covers the majority of the rule set for quick automation for our week night internet based slaughter sessions.


\section*{more freedom, more options, fewer musts, deeper tactics}
%-----------------------------------------------------------------
This is not a class based game and does not require a specific typical party makeup to function. Look for synergies in game style and tactics, but there are no "must have" classes or professions in a party composition. Consider: There are many more dimensions to the fighting here than in most regular tabletop and rpg games, allowing for more flexibility and larger option space. Positioning and map control is very important. Player coordination is essential. Healing is slow and very expensive, not generally a mid battle possibility. Just these basic traits mean it plays very different from for example D\&D, and more like tactical tabletops such as Space Hulk and Descent.


\section*{balance}
%-----------------
The fundamental design allows for a wide variety of characters to be "balanced" in the sense that a lot of different build types can have similar impact on an adventure at similar xp levels. It depends on what the adventure is about, what situations they encounter, opposition composition, etc. They will play drastically different and will not be equally potent against different types of enemies and encounters. This is by intent.

When balancing encounters to your party you are in effect balancing your own play style of the monsters against your players' play style of their characters. The tactical depth of the game means it's next to useless to say that X of monster A and Y of monster B will be a good balance against four heroes at around 300xp. It totally depends on how evil and clever you want, \emph{and can}, play versus your players. Again, the old Space Hulk and Descent games were totally different games if your Genestealer opponent had IQ 85 or 145, or if your Descent overlord was nasty, clever, and really knew the game versus just playing for coffee, cake, and the social time.

Games with a fixed balance, threat rating, etc, essentially and fundamentally says: \emph{No matter what you do, this is an approximate balance}. For me that's a red flag that tells me it is a game I will not enjoy.

When you start playing with your player group, you'll learn how they play and what opposition they can manage with their current Heroes. Adapt. You'll need to think for yourself here. That said, all adventures have examples of opposition composition, gameplay and playthrough examples available.

\begin{readoutloud}
Sitting in a tavern: Bow-Lars, Hammer-Mike, and Silk-Tongue-Taylor. After a few beers they discuss how they have handled the typical goblin bandits everyone always run into sooner or later.

Hammer-Mike stands up on the wiggly chair and gestures wildly as he loudly proclaims the benefits of hammering away goblins so they can't get hits into him as easily as his less lucky friend Sword-Stephen, now Resting In Pieces. The two of them quickly made stinky minced meat of a wall of goblins. The cave walls were smeared in blood, gore, and brains faster than you can eat a greased pig. Hammer-Mike even gave some of the meagre loot to Sword-Stephen's poor mother.

Bow-Lars leans back and softly starts explaining that it took a lot longer for him to solo the last bandit hideout. He had to run around, take shots when he could and generally keep his distance, sprint and sneak to remain safe. But in the end he got it done with barely a scratch. But had the bandits had any good archers, or brains, or lived in a small cave, he would not have been able to manage it alone.

Silk-Tongue-Taylor smiles and nods his head over to the dark corner of the inn. There sits a trio of goblins feasting on some flamed bird. "I just had a chat with the pretty one to the right. Bribed him with a small sack of gold, and he gleefully slit the throat of most of his gang for me while they slept. So now he and his two buddies tag along as hirelings on most of my expeditions. Really useful to have a sneaky stabby goblin around, you know."
\end{readoutloud}


\section*{let the players help}
%------------------------------
Bring the players into decisions on rules, skills, spells, equipment, mutations, gear, monsters, bosses, etc. My todo list is hundreds of items long, many of which are ideas from players, and quite a few of the skills, spells, items, etc are at least to some part suggested, built, adjusted, or thought up by players.

Be careful with the combinatorial complexity. Adding things easily creates combinations that become game breaking OP. Usually it's possible to tweak and twist the details so it works, but sometimes it's just better to skip it. The game is simply not fun if there are \emph{obvious} selections and combinations. Always strive to present nothing that are simply and clearly better than already existing components without having downsides and tradeoffs.


\section*{make it your own}
%--------------------------
This here is a dire tome stuffed with stuff, but it is all meant as suggestions. You \emph{should} build your own monsters, weapons, skills, spells, adventures, etc. Make the world your own. That goes for the world setting, campaigns, adventures, towns, npcs as well. The adventures and campaigns can be played just fine right out of the box, but more fun if you tweak them specifically to the game style you and your players have the most fun with.

Post your stuff as you go. Don't be shy.







%-------------------------------------------------------------------------------
%T E R M I N O L O G Y
%---------------------

\phantomsection\addcontentsline{toc}{section}{terminology}
\section*{Terminology}


\begin{description}


\item[The Almighty 1D10.]
All skill and trait checks are done by rolling a D10 (1D10) (outcome [1-10], not [0-9]).\\
\textbf{Success level = skill - 1D10} \\
E.g: skill 6, roll 4 = success +2 \\
E.g: skill 7, roll 7 = success =0 \\
E.g: skill 4, roll 7 = failure -3 \\
success =0 is just barely a success \\
success +3 is a good success \\
success +6 is a very good success \\
success +9 is an excellent success \\
failure -3 is a bad failure \\
failure -6 is a very bad failure \\
failure -9 is a critical failure or fumble

A roll of 10 is always a failure, regardless of skill level and modifications. This is to get more chance into the game, so that you cannot guarantee a certain behaviour even if you plan it well and have mighty skills.
It is not a fumble.

\

\textit{The D10 is of course a flat distribution ]0-1[, D100 percentage, with granularity of 10. 1-10 is fast and simple for even pre-school kids to count and keep in their head, while still giving good enough resolution.}


\item[Modification, penalty, bonus (mod):]
all the same, although penalties tend to be bad for you while bonuses are usually good. Modifications are additive to an action's chance of success. \\
E.g: skill 5 mod+3 has a chance of 8 on a D10 to succeed, skill 6 mod-2 has 4.

The \textbf{ams}, total Accumulated Modification Stack, is the sum of all mods added together. A.k.a: Action Modification Stack, modification stack / mod stack / mods / mod / ams / ms.

E.g: walk (mod-3), and below 66\% hp (mod-1), and declaring 5ap (mod-2), gives ams = mod-6. So with mod-6 even a fighter with sword 8 has only a chance of 2 on a 1D10 to hit. Be careful with moving about and trying to do lots of stuff at the same time while being seriously wounded.


\item[Actions and Action Points (ap):]
Attack, jump, avoid, climb, take, sneak, give, drink.
Actions are most of the stuff the character actively does, except for regular movement. Actions cost action points, and most actions cost three action points (3ap).

Action points are declared in the beginning of each round. The more actions points you want to use the more mods you will have on your rolls.

Don't think of action points as units of time. It is not designed to map to time and is thus a poor analogue and \emph{will} lead to broken reasoning, contradictions, and paradoxes.


\item[Movement and Movement Points (mp):]
Each square the character moves costs movement points. Movement speed is declared in the beginning of the round: maneuver, walk, run, or dash, giving a set amount of movement points. The faster you move the more mods you will have on your rolls in that round.

Typically you can \emph{maneuver} (M) 1 square without any difficulty to your actions, or \emph{walk} (W) 3 squares with mod-3 to all actions that round, or \emph{run} (R) 6 squares with mod-6 difficulty, or \emph{dash} (D) 9 squares with mod-9.

So, trying to hit something with your sword, or calculate how to divide the gold you just stole, all while dashing away as fast as you can, is not going to work.


\item[Skill, spell, ability, power, mutation, maneuver, trait, etc:] gives the character the option of performing actions that he otherwise would not have access to, or improves the chances of succeeding with rolls he already has available, or changes the rules or how the rules apply to the character under certain circumstances. The terms are treated as the same thing. They often change rules or add their own rules to the game, which start giving the characters a lot of interesting flexibility and combinations. Some skills can change the rules and logic of the game significantly.

Maneuvers are actions or variations of actions. They add alternatives and flexibility. Think of them as mini skills. Examples are special attacks, movements, etc, or bonuses in certain situations.

\todo Maneuver is also the name of the slowest combat movement mode, sorry about that. I'll think of something else for this and rewrite later on... \emph{Any good suggestions?}


\item[Round (r):] is a measure of time, one game round. A round does not translate to a specific number of seconds. Each round all characters usually move a bit and take one or more actions. All characters have their own turn during the round, but they can often take actions and movement during other characters' turns as well.

A round is split into the following phases: declaration, all turns, end.
During declaration all characters declare movement and action points. Then all characters take their turns. Then the round ends and book keeping is done to prepare for the next round.

Declaration and end phases usually just takes a few seconds each if the players are awake. Most of the time is spent on the characters' taking their turns to move and act, i.e. actually doing something.
Perhaps Set a five or 10 second timer to limit dilly-dallying on declaration if you have players who can't make up their mind.


\item[Square (sq):] is used as a measure of distance, i.e. the length moved between two squares on the game map. The game is generally played on square tiles, but works just as well on hexes or gridless maps with some very minor adjustments.

The sq distance on diagonals is usually measured 1-2-1-2-..., meaning the first diagonal counts as distance 1sq, while the second counts as distance 2sq.\\
In some cases the sq is also used as a surface area.

When playing on gridless maps or real world tabletop just define a unit distance as 1sq. E.g: 10 mm is 1sq, or 1 inch, or the length of the eraser you happen to have 10x of lying around, or the circumference loop around your finger when creating movement speed twine strings and crossbow range measurement tapes.


\item[Base contact:] is when two characters are standing in adjacent squares, connected by a side or corner. Or if playing on gridless map; when the characters are within their base radius, or contact radius, or simply if the figures or character representation minis or toys touch.


\item[Round Down:] (rd) is the norm. When a value is written as x/y it's assumed to be rounded down to nearest integer unless otherwise stated. E.g: 10/3 = 3, 11/3 = 3, -10/3 = -4.
Unless explicitly written as "round up" (ru) or "round nearest" (rn) assume it's intended to round down. In some cases "round towards zero" (rz) is used, e.g: 11/3 = 3, -11/3 = -3.


\end{description}









%-------------------------------------------------------------------------------
% B A S I C   R U L E S
%----------------------

\phantomsection\addcontentsline{toc}{section}{basic rules}
\section*{Basic Rules}
%---------------------
We start with the bare bones basic rule set, and then expand. Even the very simple rule set is enough to build deep tactical game play, and is very fast.

The more you try to do in each round the more difficult each action will be, and the faster you move the more difficult things get. Movement and actions can be taken in any order and split up over a round. Higher initiative characters can go first, last, or interrupt a character with lower initiative. And when being the target of an action you always get one action for immediate response, which cannot be interrupted, regardless of initiative.

In the beginning of each round you decide how many action points and movement points you want to have available for that round. Unused points disappear at the end of the round. You take mods depending on how many ap and mp you declare.

With some skills you can try to use more points than you have declared, but you get penalties that continue through future rounds.


\subsection*{All effects, skills, modifications and penalties stack together}
%----------------------------------------------------------------------------
The general goal is that all effects and skills stack together unless stated otherwise. This allows powerful combos and and multiple characters can work together to accomplish mighty results. Some of the combination results can be a bit strange, but should hopefully add interesting depth and options.
All penalties and mods also stack together and will in severe situations make success of even the simplest actions a remote possibility at best.

Some combination effects might be bugs that have not yet been seen in game nor foreseen during writing and testing. \emph{Let me know if you find weird or broken stuff.}


\subsection*{All rolls are affected by the mod stack}
%----------------------------------------------------
All rolls: skill rolls, str and dex rolls, balance rolls, climb rolls, swinging a sword, shooting a bow, casting a spell, etc, are affected by the accumulated modification stack, \emph{ams}. Unless it specifically says the roll ignores ams.

Some actions take no mods, or some other set of modifications. E.g: jump, possum, etc. But there are actually quite few exceptions, and they are clearly marked and explained.

\

\todo I know there are exceptions I've missed/forgotten to clarify. \emph{Please let me know when you find them!}


\subsection*{Simultaneous turns during rounds}
%---------------------------------------------
All characters and monsters take turns during every game round. Generally each combatant makes a short series of movements and actions. Usually one or two attacks and/or parries, and moves a few squares. Movement can be split up, and occur before, during, or after actions.

Characters act in order of initiative. Characters with high initiative can choose to act before or after characters with lower initiative. Characters with higher initiative can also interrupt the movement and action sequences of characters with lower initiative.

In practice, characters will be performing actions during other character's turns. This will in effect often yield simultaneous turns during a round, just with different action timing flexibility depending on relative initiative.


\subsection*{Fuzzy time}
%-----------------------
This rule set is built around a flexible and fuzzy outlook on time. Before and After is usually preserved quite well, but can fray at the edges when it includes more than two people.

This is not realistic, but it does make the fights more interesting. It adds tactical depth. Actions and movements are often carefully arranged within the rounds to maximise effects of cooperation, to take advantage of temporary opportunities or limit exposed weaknesses.


\subsection*{Movement}
%---------------------
All movement cost movement points (mp), usually 1mp per map square. In the beginning of each round all characters declare how many movement points they want for the the round by declaring their movement speed: maneuver, walk, run, dash, (m,w,r,d) further explained below. The faster they want to move the more difficult things will be.

Movement across square sides cost one movement point. Diagonal movement across a square corner costs one or two movement points depending on how many diagonal moves you have made before in that round. Every second diagonal move costs two movement points. I.e: 1, 2, 1, 2. Clear the count for each new round.
This assumes a square grid. Hex boards usually allow only side moves. On gridless boards, or the living room floor, use a direct measurement scale and round up to nearest whole movement point if necessary. I suggest cutting movement strings or paper strips for the character's different speed modes: m, w, r, d.

Some terrain types or obstacles are difficult to pass, and may cost more movement points. It is always the square you move \emph{from} that decides the cost of the move. E.g: Moving into a double mp shrubbery from flat grass costs only one mp, while moving within a double mp shrubbery costs double mp and moving out of a double mp shrubbery to flat grass costs double mp.
%\\ \todo perhaps change this to : the square you move \emph{to} that decides cost. Simpler to see, even if it reduces planning requirements.

If a character tries, or is forced, to move more than he has movement points to do he will fall down into the square he's moving to, going prone, and get a persistent mod-3 to the rest of the round and the entire next round. Skills like off balance, defensive step, etc modifies this.

Diagonal movement around corners and between objects is usually restricted.
Objects or terrain with corners that leave a lot of space to the corner of the square can be treated as "round corners" Friendly Heroes also have "round corners".
Object corners that go mostly or completely to the corner of the square are "sharp corners". Enemy units also have "sharp corners".
Moving diagonally around a "round corner" is acceptable, while you cannot move diagonally around a "sharp corner". In some rare cases it is possible to squeeze diagonally between two objects when both have "round corners", but not when either or both have "sharp corners".


%\small \begin{verbatim}
% -------
%|       |
%|       |
%|       |
% -------
%\end{verbatim} \normalsize


\subsection*{Taking Action}
%--------------------------
Actions include attacking an enemy, parrying an incoming blow, giving a quick order, kneeling down, drawing a flask from a belt, casting a spell, or most other interesting things the Heroes might want to do. Moving is generally not an action in itself, but can be done before, during, or after an action, as part of that action, or separately.

Actions cost action points (ap). A normal action costs 3ap, but some are faster, and others are slower, affecting their ap cost. In the beginning of each round all characters declare how many action points they want to have available for that round. The more action points they choose the more difficult all actions get.

The difficulty modification is the difference between declared ap and the character's baseline action points. The mod is always negative or 0, you don't get a positive mod from declaring fewer ap than your base. E.g: having base ap of 4 and declaring 6ap gives mod-2, and declaring 3ap gives mod=0. The maximum ap a character can declare is his base ap + dexterity.

Most characters start out with base ap of 3, meaning they can do about one action per round without extra difficulty modifications. The characters can increase their base ap by training skills like \emph{quick}.

The ap declaration modification stacks together with mods from movement, etc.


\subsection*{Movement gives action penalties}
%--------------------------------------------
Movement speed is declared at the start of each round. I.e: maneuver, walk, run, dash (m,w,r,d). The faster speed you select the longer distance you can cover, but actions will be more difficult to perform. Movement mod penalties stack with all other penalties and set a baseline difficulty on the mod stack for the round.

\

\small \begin{verbatim}
Movement base modifications:
maneuver mod-0
walk     mod-3
run      mod-6
dash     mod-9
\end{verbatim} \normalsize

\

The faster move speed you declare the more movement points (mp) you get to spend. It depends on the character traits how many mp the different movement speeds give. The speed is also modified by skills and equipment.

\

\small \begin{verbatim}
Typical base movement: maneuver walk run dash (m w r d):
human fighter     m1 w3 r6 d9
dwarven warrior   m1 w2 r4 d6
elven bladesman   m2 w4 r8 d12
\end{verbatim} \normalsize

\

E.g: Heroic Herman wants to move 2mp . He then declares movement speed Walk and must take a mod-3 penalty to all actions that round, even if the action is taken before HH moves.

Movements can be split around actions. However, other combatants can often react to actions before character can move again. E.g: Heroic Herman declares run, moves 4sq, attacks Monster Mike but misses. MM then attacks back before HH can move away. HH chooses avoid and yield and then moves two more steps, now out of harms way. Until next round... This is the \emph{right to react} rule, (\hyperref[righttoreact]{see below}).

Some characters get different amount of movement points for a given movement speed, but the modification is the same: M mod=0, W mod-3, R mod-6, D mod-9. So regardless of whether you have dash 7mp or 11mp you still take mod-9 when dashing.


\subsection*{High initiative selects order of actions}
%-----------------------------------------------------
The order of different characters' actions during each round is decided by their relative initiative. Characters with higher initiative can choose when they want to perform movement and actions relative to the timing of opponents with lower initiative. They can go before slower characters, or wait and then interrupt others' actions when it suits them. The faster can thus also split their different actions to fall between the actions of slower characters.

Initiative is based on the character's dexterity,
\hyperref[sec:charsheet]{see the character sheet section below},
and is a dex+1D4 roll, re-roll every now and then, e.g. each separate battle encounter, or might change in battle lulls. If you have script automation it's tactically interesting to have initiative re-rolled at the beginning of every round of combat.

Characters with lower initiative must also declare action points and movement speed before characters with higher initiative.


\subsection*{The right to react}
\label{righttoreact}
%-------------------------------
All characters can react to actions being performed to them. This means that a slower character can choose to perform \textit{\textbf{one}} action in response to actions being done to them by faster characters with higher initiative, \emph{before} the faster character can do anything else.
The right to react gives the right to perform one action, including movement before and/or after that action. The reaction action does not have to target the attacking character, but can be a non target action or target any other character.

The reaction action can be interrupted by characters with higher initiative, just like any other action, \emph{but not by the original attacker}.

E.g: Attacking Astrid moves in and attacks Defending Doris. AA has higher initiative and can move and attack before DD can do anything at all. DD can then for example parry the attack with one action as right to react. Or if AA misses her attack DD can make a return attack under the right to react before AA can move out of reach or make another attack.

Right to react continues in a chain if the two keep attacking each other.
E.g: Attacking Astrid attacks Defending Doris but misses. DD attacks AA under right to react, but also misses. Then AA can attack DD if she wants, still under right to react. And so on until they don't attack each other any more due to lack of action points, other interests, moved out of combat, attacked someone else, etc.\\
Note: If AA attacks DD, hits, and DD successfully defends, there is no continuation within right to react. AA took an action and DD reacted. Finished.

A right to react reaction chain can be interrupted at any point by characters with higher initiative, but it continues once the interruption and any reactions have been resolved.

% is this correct, or should a right to react chain always proceed at fixed initiative of the initial attack? There are edge cases hiding here...  :
If using the rule where spending action points decreases in round initiative it is possible that the participants drop in initiative order to the point where other characters can interrupt and intervene in a right to react action chain. However, the right to react by the target will not disappear even if a third party injects an action in the chain. The target still has right to react to the first action even after a second attacker has interrupted the response action and the target has taken a reaction action to the interrupt action.

With multiple people interrupting this can get gnarly, but essentially, just keep the right to react order as a lifo stack of "rights to react". E.g: A1 attacks D1, D1 has a right to react, but A2 interrupts and attacks D1 as well. D1 now has a right to react to A2 first, then a right to react to A1, before A1 can do anything else.

\

Note: defence actions are treated slightly differently from other reactive actions. A defence action, e.g. parry, does not allow for movement or turning \emph{before} the attack-defence sequence is resolved. The defender can still move or turn as part of, during or after, the defence action but only after the attack-defend is resolved and completed at present location and facing.

Attack actions are also a little different in that the attacker cannot move away after an attack before the target has taken their right to react action. But can move away as part of the attack before the target moves as part of the reaction action, provided that the attacker had higher initiative when the attack was initiated.

%E.g: Attacking Astrid strikes Defending Dave. DD choose to parry + yield + move three steps away to negate further attacks from AA. In sequence: 1) the attack is rolled, and is successful. 2) the defence is declared, the yield bonus is applied to the parry, but DD is not moved yet. 3) the parry defence action is rolled, and fails. 4) damage is applied. 5) DD is moved for the yield maneuver. 6) DD is moved for the further movement. Note: that DD could not have chosen to move 3sq then skip the defence on grounds that by then AA would be out of reach. If DD had higher initiative than AA she could have interrupted AA's attack by moving away, but then it would have happened before AA spent the attack ap and rolled for the attack action.

\

This is all simpler than it sounds. Try it and you'll get it working in no time. The reason for "the right to react" is that it creates very interesting tactical depth when there are more than two actors, and gives rise to cool duels and attack -- defence sequences, like old sword fighting movies.


%TODO: decide which resistance roll version to use
\subsection*{Resistance rolls}
%-----------------------------
% original [-5,4] version, which only have one roll :
In some cases it' not relevant to separate the action into separate offensive and defensive rolls. Instead one single \emph{resistance roll} is made, and called for as: "attack versus defence" or "xxx vs yyy".
% or "xxx | yyy", "xxx \vs yyy".\\
The attacker rolls \verb|attack - defence + 5 -1D10|. On success $\ge$0 the attacker has succeeded with the action, with the diff indicating the level of success as with a normal roll. Note that a roll of 10 does not mean an automatic failure in the context of resistance rolls.
%-----------------------------
% the 2018 dual roll [-9,9] version, has issues
%Resistance rolls. "attack versus defence",  "xxx vs yyy" or "xxx | yyy", "xxx \vs yyy".\\
%Both attacker and defender roll \verb|value - d10| and compare their differences. Best diff wins. The difference between the attack diff and defence diff is how good the success is. I.e. success diff is:\\
%\verb|(attack - d10) - (defence - d10) == attack - defence + d10 - d10|\\ If both attacker and defender have the same value this gives an outcome range of [-9,9], where the usual success=0 is a regular success, success+3 is a good success, success+6 is a very good success, and success+9 is an excellent or perfect success. Same with fail-3, -6, -9. Also note that a natural 10 roll has no "automatic fail" effect in resistance rolls.
%
%TODO: skip this ?
%In some cases a resistance roll is requested where the defender must or can spend action points only if the attacker actually succeeds with his part of the roll, i.e. \verb|attack - d10| $\ge$ 0.














%-------------------------------------------------------------------------------
% B A S I C   A C T I O N S
%--------------------------

\phantomsection\addcontentsline{toc}{section}{basic actions}
\section*{Basic actions}


Below is a list of some basic common actions.


\openactionslist


\action{Action: "attack"} make an attack, roll for weapon skill. If you succeed and the target doesn't defend you get to roll for damage and hurt'em. Weapons have values for what damage they can do.


\action{Action: "defend"} try to not get hit by an incoming attack. Defend is done as one of several different forms of defence actions, such as "parry" or "avoid". Defence actions are sometimes done together with optional action maneuvers like "yield" or "dodge" to improve the odds of success.


\action{Action: "parry"} to block an incoming attack, roll for weapon skill.
Parry is a form of "defend" action. Some weapons or attack maneuvers have todefend or toparry modifiers which can make the parry roll more difficult. Some shields or weapons have bonuses to their parry or defend actions.


\action{Action: "avoid"} an attack instead of parrying it. Twist, bend, or move a bit. This requires the "avoid" skill. Avoid is a form of "defend" action. Some attacks have todefend or toavoid modifiers which can make the avoid roll more difficult.


\closeactionslist


\noindent
Attack, parry, avoid are typical combat actions and generally cost 3 action points (ap) to perform. Under normal combat this means the Hero can perhaps do one or maybe two such actions per round. High skilled and quick Heroes can under severe conditions make take more such actions per round, but with reduced chance of success.


\openactionslist


\action{Actions: "give \& receive"} hand over an item between characters. It takes a full round action to give something and a full round action to receive something. Both actions requires a dex+9 roll, or they have dropped the item. Normally this will not fail but when the action and movement mods are piling up the chances can drop.
% give, receive, hand over, handover, relay


\action{Actions: "pick up, put down, drop"} are full round actions with a dex+9 roll. Just dropping things takes no time and requires no rolls, but carefully putting something down and grabbing an item always does.


\action{Action: "draw, sheath/holster/put away"} take something from inventory and make it ready for use, or put it back again, takes a full round as long as the item is easy to reach and not stuffed in a container. E.g: drawing or sheathing a sword, hanging a bow, etc. It's an easy dex+9 roll.


\action{Action: "drink a potion"} is a full round action (1r) and requires a dex+9 roll. This does not include drawing the flask from equipment, or putting it back.


\action{Actions: "crouch, sit, lie, stand"} all takes a normal 3ap action that requires no roll. Rising after falling or when off balance is a normal 3ap action with a dex+9 roll.


\action{Action: "jump"} an obstacle or chasm is a full round action and requires a dex roll, modified by distance and speed. A mod-3 for every square the character tries to jump over beyond the first. E.g: jumping over a one square chasm is a regular dex roll mod-0, a two square burning pit is a dex-3 roll, and a 4sq stream is mod-9.

Jump action does not take any negative modifications from movement speed. It instead gets mod +speed/3, where speed is the speed of approach: the number of squares moved in a straight line \emph{before} the jump, (\hyperref[approach]{see below}).
E.g: Dashing 7sq before the jump gives Leaping Laban a jump mod+2.

The jump movement also costs movement points as normal. If the character moves further than his max movement including the jump he will continue the jump but fall on landing unless he can go off balance enough to cover the difference.

On a fail -1 to -2 the character came up short and missed the mark but managed to grab hold with his hands. He now hangs from the ledge, if that is possible on the material and square, and must climb up next round. On a fail-3 or worse jump roll, the distance short is fail/3. Then the character is probably on his way towards a spiky/fiery/wet meeting further down.

The jump skill will improve the odds when jumping. A roll of 10 is not an automatic failure for the jump. The result diff decides. Note that the jump action roll ignores the usual movement mods, but it does not ignore other mods to actions such as having declared high ap, low hp, off balance, or other effects.

Jump rolls are modified heavily from being over encumbered, with mod-3 per over encumbrance point instead of the usual mod-1.


\action{Action: "climb"} a wall or tree. Allows to climb one vertical square per full round action and requires a normal dex roll. Fail -1 to -2 means the character is stuck this round. Fail-3 means slip down one square, and fail-6 means falling. Each climb action also costs one movement point (1mp) even if no movement is made.
%\todo: remove: A roll of 10 is not an automatic fail for the climb. The result diff decides.

Climbing is modified by the difficulty of the wall. A regular cliff face or uneven castle wall is mod-0. A smooth castle wall is mod-3 or mod-6. A simple tree is mod+3. Equipment such as rope is also helpful, e.g. mod+3. Claws help on suitable surfaces, tail or tentacles on others, and so on.

The climb skill will improve the chances of climbing. Climbing assumes the character has nothing in his hands. One hand occupied is mod-6. Heroes with extra hands get a mod+3 per extra hand. Carrying a heavy load makes climbing difficult. Each encumbrance gives mod-3 to climb.


\action{Action: "swim"} or sink. Swimming is a full round action and the Hero moves 1sq/r. Roll dex or take con penalty. If your con drops below 0 you start drowning. Roll well and regain control and con.\\
\verb|con = +diff/3 (ru)|
E.g: fail-4, loose 1 con, fail-6 loose 2 con, success+2 regain 1 con.\\
Resting out of the water regains 1+con/3 con each round.

Just staying afloat and threading water is mod+3 and does not fail on a roll of 10, but the Hero does not move forward.

Every round the Hero is $\leq$0 con he must roll against psy modified by con or be panicked for the round. Panicked Heroes the psy fail diff as mods to all actions.

%\TODO: old "swim" skill version:
%Roll each round and every fail gives diff/3(ru) mod to con. Recover +1 con each round of successful swimming or out of water. Just staying afloat is mod+3 and also does not fail on roll of 10, but gives no movement.


\action{Action: "unpack/rummage"} is used to take something out of a container, like a backpack, sack, etc. It takes 2+1dX rounds: \\
One round to take off / open the container. \\
1dX rounds to find and take out the item. X = number of items in that compartment of the container. Where a successful int or per roll will halve the time.\\
And finally one round to close and put back the container.


\action{Action: "stow/pack"} is used to put something back in a container. Takes three rounds: 1r take off / open the container, 1r put the item back, 1r close and return the container.


\closeactionslist


\subsection*{Other actions}
%--------------------------
Basic actions like the ones listed above or common to the Heroes' normal lives are generally easy enough under normal circumstances, or require some char stat roll in difficult or hectic situations. Other actions are only possible if the Hero has trained the required skills. E.g: swim, tackle, block, wield a weapon, haggle, read, write, count, hide or camouflage, follow non obvious tracks, find food in the wilderness, etc.











%--------|---------|---------|---------|---------|---------|---------|---------|
%       10        20        30        40        50        60        70        80
%-------------------------------------------------------------------------------


\phantomsection\addcontentsline{toc}{section}{character sheet}
\section*{The character sheet}
\label{sec:charsheet}

The character sheet holds the basic information describing the character, and determines his chances for rolls, actions he can take, and so on. Just like most rpg games. It is stripped down to be small and simple. If you want it even simpler just remove the stuff you don't need. Find examples in the monster section.

The specifics of rolling characters is explained in the "campaign" chapter.

When playing, always keep track of the original rolled stats of the character. E.g: take a photo with your phone. Or, write the current stat first, summed up from the base original stat and all modifications from skills, abilities, equipment, etc. Then write the original base stat behind, in parenthesis. E.g: "hp 14(11)". This way it is easy to track and recalculate the current stat values. With a lot of skills and modifications, as might happen with experienced characters, it's good to have the original baseline written down.

\

\goodbreak
\begin{samepage} \begin{verbatim}
===================================
name                        (token)
-----------------------------------
str          hp abs
dex          m w r d
con          stamina
int          vision arc
psy          mana
per          ap
cha          xp
----------
skills
----------
spells
----------
equipment
money ( g s c)
===================================
\end{verbatim} \end{samepage}


\subsection*{First column character stats}
%-----------------------------------------
\begin{description}

\item[str] Strength is the physical strength of the character. It determines how large weapons he can use, how much he can carry, what his chances are to prevail or loose in a wrestling bout, etc.

\item[dex] Dexterity is the character's balance, agility, coordination, precision, etc. It determines if he remains standing after a tackle, if he falls when running over the rubble, how fast weapons he can use, etc.

\item[con] Constitution is the general physical sturdiness and resilience of the character. It determines his chances of keeping running and fighting when getting tired, his chances of staying awake and alive when he gets lethally wounded, as well as his resistance against poisons, and the base distance he can travel in a day.

\item[int] Intelligence is the reasoning capacity, cleverness, schooling and general knowledge of the character. It determines his chances of figuring things out, putting clues together to reach a conclusion, recalling some piece of knowledge, etc. It also determines how easily he can learn spells.

\item[psy] Psyche is the mental strength and resilience of the character. It determines his chances to avoid panicking when faced with scary stuff, of maintaining control over his faculties when opposing wizards try to manipulate him, etc.

\item[per] Perception is the character's general alertness to his surroundings. It determines if he manages to spot hidden monsters and items, as well as incoming attacks from behind, etc.

\item[cha] Charisma describes how pleasant or trustworthy the character seems to be. It determines if npcs will like him or not, if henchmen will stick around, what information an evening in the ale house will yield, as well as his chances of acting and bluffing, etc.

\end{description}


\subsection*{Second column character stats}
%------------------------------------------
\begin{description}

\item[hp] Hitpoints is the amount of damage the character can take before running the risk of falling unconscious or dying.

\item[abs] Absorption is the amount of damage that the character's skin, clothing, armour, etc will subtract from incoming attacks.

\item[m w r d] Movement: maneuver, walk, run, dash. These are the four movement speeds of the character. It determines how many squares the character can move in one round, as well as how difficult he is to shoot when moving, his chances of jumping across the chasm, etc.

\item[sta] Stamina is the endurance under high physical activity. It determines how long the character can move at high speed and how many strenuous actions he can perform before he has to rest.

\item[vision] Vision is the distance the character can see clearly, as well as the kind of light sensitivity he has. E.g: "20 dusk" means that the character can see 20sq clearly, and that his eyes are light sensitive enough to retain full sight under dusk-like brightness conditions.

\item[arc] Arc is the total angle of vision, centred around the facing of the character. I.e: the character sees arc/2 degrees to each side of his facing.

\item[mana] Mana is the magical energy the character can spend on casting spells and empowering magical equipment and such.

\item[ap] base Action Points determine how many action points the character can declare in a round before taking action mod penalties.

\item[xp] Experience points are the current and total experience the character has accumulated through his life. The current xp is the "xp pool" that can be used to buy new skills, use special meta skills, etc. Keep the total XP in parenthesis behind the current XP pool, e.g: 14(195). Some effects in the game are calculated from the total XP.

\end{description}

For simplified tabletop gaming the vision arc should be ignored or set to values that are directly easy to track on the board. E.g: arc 90, 180, 270, 360\degrees on a square board, since that is quick to trace visually. On a hex board it's easier with 120, 180, 240, 360\degree. With virtual tabletop tools, e.g. maptool, the actual 1\degrees resolution is sometimes supported.

%\
%
%With the basic rule set it is possible to run fast and simple combat with good tactical depth.


% Base stats should cover a lot of actions and activities
% Base stats should be orthogonal. If you have e.g: dex and five other values which depend on dex, then you still only have one value, displayed from six different angles. Just cosmetics. Pointless.





\phantomsection\addcontentsline{toc}{section}{simple enough}
\section*{Basic is enough}
%-------------------------
\label{sec:basicenough}  % Ominous Crown
With just this basic rule set it's enough to run very fast but tactically deep tabletop fights, as well as fun lightweight role playing of course. We've used the basic simplified rules when playing with kids as young as five years old with little or no previous tabletop experience. Smart kids, but still. They just need to be able to count to 10, and add and subtract a bit.

For most basic tabletop scenarios I suggest to skip vision range and arc, stamina, pain, etc. All depending on what is actually relevant for the players' experience of the scenario, adventure, or campaign you intend to play.

Then choose a set of skills to be available to the players, again, only those that are relevant for the game experience of the intended scenario. But enough to allow for some choice that have meaningful impact on what the characters can do to influence the story and how the adventure plays out. See how little is actually needed for the example below.

\

\noindent Let's pretend you want to present your players with an adventure, \texttt{Ominous Crown:}

\begin{enumerate}

\item They start in the village, at the inn. They get hired to go rescue the merchant's daughter. She was kidnapped when playing on the field behind the store. The Heroes can find strange prints in the sand where she played, and follow them through the forest to a dark cave.

\item In the cave the Heroes fight a bunch of goblins, and rescue the girl. But they also find a letter from Lord Ominous ordering the goblins to the Ruined Temple in the Old Forest to find the Crown of Power. He will reward them with lots of gold and small babies to eat.

\item Travelling through the forest they encounter some random animal to fight, evade, or befriend. At the Ruined Temple they must solve a riddle and get past some traps and avoid waking the Undead Priest.
Taking the Crown of Power they also find an old legend, saying the crown is very powerful, granting the king's warriors superior strength. But the crown only works if worn by someone of noble birth, seated on a throne at a very special location. Reading the legend and knowing the history of the region they figure out that the special place is in the middle of the village square. Which means Lord Ominous must be planning to invade the town if he wants to use the crown.

\item Now they must either go fight Lord Ominous in his castle, or waylay him and his warriors when they travel to the village, or get the villagers to prepare for war and defend their home, or find someone else of noble birth to take the crown and help defend the village.

\item Depending on what they choose they now must prepare for the final epic battle. This could mean scouting Lord Ominous' castle, sneaking about and finding a way when he could be vulnerable to attack. Or reconnoitring the roads and prepare to ambush the Lord and his warriors. Or build palisades and train the villagers to fight. Or dig through the village archives to figure out that the long lost great, great, great, granddaughter of the last lord of the land is still alive and must be the blacksmith's adopted daughter.

\item Finish with a serious fight, win or loose depending on preparations and tactics.

\end{enumerate}

Let's see what's needed. Assume first time players, young kids, playing on a dinner table. Keep hidden map sections and fog of war, simplified to exposing rooms and critters as the players enter the various map segments, or simply build the map on the table as they go. Hence, skip vision and arc. For young players with no experience: skip stamina, pain, and low hp mods. For older kids I suggest you keep mods for 66\% / 33\% / 0 hp, and death at negative constitution. This is a good dramatic signal when things start to go wrong and it requires minimal book keeping. Keep mana but skip arrow counter, to emphasise the power of magic, and it's fun to have a small heap of mana tokens on the character sheet for the player to fiddle with. Skip off balance but perhaps keep yield, another good dramatic signal. Skip encumbrance, skip food and drink other than for storytelling flair. Assume they have tent, bed rolls, and a cook pot for the evening camp, and torches when they enter the old ruins. Perhaps they have a donkey, old and friendly, named Goober? He'll kindly wait for them outside the temple when they go explore.

For a very first game, just create a set of archetype Heroes for the players to choose from. A few more than needed. Choose a small set of skills that give the players different options depending on which skills their Heroes have. For the simplest playthrough it's perhaps enough with just a few weapon skills: sword, shield, axe, spear, bow, avoid, and some simple magic spells: black bolt, fire wall, heal. This would be enough to give the players plenty of options for a group of Heroes to hack their way through the campaign. Just assume they can read, everything is written and spoken in Common, and so forth, the Castle has vines growing over the wall at some point if they look, etc. Choose a limited weapons set to showcase differences: 1h sword + shield, 2h axe, 2h spear, bow. So the players get a distinct difference between the relative strength of offence and defence, reach, and ranged fighting. And it will force them to fight better together and find tactical solutions to their respective weaknesses.

If the players figure out or remember to get a piece of equipment or a skill that could potentially be useful for the adventure, let them get it, and see if you can whip up a situation where it turns out to be useful. Makes the player feel good.

\

\noindent This very small set of rules and skills is enough for a great tabletop roleplaying experience with young kids or adults new to tabletop rpgs.


\subsection*{A little bit more?}
%-------------------------------
To give the players some more options and tradeoffs from the beginning,
add: track, sneak, climb, literate, histography, dungeoneering, gossip, animal control and perhaps the spells speak to animals and illuminate history. This can allow, or limit, them to find alternative solutions and extra information, and force them to see that some solutions would be possible if they had chosen differently for their Heroes. For most styles of play that is important! The Heroes are not all powerful, the players must find ways to reach goals while \emph{limited} by their Heroes' abilities.

Perhaps the animals of the forest can warn them about the undead priest in the Temple ruin, or tell which route Lord Ominous' warriors take when they travel. Dungeoneering can help them with temple traps and basic info on Lord Ominous' castle. Histography and literate explains the legend and the throne location to the middle of the town square. Gossip and literate find the long lost noblewoman. Sneak and climb give access to Lord Ominous' castle.

Here I would also suggest to keep stamina as part of the character management. It quickly adds a whole new dimension to the game and is not difficult to manage. It could also be a good idea to look into the difference between one handed and two handed weapons if that was skipped in the simpler version.

\

\

\todo add example summations of the two variations of simplifications.

\

\

With full complexity and some tweaking this little adventure can become a tricky mini-campaign.
















\clearpage
%--------|---------|---------|---------|---------|---------|---------|---------|
%       10        20        30        40        50        60        70        80
%-------------------------------------------------------------------------------
% M O R E   R U L E S
%--------------------
\phantomsection\addcontentsline{toc}{section}{more rules}
\section*{More rules}

More, optional, rules. Ignore or use as you wish. Most extra rules can be picked up in isolation, but some depend on others to make sense. Pick and choose to find what suits your group and game style.


\subsection*{Declaring defence}
%-----------------------------
A defender can wait until he knows the success or fail roll result of an opponent's attack before declaring a parry or counter attack. This way the flow of the battle shifts every now and then, as offensive players "leave themselves open" to counter attacks.

The attacker who just missed and is now under counter attack can spend a new action parrying the incoming counter attack.

Defence action must be declared before damage is rolled, however.


\subsection*{Activating a friend}
%------------------------------------
A character with high initiative can spend an action to activate a character with slower initiative, e.g. shouting at a slow friend to get moving, or shoving a friend aside, out of harm's way. This triggers "The Right to React" just as if the fast character was attacking the slower one.


\subsection*{Fumble, fumbly, fumbles}
%------------------------------------
\emph{I haven't used fumbles in my GnG campaigns thus far.}
%jiro: I think it is fun. I've used it in other games for many years. It can be hilarious. But it takes more time than I feel it gives back in fun.

If a rolled 10 on the Almighty D10 is a failure then roll again for a fumble check: if the new roll is higher than the chance to succeed it's a fumble! Think up some way the fumble plays out: It must be interesting \textit{\textbf{and fun!}} Don't just say it's a fumble and it's always something bad and boring. That just sucks and doesn't improve to the game.

Perhaps a 10 + fumble roll is a regular fumble and a 10 + 10 roll is a critical insane fumble. A 1 in 100 chance is rare enough that you can go for excessive effect.


\subsection*{Perfect, excellent, exceptional}
%--------------------------------------------
\emph{Like fumbles, I haven't used this one either.}
%jiro: same as with fumbly, poor fun/time trade.

If a roll sequence of 10 + fail is a fumble, then perhaps a sequence of 1 + success is a perfect success? Prefer to find a \emph{fun} way of presenting it. Involve the player? Let'em describe the event.

A 1 + 1 is an exceptional rare success. Again, a 1 in 100 chance is rare enough to bring out all the stops and go excessive. Again, perhaps involve the player?


\subsection*{Actions that do not require rolls}
%----------------------------------------------
Some actions do not require any rolls. In that case they always succeed. However, in some situations there might be modifications stacked against them. In those cases the roll-less action is considered a "careful 10", meaning it has 10 of 10 to succeed as base chance, and does not fail on a rolled 10.

E.g: The revival powder can be used to bring a fallen comrade back to life. However, if the recently deceased character had the "Hand of Deity" ability he mods all resurrection attempts with mod-3. In this case the revival powder is considered to have a "careful 10" as base chance of success, and will thus have a chance of 7 to succeed in reviving the unfortunate dead.


\subsection*{Modified resistance rolls}
%--------------------------------------
Resistance rolls are sometimes done as a second step of an action roll. In these cases the resistance rolls are sometimes set to be modified by the success diff of the initiating action.

%TODO: rewrite domination/dominate to instead use skill check then psy check each round.
%TODO: rewrite this example, don't use domination:
E.g: Magus King rolls for "domination" against minion Weakie. Domination calls for a skill roll + mod psy vs psy. So, Magus King has domination 6, rolls a 4, which gives diff +2, i.e. he succeeds with +2. That +2 is now used as mod for the psy vs psy resistance roll. Magus King has psy 6, minion Weakie has psy 4. Thus the modified psy vs psy resistance roll is now 6+2 vs 4, i.e. 8 vs 4.\\
% vs version: attack - defence +5 -1d10
Magus King rolls mod psy vs psy : \verb|8 vs 4 : | \verb|(8 - 4 +5 -1d10)|, rolls 3, wins the resistance with +6. Minion Weakie is now under his control.
% vs version: (attack-1d10) - (defence-1d10)
%. So Magus King rolls \verb|8-d10| and minion Weakie rolls \verb|4-d10|: Result \verb|8-9 = -1| and \verb|4-7 = -3|. Diff \verb|-1 - -3| is success+2, and minion Weakie is now under his control.

%A mod resistance roll macro for maptool: \\
%\verb|/roll 5+mod+attack-defence-d10| \\
%A result >=0 is success, <0 is failure. \\


\subsection*{Assisting with actions}
%-----------------------------------
Some actions and skill checks for the character can potentially be assisted by other characters. The assistants roll and add their success diff/3 as positive mod to the primary Hero's roll. GM discretion regarding who many can assist, how long time they have, etc.
% think D&D have something similar, calling it skill challenge or some such














%\phantomsection\addcontentsline{toc}{section}{initiative}
%\section*{Initiative}




\subsection*{Declaring extra action points raises initiative}
%------------------------------------------------------------
Each ap you declare above the character's base ap will give initiative+1.
E.g: Fast Fabian has max ap 5 and declares 7ap for the round, raising his initiative by 2, pushing mod-2 to ams.


\subsection*{Taking action lowers initiative}
%--------------------------------------------
And when taking actions, each ap you spend will decrease initiative by 1. The initiative is decreased when the action is completed.
E.g: Making a 3ap attack decreases initiative by 3 after the attack is completed.


\subsection*{Fast movement raises initiative}
%--------------------------------------------
The faster movement you declare the higher your initiative becomes. This is to make it easier to move and regroup which creates more flexibility in skirmishes.

\

\begin{verbatim}
Modifications to initiative from declared movement:
maneuver +0
walk     +3
run      +6
dash     +9
\end{verbatim}

\

E.g: Flightful Fred sees Mauling Morgan approaching. FF has initiative 7, MM has initiative 9. FF must declare his movement before MM since he has lower initiative. FF chooses to declare run and gets a +6 to his initiative resulting in 7+6=13. If MM wants to be able to move into strike position first, he must declare a run as well, since a walk (9+3=12) is not enough to move in before FF runs away.


\subsection*{Moving lowers initiative, just like actions}
%--------------------------------------------------------
\label{sec:movinglowersinitiative}
Each initiated movement, not each movement point, reduces initiative by 3. E.g: Running Runar has declared speed R, and gotten a initiative+6 bonus for it. He takes a small 2mp movement and his initiative drops -3. He takes another 3mp and his initiative drops another -3. Taking his last 1mp his initiative drops another -3. If he had taken only one movement, of 1-6mp, his initiative would only have dropped by -3.

%jiro: perhaps relevant here to note that this could remove the usefulness of enforcing that all movement costs action points even when done alone without any actions.


\subsection*{Streamlining initiative list: waiting}
%--------------------------------------------------
When there are lots of Heroes and NPCs on the map things tend to get chaotic, especially when high initiative characters choose to wait until others with lower initiative have acted. This leads to having to loop through the upper parts of the initiative list again and again while only the lower initiative characters are acting for the most part.

To get away from this problem we add:
\begin{description}
\item[wait on other] will set the initiative of the character to one lower than the target he's waiting for. It does not cost any AP. The waiting character moves down the initiative list and gets a new initiative, one lower than the character he is waiting on. The waiting character must be able to see the target he's waiting on.
\item[wait] will retain the initiative but costs one AP. This flags the character with a clearly visible "wait" marker in the initiative list and makes it clear he's ready to jump into action at any time. The player must clearly and quickly speak up when he wants to take action again. He will not be asked by the GM.
\item[done] flags that the character is more or less done for the round, unless something happens to him. The character's initiative is set to 0 and he moves down the initiative list. Mark the character with a clear "DONE" marker.
%\todo setting done should give a bonus to next round to encourage it. Can't use a penalty mod-1 since that would slow things down.
\end{description}

With this it's possible to enforce a rule where the top initiative always have to take some action or choose a wait state. The Hero can wait on a target for free, but sacrifices initiative, wait flexibly, without target, costing 1ap, or flagging as done but can only take reactive actions if the situation around him changes significantly.

For battles with 30+ characters involved this will speed up the process significantly.

The GM must decide what constitutes a \emph{significant change of situation} for a done character. A good rule of thumb is that the character can take reactive defensive actions and movement only, and no offensive actions, counter attacks, pursue targets, initiative interrupts, etc.

%jiro: an alternative could be that any form of waiting always cost 1ap for each time the initiative list loops and the waiting character is on top.


\subsection*{Initiative of persistent area effects}
%--------------------------------------------------
Persistent area effects, e.g. fire spells, poison clouds, etc either have an initiative of their own or inherit the initiative from the action that created them. They act and apply effect in order of initiative just like everything else.

E.g: At initiative 9 Wizzy Wizard casts a Fire Ball with duration 3r. It does damage immediately when it's created, then again at initiative 9 in each of the following 2 rounds. Anyone stuck in the fire at the end of the first round better make sure to have initiative above 9 next round and enough mp to be able to escape, or take damage again at initiative 9 when the fire effect activates again.

Also, anyone running into an effect area has that effect applied immediately upon entry, but only once per round even if leaving and entering multiple times in one round or if the effect activates after the target entered and was affected.













%\phantomsection\addcontentsline{toc}{section}{movement \& facing}
%\section*{Movement \& Facing}




\subsection*{Moving}
%-------------------
Moving a character is not an action, but it follows the rules of initiative. Movement can be done by itself or as part of an action.

\emph{Stutter Step:} To avoid players making small incremental moves without actions the GM can decide that movement together with an action is free, but movement without an action costs 1ap. And if players cleverly start using free 0ap actions to trigger movement the GM can force a cost of 1ap there as well.

Moving lowers initiative, \hyperref[sec:movinglowersinitiative]{see section Moving lowers initiative}, page \pageref{sec:movinglowersinitiative}. Each initiated movement path lowers initiative by 3.


\subsection*{Facing}
%-------------------
On a square grid cardinal and diagonal facing is used: N,NW,W,SW,S,SE,E,NE.
On a hex grid only the six main faces are used: 0,60,120,180,240,300\degree.

The facing of a character is generally the same as the direction of the last leg of the previous movement per default, but can be changed by turning. E.g: Moving Morgan is starting his movement path with 2 steps to the north, then places a waypoint, and then heads three more steps east. His facing at the end of the movement is then east by default.

Maneuver movement is in any direction, and the character does not have to be facing in the direction he is moving. For walk, run, dash the character will be facing in the direction of movement.

Note that turning to any facing is generally free when done together with movement. E.g: Finishing the walk facing east, then just turn the character facing north without spending any extra movement or action points as part of the move, before doing anything else.

It's not free to move then take action then turn for free. In that case the turn is separate or together with the action and may cost ap or mp.


\subsection*{Turning}
%--------------------
Turning can be done by either taking actions, or by spending movement points. Turning follows general initiative just like other movement or actions.

Turning to a specific facing directly before, during, or after a movement is free and considered part of the movement. \\
Spending one movement point separately also allows to turn to any facing, and can be done apart from any other movement or action. \\
One can also turn by spending action points. The cost is dependent on the amount of facing change: \\
Turning 45deg is free with most actions, before or after the action \\
Turning 90deg costs 1ap \\
Turning 135-180deg costs 2ap \\
Certain armour and equipment will make it more costly to turn.\\
On a hex grid 60deg is free, 120deg is 1ap, 180deg is 2ap.

\emph{Note:} Defensive actions such as parry and avoid do not allow a free 45deg turn \emph{before} the action. Offensive action generally do allow a free 45deg turn as part of the action done before \emph{or} after.


\subsection*{Flanking attacks}
%-----------------------------
Attacks from the side or back are more difficult to defend against. Attacks incoming at 90deg from the character's facing are mod-3 to parry. At 145deg it's mod-6 and at 180deg it's mod-9.\\
For hex boards: 60\degrees is mod=-1, 90\degrees (reach) is mod-3, 120\degrees is mod-5, and 180\degrees is mod-9.

The target can not change facing before taking a defensive action. Not even if that action is avoid + yield. He can generally take a free 45\degrees turn together with the action \emph{after} the defensive roll, or pay extra ap or mp for a larger turn, or together with any movement \emph{after} the defence roll.

The target is also not allowed to spend ap or mp before a defensive action unless he has a higher initiative, since the right to react only allows for one activation, where the defend (parry, avoid, etc) must come first.

A target with higher initiative than the attacker can choose to interrupt the attack and turn to face the attacker any way he chooses to negate the mod, but that must be done before the attack is rolled as usual.


\subsection*{Attacking from outside the arc of vision}
%-----------------------------------------------------
When attacking from outside the target's vision cone, but within the target's perception range, the target must roll for perception:

\noindent Success: The target is aware of the incoming attack. If he has higher initiative he can interrupt the attack with his own movement and actions.

\noindent Fail: The target is not aware of the attack, he cannot interrupt, and he has mod-3 to any defence against it.

\

If the attacker is successfully sneaking, his sneak success modifies the perception range and perception roll of the target. If the target fails his perception roll against a sneak attack he is unaware of the attack and can not make a defence under right to react or interrupt with higher initiative. Even if unaware the target still has the right to react after the attack, and can move and take actions as usual.


\subsection*{Attacking to the side}
%---------------------------------
Attacking forward or diagonally forward, within +/- 45deg from facing, is normal and carries no modification. \\
Attacking an enemy to the side ($\ge$90deg) is mod-3. \\
Attacking an enemy diagonally back ($\ge$135deg) is mod-6. \\
Attacking an enemy directly behind (180deg) is mod-9. \\
The enemy must be within the cone of vision for the attack to be possible at all. \\
These are the same mods as for parrying incoming strikes from the sides or behind.

On a hex grid it's: 60deg mod=1, 90deg mod-3 (reach attack), 120deg mod-6, 180deg mod-9.


\subsection*{Small and large targets, partially hidden targets}
%--------------------------------------------------------------
Small targets are mod-1 to hit, very small targets are mod-2 to hit, tiny targets are mod-3 to hit. Large targets are mod+1 to hit, very large mod+2, huge mod+3.

Normal sized targets are humans, elves, dwarves, orcs. Small are goblins and halflings, dogs, panthers, eagles. Very small are goblin runts, small dogs, large cats, snakes. Tiny are

Partially hidden targets are protected by the coverage. Half figure is mod-3, head only is mod-6. Peeping through slit is mod-9.


\subsection*{Partial squares, corners and difficult terrain}
%-----------------------------------------------------------
A square that is only partially clear is still possible to stand on, but will give a mod-X to actions performed while standing there. The typical mod for a partially blocked square is mod-3, while severely blocked squares can be mod-6 to mod-9 depending on surroundings and situation.

Some squares contain difficult terrain, such as rubble, debris, etc. These squares also give mods to all actions performed while standing or moving through the area. E.g: Light rubble and undergrowth mod-3, heavy debris or marshland mod-6, etc.

Moving through difficult terrain or partially blocked squares require a dex+9 roll, modified by ams as usual and the terrain modification of the square.

Attacking around a blocked corner gives mod-3 to the attack. Enemies are not considered to have blocked "sharp" corners for this situation.

Failing a roll when moving through difficult terrain could me the character has to pay extra movement points, or that they fall down, or that they can't move again that round. In some cases the terrain allows for a recovery action, where the Hero can spend 3ap to


\subsection*{Crawl spaces and tight tunnels}
%------------------------------------------
Severely blocked squares may not allow for normal actions and movement, but will only allow for crawling through. Crawl speed is dex/3 sq/r, round down but minimum 1sq/r.

Small character such as goblins and halflings will be able to run or dash in areas where larger characters can only maneuver or walk.
Common narrow tunnels will allow movement as follows: \\
Humans and orcs can only maneuver. \\
Elves can walk. \\
Dwarves can run. \\
Halflings and Goblins can dash.


\subsection*{Small characters can pass friendly diagonals}
%----------------------------------------------------------
Goblins and Halflings are considered small characters and can pass through a "friendly diagonal", i.e: a diagonal between two friendly characters which are considered to have "round corners".
Normal size characters such as humans, orcs, dwarves, elves, etc cannot.


\subsection*{Forced movement, falling, off balance}
%--------------------------------------------------
Forced extra movement, beyond declared, causes the character to fall down and go prone, and forces a mod-3 for the rest of the current round and the full next round. Rising from a prone position takes a normal 3ap action. Falling down can be avoided by purposefully going \emph{off balance}, a maneuver skill which most races have from the start. Controlled falling is done by \emph{face plant / dive} and avoids the persistent mod-3.

When a character goes off balance he immediately suffers a mod-3 for the rest of the round. The off balance mod also persists for the next round. E.g: Hero Albin has spent all his movement to get into attack position on Monster Boo for next round. But then, \emph{supplies}, Monster Boo attacks and Hero Albin wants to "avoid + yield". To do that he must spend one extra movement point which he does not have for the final move of the yield action. This will cause him to go off balance, incurring the mod-3 which will also persist for the next round.

Note that the off balance penalty kicks in after the parry + yield roll in the example above. Otherwise most situations would not gain any actual benefit from going off balance when defending and yielding since the off balance penalty would cancel the yield bonus.

It's possible to have multiple off balance if the character purchases the skills.
When the character tries, or is forced, to move beyond his off balance level he will fall down, just as if he didn't have off balance.


\subsection*{Speed \& distance of approach}
\label{sec:approach}
%-----------------------------------------
Some skills/actions like jump, tackle, charge have bonuses based on the speed/distance moved in a straight line just before the action.

If the character makes a turn in the move it's only the final straight line that counts. It's also the distance covered, not the movement points spent, that determines the speed/distance of approach. E.g: moving 3sq on rough terrain costing 2mp/sq is still approach speed of 3 not 6.

Normally the bonus from speed/distance of approach is the distance/3.


\subsection*{Collision}
%----------------------
One dude running into another, by intent or otherwise, on foot or riding a mount. It's a resistance roll str vs str.
Both parties have the option to take a 3ap action to resist the collision. Taking an action allows that character to use full strength, not taking an action forces him to use half strength. Both parties can add speed/3 to their side of the resistance roll, where speed is the distance of approach.
The winner will occupy the collision target square and the looser will be pushed away or stopped in track, GM decides final location.

Both parties must then roll dex for a balance check, modified by the strength resistance roll diff, success or failure for winner and looser respectively. Failed balance check means falling down.

If mounted use the strength of the mount instead, and add the ride skill of the rider. After collision the mount must roll balance check and the rider must make a ride check, both modified by the str resistance diff/3.

To do it right, use tackle and block skills.










%\phantomsection\addcontentsline{toc}{section}{action duration}
%\section*{Action durations}




\subsection*{Fast and slow actions}
%----------------------------------
A normal action takes "one action (1a)" time to perform and costs 3ap. Some actions cost more, some cost less.

\noindent
A very fast action costs 1ap (vfast / fast 2) \\
A fast action costs 2ap (fast / fast 1) \\
A normal action costs 3ap \\
A slow action costs 4ap (slow / slow 1) \\
A very slow action costs 5ap (vslow / slow 2)

\noindent
Under special circumstances some actions are 0ap instant actions and take no time at all.


\subsection*{Actions that take a full round}
\label{sec:fullroundactions}
%-------------------------------------------
Throwing a javelin takes a full round, that means the character can take no other actions that round. He will fire in order of initiative. A high initiative character can choose to fire before or after a low initiative character.

A full round action is considered to take all the ap the character can declare without incurring extra action mods. E.g: The Hero had max action points 3 when rolled up, then bought quick 2, giving him a base ap of 5. For him all full round actions will take 5ap. This is mostly irrelevant but important for some skills and situations.

In some cases it's relevant to calculate divisions of rounds, in which case a round is generally considered to be three actions (3a), and nine ap (9ap).
%Hence a super fast character ($>$9ap) can in some cases be considered to be able to take more than a full round action per round.

%TODO: change example
%E.g: fast magic 3 allows to cast spells in half the time. A one round spell would then take 5ap.

\

Unless otherwise specified, any action taking 0-9ap can be expanded into a 1r action, regardless of what base ap the character can declare. E.g: Slow Sune has base ap of 4, and want to perform a strange ritualistic wavy-arms-move requiring 6ap. Instead of declaring 6ap and taking mod-2 for the action, and round, he simply expands 6ap action into a full round action, declaring his normal max 4ap, and taking no mods.

For actions taking over 9ap it's possible to expand into multiple round actions:\\
\verb|rounds = ap/9 (round up)|


\subsection*{Actions that take more than one round}
\label{sec:multiroundactions}
%--------------------------------------------------
Some actions take one or more rounds. Spellcasting is a typical example. Those actions retain the maximum mod penalty that they have taken through all the rounds during the performance. The roll is done at the end though, just before the action comes into effect.

E.g: Aiming the arrow. Assassin OilySnake is reloading and aiming his crossbow for three rounds before firing. During these rounds he better be sitting very still, since otherwise he will be taking movement penalties. He also should not take any other actions during this time since that will require him to declare more ap than he can do without modification, and thus incur an action mod. So, in the end of round three he fires, rolls a 5. He has crossbow 4 (shitty assassin), aim mod+1. He succeeds =0. If he had moved much or taken any other actions he would have missed.

E.g: Wizard Willful is casting StaffLight which takes him 5r. He maneuvers slowly the first three rounds, busy building the spell. But in round four he walks and gets a mod-3. In the last casting round, round five, he is again maneuvering slowly. So at the end of round five, he rolls his casting StaffLight casting skill roll, but with a mod-3 for the walking in round four. He has StaffLight 7, rolls a 6, fails -2, and as the spell fizzles out he curses the damned goblin that forced him to move too fast.












%\phantomsection\addcontentsline{toc}{section}{stamina}
%\section*{Stamina}

\subsection*{Stamina}
%--------------------
Stamina is a measure of how long the character can keep up with high power activity, such as hard battling, running, etc. When the stamina runs out the character will have problems performing taxing actions and movements. \\
Each attack costs one stamina point. \\
Defence actions cost no stamina.\\
Maneuver movement speed costs no stamina.\\
Every round you walk costs one stamina point. \\
Every round you run costs two stamina point. \\
Every round you dash costs three stamina points. \\
Some skills and maneuvers also require drawing stamina.

%\todo: should it cost 1 stamina to declare extra AP? or perhaps 1 stamina per extra 3 ap: stamina cost = extra ap / 3 (round up as usual)

Characters regain one stamina at the start of every round. They can rest to recuperate faster.
If stamina reaches 0 or below, they must roll for con for every action or movement declaration that requires drawing stamina. The con roll is penalty modified by negative stamina but ignores the mod stack, etc.
If the con roll fails the character can just stand there catching his breath, still loosing the action ap anyway. Failing a con roll when declaring walk / run / dash means the character will be limited to maneuver movement speed for the round.

With negative stamina, all actions are also modified by the negative stamina, above the normal mod stack.
E.g: Wheezy the Dwarf has been battling heavily and is now at stamina -2. He wants to makes an attack, costing 1 stamina. His has con 7, and must roll con 7 mod-2 =5. He succeeds and can draw 1 stamina. The action is performed at stamina=-2 though, it costs 1 stamina to perform, drawn after the action is completed. He must thus roll for his attack as usual but with an extra mod-2 for the low stamina. Luckily he succeeds. Now he has stamina -3. He decides to do a second attack the same round, and must now roll a con 7 mod-3 =4 to be able to attack or he is too tired. He rolls a 6, a fail-2, and just stands there panting for 3ap instead of smiting the giggling goblin.

A character regains 1 stamina at the beginning of each round regardless of declared movement and activity. Resting characters regains 1 extra stamina each round if they pass a con roll with all normal mods. Since rest means 0ap 0mp the main mods will be pain, low hp, and such.

Some attacks and actions cost no stamina for every second use, or every third, or every second and third in sequence or round. This can be written as \verb|stamina(-1,0,0)| meaning that the first use in a sequence costs one stamina, while the second and third costs no stamina. E.g: if a \verb|stamina(-1,0)| action is done three times it costs -1, 0, -1 stamina.


%TODO: change stamina recovery when resting in maptools
%      see TODO 201220 "change stamina recovery"

%TODO: remove the "con/3 extra stamina" regain when resting. Rests too fast.
%A resting character regains con/3 stamina each round.
%CHANGED TO:
%Resting characters ... above: +1/r if pass con roll with all normal mods

%DONE: the low con regain is currently in the maptool code 191209
%Characters with low con regain extra stamina when resting slower than normal:\\
%con=2 regains 1stam/2r\\
%con=1 regains 1stam/3r\\
%con=0 regains 1stam/4r\\
%con<0 does not regain extra stamina when resting.











%\phantomsection\addcontentsline{toc}{section}{carrying stuff}
%\section*{Carrying stuff}




\subsection*{Encumbrance}
%------------------------
If a character carries too much stuff he gets encumbered and will have modifications to movement and actions.

\noindent
Encumbrance is calculated as: \verb|encumbrance = total weight carried / str.|

E.g: Trekking Tom has str 6, and carries a total load of 4.0 weight. His encumbrance is then 4.0/6 = 0.66 which is rounded down to 0. He thus has no encumbrance modifiers. When he then packs on another 5.0 weight he has encumbrance 9.0/6 = 1.5, rounded down to encumbrance 1. With another 4.0 weight he has encumbrance 13.0/6 $\approx$ 2. He will then have a mod-2 encumbrance penalty which will affect actions and movement.

\

\begin{samepage}
\noindent
Each full encumbrance gives modifications: \\
All actions are mod-1 per encumbrance \\
Dash speed is -1 sq/r and +1 stamina per encumbrance \\
Run speed is -1 sq/r and +1 stamina per encumbrance/2 \\
Walk speed is -1 sq/r and +1 stamina per encumbrance/3 \\
Maneuver speed is -1 sq/r and +1 stamina per encumbrance/4 \\
Long distance travel is also reduced by 1 per encumbrance.
\end{samepage}

Also, stuff has to be carried somewhere. Item slots are hands, one each, shoulder and back, three slots: slung left, right and centred. Some belts have carry slots. Some sheaths and quivers can be fastened along arms or legs, etc. Shields can be carried on an arm or outside a backpack.

Containers are useful in that they only take one slot, while providing several slots for items. Containers are also good in that they can reduce the enc of items. A good backpack halves the enc from all items in it.

Some actions like climb, swim, etc take heavier mods from encumbrance.


\subsection*{Carrying heavy objects}
%-----------------------------------
Some objects require a certain str to be carried. If the character has 0 to 2 more strength he can only move maneuver while carrying the object. If he has +3 strength he can walk, and with +6 he can run with the object, +9 for dash.
Carrying heavy objects costs 1 + required strength / 3 stamina each round.

If several characters cooperate to carry a heavy object they must each have the required diff to be able to move faster, not the diff in total.

E.g: Rock and Cliff are carrying a stretcher with loot. The stretcher requires str 6 to carry. Rock has str 8 and Cliff has str 9. They each carry 3 of the 6 points of weight of the stretcher. Rock has diff=+5 and Cliff has diff=+6. Since Rock only has +5 he cannot run. Even if Cliff had taken a greater part of the load they would not be able to run with the loot, since he would then be below +6 instead of Rock.

%NOPE:  Alternative: \\
%Heavy objects count to encumbrance just like other gear. That stretcher, heavily loaded with loot, might have enc 100.0 or 200.0, making it very slow and stamina sucking to carry around.















%\phantomsection\addcontentsline{toc}{section}{vision, spotting, sneaking}
%\section*{Vision, Spotting, Sneaking}




\subsection*{Visible, sneaking, hiding}
%--------------------------------------
\todo Three different levels of how discoverable or hidden something is:


 *   Plainly visible. Will be found directly when in vision cone, or heard / smelled / felt when within perception range, if possible.


 *   Sneaking. Will be found if passed perception check when in vision cone or by sound / smell in perception range, if possible
Perception rolls can be modified by sneak / hide success, environment, light level, wind, smells, etc.


 *   Hidden, camouflaged. Cannot be found by perception. Must be found by actively looking, using find.
Modified by how well hidden something is, and environment



sneaking. The seeker will always find if he has find skill and take the time to look, it requires

Something / someone is hidden / camouflaged to the point where the seeker need a successful find roll to discover the hidden. The find roll is negatively modified by the success of the sneak/hide/camouflage roll.


This also goes for other senses. Smelling the orc, hearing the metal armour, feeling the cold of the wraith.






\subsection*{Awareness, spotting, finding}
%-----------------------------------------
\todo update passive vs active vs find usage\\
Passive perception\\
Actively looking\\
Using find skill\\

The field of view is the area which is within the view arc, in view range, and sufficiently lit. The perception range is the character's perception value. Perception range and rolls for spotting are modified by the usual action modification stack of the spotter, such as movement, extra ap, pain, low hp, etc. Spotting is best done when standing still, attentive, doing nothing else.

A sneaking/hidden target modifies perception range and perception rolls of the spotter by the success diff of the sneak/hide roll. This is on top of the spotter's own ams mods.

\

A target character or object can either be obviously visible, sneaking/hidden, or in cover by some other object such as hiding in a bush etc. When obviously visible the target is out in the open and not trying to hide. When sneaking/hidden the target is possibly in the open or in some limited obscuring object but having passed a sneak roll. When hidden in cover the target is hiding in some object that obscures vision, e.g. a bush or some debris. A target can also be completely hidden behind a vision blocking object, e.g. behind a wall, but then the target is not possible to spot at all.

\begin{itemize}

\item An obviously visible target will be automatically spotted in the field of view.

\item A partially obscured but otherwise not sneaking/hidden target within field of view can be spotted with a perception roll, further modified by how obscured the target is. E.g: crouching in a bush, mod-3.

\item A fully obscured but otherwise not sneaking/hidden target can be detected (heard, sniffed, felt) when inside the perception range with a successful perception roll.

\item A sneaking/hidden target can be spotted when inside the field of view but outside perception range if the spotter passes a perception roll.

\item A sneaking/hidden target will be automatically spotted when inside the field of view and within perception range.

\item A sneaking/hidden target can be detected when inside detection range but outside field of view if the spotter passes a perception roll with a further mod-3 difficulty.

\item A sneaking/hidden target cannot be spotted when outside the field of view and outside modified perception range.

\item For sneaking/hiding targets, obscuring items and terrain modifies the sneak roll instead of the perception roll.

\end{itemize}

Some targets are hidden very well. In such situations a normal perception roll is not enough, and they must be found by taking the time to search using the "find" skill. Example of this are characters and objects camouflaged or hidden by a multi-round sneak roll.

Calling for perception rolls of course notifies the players of hidden stuff in the vicinity.
%Roll only once for all hidden stuff, and see if the perception success is better than the hidden success of each respective hidden.
The GM can of course roll the perception rolls hidden from the players, but that is a bit more boring. Alternatively the GM can call for per rolls every now and then when there is nothing to find as well...


\subsection*{Sneaking and Spotting}
%----------------------------------
\label{sec:sneakspot}
Sneaking and hiding, both self and other objects, are done with the sneak skill.
Conceptually the sneaking or hidden Hero or object is not necessarily fully hidden from view, just partially obstructed, in shadow, or placed / moving so that it is less likely to be noticed.

The sneak roll is modified by ams as usual and by various environmental modifiers. The success diff of the sneak is applied as mod to the perception / find rolls of the opponents. A failed sneak does not make the Hero easier to notice than if he had not tried sneaking.

Some basic sneak modifiers:\\
area is clean/open -3\\
area has some stuff around =0\\
area is cluttered with large objects +3\\
daylight or otherwise well lit -3\\
lighting by torch and otherwise dark =0\\
night or unlit dark area +3\\
weather is calm, still and quiet -3\\
weather is as usually, some wind and moving vegetation =0\\
weather is loud and everything moves, e.g. stormy  +3

Spending rounds, not actions, to seriously camouflage a character or hide an object will require a find roll to detect them. The environment modifiers are still applied to the sneak roll and the success diff is applied as modifier to the find roll.

Spotting by characters that are inattentive, sleepy, or fiddling with something else is generally a mod-3 to the perception roll even if the spotter has mod-0 ams. Spotting when actively engaged in moving or actions are modified by ams as normal.


\subsection*{How often to roll for sneaking and spotting?}
%---------------------------------------------------------
Under normal action conditions roll for sneak once per round when the character is moving in an area where he might get noticed, in view or detection range. And the same for perception for the spotting opponents.

For quick and easy sneaking around when scouting outside of combat one can simply roll for sneak occasionally when the environment changes to determine the spotting mods. Then only roll for detection by those opponents who have high enough perception, as long as the sneaky bastard doesn't run straight past the nose of the tired guard.


\subsection*{Unaware opponents}
%------------------------------
\label{sec:unaware}
Unaware characters, such as city guards, and goblins too focused on their stamp collections, are easy targets. They have perception mod-3, and will take some time to react. When attacked, startled, or succeeding a perception roll to notice the advancing hero, they will still require an int roll to get in gear. Otherwise they will be surprised and not do much until the next turn.


\subsection*{Surprise}
%---------------------
\label{sec:surprise}

\todo surprise, roll int to get Maneuver mp and 3ap, fail int gets total surprise \\
\todo or perhaps successful int roll allows declaring success diff ap and success diff/3 mp

Unaware and relaxed characters can be surprised. It then takes them one round to gather their wits. Roll int to see if they gather their wits fast enough.

A successful roll means they are not surprised and cannot be considered unaware. They have full normal ap available and movement speed M.

A failed roll means they take the fail diff as mods to all actions for the rest of the round, have only half ap (ru), and can move at most 1sq. They can still expand one action to a full round action if they don't have enough ap.


\subsection*{Long range vision}
%-------------------------------
Past his vision radius the character cannot discern details and must ask the GM what is further away. It is not enough to designate targets properly. Firing at a target beyond vision range is mod-3. Casting spells is mod-3.

















%\phantomsection\addcontentsline{toc}{section}{damage, healing}
%\section*{Damage, Healing}




\subsection*{Area damage, area of effect (aoe), persistent damage}
%-----------------------------------------------------------------
Damage that applies to everyone in an area is generally supposed to be rolled individually for each target. E.g: fire dam 5/r, dragon tail swipe dam 8 knockback 2, stun pulse dam 1 penetrating stun 9.

Area damage that persists over time has a damage per round stat and its own initiative value. First, anyone in the area takes damage when the effect comes into existence. Then, for each round, anyone in the area of effect takes damage at the effect's initiative. Anyone leaving the area before the effect's initiative does not take damage that round. Anyone entering the area of effect at any time in the round takes damage upon entry. Damage application is limited to once per round. Running in and out of fire does not give more damage than just standing still. Generally, persistent damage areas retain the initiative of when it was put into effect. E.g: a poison cloud grenade thrown at initiative 7 will generate a poison cloud area of effect with initiative 7.

Persistence is measured in rounds, including the first round the effect starts. E.g: casting a fire ball dam 5 area 2sq duration 2r disappears at the end of the round after it was cast.


\subsection*{Some special damage effects}
%----------------------------------------
Some attacks have special effects that take place when the attacks succeeds. Sometimes it is required that the effect penetrates armour and does at least one hp damage, sometimes not.


\begin{description}


\item[poison]
Poisons do damage or other effect over time. Most poisons has a few characteristics: strength, duration, and damage per round. Each round roll poison strength vs constitution to see if the poison does damage or not. This will persist until the poison duration expires. Common poisons have str 5, duration 5 rounds, and does 1 hp damage per round. Strong poisons probably have str 10 and does 2 hp damage per round. Other poisons can be slow, and only does damage every other, or third round, etc...

Poisons do not have to do damage, they may just paralyse the victim, confuse it, or have other strange effect. Perhaps the target must pass con rolls or go into frenzy attacking the nearest person.

Antidotes, first aid, and medicine can be used to counter or limit the effects of poisons.


\item[stun]
A stunned target will get a stun mod that affects his actions and movements. Each stun point removes one movement point, one action point, and pushes a mod-1 to the mod stack. At the end of each round the character will remove stun points equal to his constitution value.

E.g: Stunned Steve was bitten by an Elequito and got a stun 5. He immediately looses 5mp, 5ap, and has mod-5 to all actions. He has con 4, so at the end of the round he removes 4 stun points, giving him a stun 1 for the next round. For the next round he starts with 1mp less than declared, 1ap less than declared, and mod-1 to all actions. Before the third round his con removes the remaining stun 1.


\item[web, ensnare, etc]
Web effects usually comes from a spider's spin attack, some magical hold field, or similar. Web effects reduce mobility and give action mods.
A typical spider web effect is dex-2, move-2, mod-3 to all actions until the effect is broken. Multiple web effects stack up to completely immobilise a target.

Some webs have residual effects such as requiring time to clean off. E.g: a persistent mod-1 to all actions and mp-1 to movement until someone spends an action, or several, cleaning it off.

Breaking a spider web is str vs web str action, note that it's modified by the web, or multiple webs in some cases.

Friends can help break loose by adding their str, and can help clean by spending cleaning actions.

Items accumulate mod-1 when used to parry spin web attacks, until cleaned off. Cleaning off requires a regular 3ap mod+9 action.


\item[snag]
The attacker's weapon and target or target's blocking weapon are entangled until the end of the round, unless some cool maneuver or action can untangle them quicker.

It is possible to make a tug of war to rip weapons out of the opponents hands when they are snagged. Roll str vs str (unmodified). If any side wins with +3 or more he has managed to rip the opponent's weapon loose. It's also possible to tug or push a snagged opponent off balance.

Weapons with the \emph{snag} property can be used to intentionally entangle the parrying weapon or shield. Snagging with the snag maneuver is tricky and carries a mod-x. Snag can also happens on a fail-6 or worse with a roll of 10 if you do not try to snag, even with weapons that don't have the snag attack property.


\item[knockback] % knock back
The target is moved one space away for each point of knockback.
Large targets, with size more than one square, ignore one point of knockback for each occupied square after the first one. E.g: a 2x2sq troll ignores 3 knockback points.
Some creatures have knockback reductions even if they are smaller than multiple squares.
The target must pass a dex roll or go prone, mod-3 per extra square beyond the first.

Some knockback effects have a str value which is then used in a str vs str against the target to see if the knockback occurs. Diff modifies the dex roll.


\item[knockdown] % knock down
The target will be knocked down and go prone. Some knockdown have a str value which is then used in a str vs str against the target to see if the knockdown occurs.

Knockdown target does not get a dex roll to try to remain on their feet, knockdown is purely str based.


\item[blinding]
Blinded Heroes cannot make defence actions based on seeing an incoming attack, such as parry, deflect, avoid, etc.
They are still triggered on right to react though, like trying to run away after being hit.

Any attack or other actions that usually relies on vision to find a target, where to put ones feet, etc, need to first first takes mod-3 baseline. Additionally, make both int and per rolls, with any fail values applied as cumulative mods to the action:\\
action mod = -3 + (int fail) + (per fail)\\
E.g: Target Terry is blinded and tries to make an attack. Has total ams-2 from wounds and movement. Rolls success on the int but fail-4 on the per. He now has total mod-9, including ams, to the attack roll.

Taking movement when blinded also require int and per rolls to head in the right direction. Fails will head off elsewhere, GM discretion. Total fail:\\
-1: 45deg off somewhere on the path\\
-3: 90deg off, 45 from beginning then 45 along the path somewhere\\
-6: 135deg off with multiple knees along the path\\
-9: 180deg off with multiple knees along the path

\end{description}


\subsection*{Piling corpses}
%---------------------------
A normal size corpse gives a square a mod-3 poor terrain modifier. Each additional corpse is another -3, to a maximum of -9. Quickly leading to very tricky terrain when standing your ground fighting enemies. At some point it's time to just move and fight somewhere else. This also means that having a heap of fallen enemies covering the squares just in front of you is an excellent defensive strategy. Perhaps you can sacrifice your less useful allies for the same benefits?

Shoving or dragging a corpse 1sq is usually a 1r action, with corpse kick it's faster.


%% TODO remove? this is already covered by the weapons. ? good to reiterate?
%\subsection*{Fast and slow weapons}
%%----------------------------------
%Attacking or parrying with a fast weapon cost fewer action points then with a normal weapon. An action with fast 1 weapon costs 2ap, and an action with a fast 2 weapon costs 1ap.
%
%Attacking or parrying with a slow weapon consequently costs more ap. An action with a slow 1 weapon costs 4ap, and slow 2 costs 5ap.


\subsection*{Breaking weapons}
%-----------------------------
When parrying a damage higher than the abs of the weapon, it might break.
Each excessive damage point gives a 10\% chance of the weapon breaking.
If the weapon does not break, it still takes a permanent -1 abs per every 3 excessive damage (round up).

Excess damage above the weapon abs both continues into the target and damages the weapon.


%\subsection*{Parrying strength limits}
%%-------------------------------------
%\todo parrying strength limits ??
%Parries are limited in how much incoming damage they can block from the attack. Excess damage continues into the target.
%A character can block incoming damage equal to three times the parrying weapon's strength requirements plus excess strength points. E.g: A sword with strength requirement 4 wielded by a character with strength 6 can block at most 14 damage from an incoming attack.
%
%This does not affect the "breaking weapons" rule above. Any damage above the parrying weapon abs will damage both the target and the weapon. Even if the parrying strength limit is lower than the parrying weapon abs.


%\subsection*{Damaging armour}
%%----------------------------
%\todo armour also has hp, and takes damage when overpenetrated.
%E.g: armour takes dam/3 as well. When hp reaches 0 it reduces abs by 1, and resets hp at half hp. Plate armour abs 3 hp 30 will be damaged plate 2 hp 15 then ruined plate 1 hp 7. \\
%? Hmm How to automate ?
%? relevant ?


\subsection*{Hacking through objects}
%------------------------------------
Hacking away at large objects such as doors, walls, statues, etc are at mod+6. If you have skills that improve damage for good strikes it should be quick work.
Some weapons are not meant to be used against objects. Using your sword or axe to hack away at a stone statue or stone wall will have a 10\% chance each strike to reduce the abs of the weapon by 1. Axes works well against wooden objects.

Pick axes and hammers are meant to be used against objects, and suffer no damage. They usually also do extra damage to objects of certain type or material.


% lacking strength or dexterity
\subsection*{Insufficient strength or dexterity penalties}
%---------------------------------------------------------
All weapons require a certain strength. If that is not met all actions with that weapon takes a mod-1 dam-1 penalty for each missing str. For weapons with a minimum dexterity requirement each lacking dex point will cause a mod-1 and cost (lacking dex)/3 round up extra action points.

E.g: Willow the Weak has str 4 and tries wielding a broad sword which requires str 5, he then has a dam-1 modifier and mod-1 to all actions with the sword.
And Fumbly Fabian has dex 5 but his new found exotic rapier of pointy murder requires dex 7. Fabian thus takes mod-1 and +1ap, which means his fast rapier is now just 3ap normal speed. So unfortunate...


\subsection*{Excessive str bonus}
%--------------------------------
\todo meh, rewrite excessive strength bonus handling

Heroes with higher strength than the weapon / armour / action requires can use certain skills and maneuvers to get bonus effects to damage, reduced stamina or action point cost, etc. Each point of excessive strength can only be used for one effect at a time, but enough points allow for combining effects.

Always use the highest base strength requirement of all equipment, maneuvers, skills, etc when calculating the free excessive strength points.
E.g: A character with strength 8, clad in plate mail (str 5), carrying a large shield (str6 at normal speed) and a spear (str4 in 1h grip) has 2 points of excessive strength available.

%TODO: rewrite example, after changes to stamina
\todo: out of date -- fix after stamina changes\\
E.g: str9 with a weapons that requires str3 gives +6 excessive strength points. With the skills "strength bonus" and "easy grip" the extra strength can be used for dam+2 \emph{or} for two attacks that do not require stamina, or for dam+1 and one attack without stamina cost.

%\TODO: new allocation example, including stronk tank
\todo new example perhaps ?\\
E.g: Giant Gerda has str 13,
plate armour (str5)
tower shield (str8 or slow 1 str5)
great axe (str9 or slow 1 str6)


She has slugger 3, stronk tank 2, fast strength.
She can spend 6 excessive strength to reduce her armour penalty class by two levels, to have the plate feel like a leather armour, and

Reallocating extra strength points takes one round per point, unless the Hero has flexible muscles.

\emph{Note:} that high dexterity requirements do not work the same way. High dex is generally just a threshold cutoff. If the character has high enough dex the bonus is available, regardless if there are other bonuses that have dex requirements that are active at the same time.


\subsection*{Reach}
%------------------
Weapons with the reach property can hit targets one extra square away per level.
E.g: a spear with reach 1 can hit an opponent who is separated from the attacker by one empty square. Attacking with reach is usually more difficult and most reach weapons have a mod-x when using reach.
A few weapons are made for fighting with the opponent a bit away and can have modifications when using them in base contact, e.g: reach 0 mod-3 and reach 1 mod=0.


\subsection*{Initiative when attacking against weapons with superior reach}
%--------------------------------------------------------------------------
If moving closer to attack an opponent wielding a weapon with longer reach than yourself the opponent can, if the attack is noticed, interrupt your attack with an attack against you instead, even if you have higher initiative. But then the interrupt attack must be taken at longer reach then you were going to attack at. This does not apply if you are not moving closer to attack.

E.g: Attacking Albin and Defending David stand 3sq apart. AA has initiative 10 and a sword and wants to attack DD who has initiative 6 and a spear. AA must first move in to base contact before he can attack with his sword. However, that means DD can choose to make an interrupt attack at reach 1 mod-3 before AA has closed to base contact. DD cannot choose to make a reach 0 mod-0 interrupt attack.

Ranged weapons, e.g. bows and crossbows, don't have this benefit. There the order of initiative will decide if the attacker can close before the shooter can let the arrow fly.


\subsection*{Pain makes life more difficult}
%-------------------------------------------
Major wounds give pain modifiers to the action modification stack until healed enough to be minor wounds.
For most people a wound gives dam/3 (round down) pain penalty mod that go to the action mod stack and stays there until the wound is treated, pain killers are administered, or some such action is taken to reduce the pain. This is why it is important to track each wound hp on the stat sheet instead of just the total remaining hp.
Veteran is a great skill to have.

Some creatures have different pain threshold and the wounds then give pain equal to damage/threshold instead of dam/3. Most normal sized creatures have damage threshold 3.


\subsection*{Wounds make life more difficult}
%--------------------------------------------
When a character gets severely wounded he becomes weaker and this makes it harder to perform actions. These modifications are not reduced by the veteran skill. As hitpoints drop it gets worse:\\
At 66\% hp he has a mod-1 to all actions. \\
At 33\% hp he has a mod-2 to all actions and cannot dash. \\
At 0 hp he has mod-3 to all actions and cannot run or dash.\\
Black Knight can ignore some mods like this.


\subsection*{Come on and die already}
%------------------------------------
Characters die when they reach -con hp. When a character reaches 0 hp or below he is in very bad shape and in risk of dying. For every round he is not fully resting he needs to roll con modified with the negative hp, ignoring ams. If he fails he is down on the ground, disabled and incapacitated for the round. To get back to action is also a con roll modified by negative hp, ignoring ams.
Incapacitated characters roll to
Passed out characters can roll at the beginning of each round. The veteran skill helps to reduce the modification of the con rolls.

E.g: AlmostDeadDave has 11 hp when whole and con 7. At -3 hp he must pass a con-3 = 4 roll or pass out each round he tries do do anything more than maneuver with no actions. Since he has con 7 he dies at -7 hp.

Some large creatures with very high hp will have a multiplier when referencing con for this effect, e.g: dies at -con*3 or some such. This multiplier also affects the chance of fainting from negative hp, etc.


\subsection*{Regaining hit points}
%---------------------------------
A character generally regains con/3 hp per day. If the character has con = 2 he recovers 1 hp every two days, and with con = 1 he gets one hp every three days. At con = 0 he's in such poor shape that he requires medical or magical treatment to heal.

A daily successful medical treatment roll also heals success diff hp even if all wounds have been treated, see \emph{medicine}. Max one attempt per day.

if tracking wounds list for pain purposes, wounds heal one point at a time in order of acquisition. E.g: if the wounds list is: 5,2,4, then after healing 5hp it would be: 3,0,3. Still giving two pain.













%\phantomsection\addcontentsline{toc}{section}{travel}
%\section*{Travel}




\subsection*{Travelling the region and area maps}
%-----------------------------------------------
When moving around between encounters the heroes probably traverse a regional map. The scale is in leagues, generally at 1 sq = 1 league. A character can walk a number of leagues each day equal to his level of constitution without getting tired enough to have extra mods. The party's movement is therefore limited by the character with the lowest constitution, but can be improved by skills.

The party's vision on the regional map is equal to the highest value of the track skill of any member. Parties without track only have regional vision of the square they are currently occupying.

Navigating the map requires successful track rolls. Following roads is mod+6, plains mod+3, moors=0, hills and forests mod-3, mountains and deep forests mod-6.
Having a map can help, as can other equipment. Knowledge of the region helps. Weather, day or night, affects.
Failing a navigation track roll means the Heroes actually move the wrong way. Roll once or a few times per day, at least every 3-5sq.

Terrain also modifies the distance they can travel per day. Roads, plains and moors are all equally fast to traverse on foot. Hills and light forest takes twice as long. Low mountains and deep forest three times as long, and high mountains four times as long.

Riding is faster. A characters can ride a distance equal to his constitution plus ride skill (con+ride) but also limited by the quality of the horse. A horse has a cruise speed rated in leagues per day. A decent horse is around 10-15 steps/day.

When riding in a wagon or cart the distance is not based on the character's con and ride skill, but only determined by the draft animals travel distance and the wagon's mods. Someone needs the skill ride to drive the cart though.

Travelling by horse and cart is fast and convenient in many cases as long as there are roads or easily navigable terrain such as plains, moors, steppe, or some such. It can be very slow or impossible in forests, hills, and mountains. A donkey with a small cart can perhaps travel 10 leagues on roads, 5 leagues on grassland, 2 leagues in hills, 1 league in forest, and cannot enter mountains. A larger wagon cannot travel in forest either and is severely limited even in hilly terrain.

%Region maps generally have the scale of 1 sq = 1 league. And a league is about how far a character with con=1 can travel per day. A con=5 character can travel 5 leagues per day.
%
%One square on the region maps is about one league of distance. How long that league is, is another matter altogether. However, when describing distances to the players, the distance is in leagues, i.e. in squares on the region maps.

%In historical reality a league was often the flexible distance that someone walks in about an hour on easy terrain. A Hero with con X will travel X leagues per day without being overly tired afterwards. If you \emph{really must} have a real world reference, think of it as about 5km.


Walking along roads or across fields, plains, moors is done at normal speed. Light forest and hills take twice as long, deep forest, low mountains, marshland take three times as long, and high mountains take four times as long.

Riding in light forest takes three times as long. Deep forest and marshland takes five times as long and horses cannot cross high mountains.

Wagons have normal speed on roads. Plains, moors, fields take twice as long. Hills and Light forest takes four times as long. Wagons cannot cross deep forest, mountains, or marshland.

Goblins and orcs on foot have one level reduced penalty in deep forests and marshlands. Elves have reduced penalty in deep forests, mountains, and marshland. Dwarves have reduced penalty in hills, mountains, but increased penalty in marshlands. Halflings have reduced penalty in all terrain.

It is the terrain you stand in that determines the cost of movement, not the terrain you enter. Entering the mountains from the plains is simple normal cost. Exiting the mountains to reach the hills takes three times as long.


\subsection*{Escaping combat, fleeing, giving chase}
%---------------------------------------------------
A character has successfully escaped from combat when he has moved off the edge of the battle map or reached some escape region. It's a simple and fun rule, creating chase scenes and affecting positioning and tactics for increased tactical depth. It's not meant to be realistic.

Allow the track, travel cruise speed, and exhaustion to determine if the pursuers catch up, and how fast. Then make a new improvised battle map to describe where they catch up, based on the terrain in that location on the regional map.\\
If the fleeing party is run down it's the chasing party that has most influence on the positioning on the new battle map.\\
It the fleeing party stops to face their pursuers they will decide most of the positioning.

Note that it's easy to get into a position where the fleeing party is always faster in short bursts on the battle maps, but the pursuing party is always faster and more resilient travelling the regional map to pursue them. In that case the travel speed takes precedent and decides that the fleeing party cannot get away indefinitely. They may flee a couple of battle maps, but sooner or later they will be run into the ground. E.g: subtract 1d6 max stamina from the fleeing party members for every map they flee after the first one, until they can rest for a day.

A league on the regional map is a huge amount of battle maps, so the fleeing party probably doesn't get far before ending up on a new battle map, blood thirsty pursuers breathing down their neck. It's probably more interesting if the first battle escape means they are run down 1sq away on the region map, that way they may sometimes be able to force a battle in a different terrain than they fled from. But probably a good idea to limit this to once: 1sq from first escape, then the pursuers decide.

After having escaped from combat a character cannot immediately re-enter the map if he changes his mind. E.g: force a minimum amount of rounds away dependent on the character's speed and some random roll. Away 1d10 rounds +0 for maneuver, +3 for walk, +6 for run, +9 for dash.


\subsection*{Hunger and thirst}
%------------------------------
The character must eat and drink. Each day he is without food his max stamina drops by -1. Each day he is without water his max stam drops with -3. Note: when max stam goes below zero it's difficult to draw stamina for actions, and all actions take the negative stamina as mods, above ams, meaning it's going to get very difficult to do anything.
The character dies from thirst or starvation when max stam reaches -con.

When food and water is available again the character recovers max stam at con/3 per day.








%\phantomsection\addcontentsline{toc}{section}{ranged attacks}
%\section*{Ranged attacks}


\subsection*{Ranged weapons}
%---------------------------
Ranged weapons have the benefit of attacking distant targets and cannot generally be parried. But they have slow rate of fire and do not do as much damage as melee weapons.


\subsection*{Short and long range}
%---------------------------------
Each ranged weapon has a base range where the attack is mod=0 and do normal damage. Short range normally gives a small positive mod. Longer ranges have negative mods and damage reduction.

\

\small \begin{verbatim}
Short is base range / 2                  mod+1
Normal range is base range               mod-0
Long range is base range * 1.5           mod-3, dam-1
Very long range is base range * 2        mod-6, dam-2
Extreme range is base range * 3          mod-9, dam-3
\end{verbatim} \normalsize

\

\noindent Some weapons behave differently than these standard numbers.


\subsection*{Rate of fire}
%-------------------------
The faster the character fires the more difficult the attack will be. In some cases the fastest attacks require that the weapon or ammunition is readied as quickdraw items, and that they can be drawn in one or zero action


\subsection*{Too short range, base contact}
%------------------------------------------
Ranged weapons suffer mod-3 when used against a target in base contact with the attacker. There should be at least one empty square between the attacker and the target. The short mod+1 is also not in effect in base contact so the attack has a total mod-3.

Note that the attacker is not affected by base contact with other characters than the target, or if the target is in base contact with any other characters. If the target is engaged in melee however, then there is a penalty, see below under \emph{firing into melee}.


\subsection*{Firing into melee}
%-----------------------------
Firing into melee is tricky, mod-3, and you risk hitting a friendly in base contact with the target.

If you fail-3 or worse you hit a friendly melee participant in base contact with the target instead. The miss will never hit a "better" target for the attacker. A fail-1 or fail-2 is a normal miss without risk of hitting a friendly target.

This penalty does not apply if the target is engaged in melee against the firing character. In this case the \emph{too short range} penalty above will usually be in effect. Therefore it is preferable to attack ranged opponents with melee weapons without reach.

The maneuvers \emph{fire support} and \emph{target pointer} change this behaviour.

\

If you fail-3 to -5 you hit a friendly melee fighter in base contact with the target, in missile flight path if possible. The "alternate" target is probably the previous or next amongst the melee-ers in the missile flight path, in that order.

If you fail-6 or worse you might also hit melee-ers in base contact with target but perpendicular to missile flight path (i.e. to the side of the target, viewed from the missile). Which target gets hit is random, but must be friendly or it's a miss as usual.


\subsection*{Arrow recovery}
%---------------------------
To make life simple one can ignore missile ammo for most dungeon situations. A quiver of 30 arrows will be seen as enough for a normal dungeon with shopping access before and after. Special arrows, smaller quivers, large complexes or series of fights will require keeping track of ammunition. Assume half the arrows need to be replaced, or one arrow per 2r of combat.

The fights can get more interesting if it's necessary to keep track of ammunition count.
Arrows and bolts have a 50\% chance of being recoverable after hitting a "soft target", and 50\% of breaking. For a quick measure just assume you can recover half the arrows you shot at a target when you reach the corpse.
Missed arrows cannot be recovered unless they strike a soft surface, and most dungeons are made of stone...

The skill \emph{arrow recovery} is another fun option to keeping track of the survivability of arrows and which corpse is kindly holding them for you.


\subsection*{Fast moving targets}
%---------------------------------
It is more difficult to hit fast moving targets. Speed is the declared movement of the target, not the distance in squares it has moved in the round. This only applies to missile attacks. Melee attacks don't get the mods. The skill \emph{lead target} reduces these mods. \\
To hit mod is speed / 3

E.g: Fast Fabian dashes at 14 and is mod-4 to hit even if he has only moved 5 squares since the beginning of the turn.


\subsection*{Stationary targets}
%-------------------------------
\textbf{Melee} attacks against a target that is unaware of you, lying on the ground, sitting still, etc, and is not in combat mode and moving around will give mod+3. Stationary targets in combat mode will not give mods. I.e. targets that declare move 0 will still not be "still" enough since they are assumed to move around a bit in combat.

Prone targets are also considered as mod+3 stationary for melee attacks even if they are awake and on the way to rise and run away later in the round.

\

\noindent \textbf{Ranged} attacks against a totally stationary target gets a mod+1 but this does not apply to prone targets. Missile attacks against prone targets instead have mod-1.


\subsection*{Targets behind cover}
%---------------------------------
Attacks against targets that are behind significant cover is mod-3. E.g: half person cover, or an archer shooting from behind a rock. \\
Attacks against targets that are mostly behind cover is mod-6. E.g: someone peeking out behind a large boulder, or a crossbowman shooting from behind a battlement or through a crenel.\\
Peeking out from behind a corner could be mod-9, and so on.


\subsection*{Shields are in the way of ranged attacks}
%-----------------------------------------------------
Attacking someone standing behind a shield, i.e. the shield is in the way as seen along the flight path of the projectile, pushes a mod on the attack depending on how large the shield is and how well the person is using it.

Some very skilled people can sometimes attempt to parry incoming projectiles but it requires the maneuver \emph{missile parry} and very high skill.


\subsection*{Cover from shields, hiding behind shields}
%------------------------------------------------------
Most shields have a modifier against ranged attacks that is a form of cover, when attacked from the shield side. Simplified to $\pm$45\degrees from facing.

%Shield cover is the 90deg region centred on (+/-45deg) the facing of the character.
%Shield side is the 90deg region centred on (+/-45deg) the diagonal forward of the shield arm. I.e: left shield arm covers 0-90deg counter clockwise from the heroes facing, while right shield arm covers 0-90deg clockwise from the heroes facing.

Heroes hiding behind their shields can choose which 90\degrees arc they are protecting with the shield, announced when they start the action. Hiding behind a shield is a normal action but requires no roll and remains active indefinitely without spending further ap until the character does something else. Some actions can be taken while still hiding behind the shield, GM discretion.


\subsection*{Mounted combat}
%---------------------------
\todo go through the mounted combat issues

Ride rolls

ride limits skills

strength bonus application - speed of mount

cornering

ride fails, deviations, falling off

mounting / unmounting

large mounts - shifting position

continue movement when mounting / unmounting







\subsection*{No rules? Just figure something out.}
%-------------------------------------------------
There can never, and should not, be rules for everything. The tiny basics cover perhaps 90\% of regular play. The extended rules probably 95-98\%, mostly in details, edge cases, options, flavour. But there will always be more obscurities and combinatorial hmmm-what-if. Just figure out something that could work, and makes sense alongside the rest of the game, for your group.

E.g: Tossing and shooting a loaded crossbow.
Trigger Tom just fired his crossbow at the brutal orc running towards him. No way can he reload in time. Loaded Laban has his crossbow ready but no target. They get this bright idea: LL tosses his crossbow to TT so TT can fire again quickly.\\ \emph{What? No rule for rate of fire when tossing loaded crossbows?}

The toss-catch is almost a regular give \& receive relay action. And throwing a held weapon is 3ap attack action as well. So, approximate: throwing 3ap, and with sufficient quickdraw the give \& receive is 1r or 3ap (or faster). Bake it together: 1r total without quickdraw (limited by give \& receive), 3ap with quickdraw (limited by throw). Both LL and TT spend ap, and they can spend different. Lowest initiative, or if the faster can trigger the other.

They still have to pass throw rolls. Let's say it's a throw for LL and a throw or dex for TT to catch. And what it the crossbow discharges in the handling? What's the quality of the crossbow? How about range? Based on LL throw and strength?

\noindent Or just say: "1r, throw rolls".













%-------------------------------------------------------------------------------
%S P E L L S   A N D   M A G I C
%-------------------------------


\phantomsection\addcontentsline{toc}{section}{magic}
\section*{Magic}


Wave'em hands and mumble something mysterious. Sure to transmogrify your internal mana to sparkling effects of wonder, death, and mayhem, to affect or afflict friends and foe alike.

Sorcerers, wizards, mages, witches, and warlocks bend reality to their will by force of mind, concentration, intricate and arcane knowledge, calling upon ancient magical compacts and constructs that echo through the fundament of the world, and so on. Or perhaps they just stroke the ego of mad gods.


\subsection*{When and how spells can be cast}
%--------------------------------------------
Almost anyone can use a simple magical device, but to actually cast a spell without tools the Hero must have the skills "magic" and "spellcaster". Further skills can help.

Casting spells require concentration, voice, and gestures. Spellcasting often takes several rounds and the spell takes effect in the last round of casting following order of initiative.

The cast roll is subject to the highest action stack modification that the character has had during the casting period. \hyperref[multiroundactions]{See \emph{actions that take more than one round} above.}

%\TODO: rewrite all Xa spells to Xap instead.
Some spells have casting time counted in actions, 1a, 2a, etc, and should be treated just like normal actions at 3ap per action. Some spells have their casting time specified directly in action points. Some spells have 0a casting time and thus cost no action points. But even 0a/0ap spells take "actions" and follow initiative unless stated otherwise.

Spells that take one round or less to cast can be cast whenever in the round, following order of initiative, just like performing regular actions. That means they can also be cast as reactionary defence actions, like a parry, or retaliations after incoming attacks.
E.g: it's possible to summon a 1r ward flash defence to reduce the damage of an incoming attack, just like parrying with a regular shield.

A spellcaster can always choose to cast a spell that takes $\le$9ap in one full round instead, declaring only his base ap for the round and potentially skipping some heavy action mods.
%This does not apply when spellcasting have been shortened by \emph{fast magic}.
%\todo ? keep this ?


\subsection*{No power draw unless successful}
%--------------------------------------------
Casting spells that fail and fizzle causes no mana loss, unless it is a serious failure.\\
Fail-1,-2 costs no mana. \\
Fail-3 or worse is 1 mana lost. \\
Fail-6 or worse will cause the loss of the full cost of the spell.

This is to make spellcasters more enduring in game. It also fits well with the image of the wizard building the spell, then empowering it.


\subsection*{Extra mana for effects}
%-----------------------------------
Many spells have optional power effects, such as increasing damage, range, or duration by spending more mana. Mana must be spent for these effects individually.

E.g: Warlock Warner casts Shock Bolt, with normal cost 1m, but adds two mana for extra damage and one mana for extra range. The total cost is now four mana.
Just make sure you have enough levels in the magic skill to be able to spend that much mana in one go.


\subsection*{Range of spells}
%----------------------------
The range of the spell applies only to the instant when it's cast on the target, unless stated otherwise. Once the spell has gone into effect the range no longer matters.

There are spells which require the caster to remain within range for the duration, or which has another range during the duration, after casting. This kind of deviation is then stated in the spell description.


\subsection*{Duration of spells}
%-------------------------------
Spellcasting completes in the last round it is being cast, in order of initiative. The spellcaster can now choose when the spell should begin to take effect: \\
The spell's effect can start immediately after casting, in which case that round counts as the first round of it's duration. Or: \\
The effect can start at the very beginning of the next turn, ignoring initiative, and have that turn count as the first round of the duration.

This can be important. E.g: The wizard casts a spell which takes one round to cast and has one round of duration. If the duration starts and ends with the same round he casts it then it will perhaps benefit his friends more if he casts it before his friend's perform their actions. However, it he wants to take advantage of it himself he would better have the effect start with the next round since he probably cannot take any more actions in the round of the casting.

This option is available also for spells that take less than 1r to cast.


\subsection*{Keeping spells active}
%----------------------------------
% search: spell maintenance, maintain spells, maintaining spells
Most spells with duration must be kept active by force of will, concentration. Unless otherwise specified, each active spell requires one point of psy dedicated to keeping it from unravelling and dissipating per mana spent casting the spell. Those psy points are "in use" and deducted from the caster's psy while the spell is active.

Some spells with duration do not require any psy to keep active, while others can require several psy, or int, str, con, dex. Some burn stamina or mana, etc.
Deviations from the norm is noted as the maintenance requirement of skill.

Casting new spells while maintaining existing spells means the caster has reduced psy available and thus will take the appropriate mods from lack of psy.

Maintaining multiple active spells is possible as long as the character has enough psy, etc, to satisfy the maintenance requirements.


\subsection*{Regaining mana}
%---------------------------
All mana is regained when the caster has rested fully from a good night's sleep. Some dungeons allow for casters to regain some mana when resting fully a number of turns, or when spending time at a certain location.
Wise wizards and skilled sorcerers alike sometimes carry mana restoring potions, crystal ball batteries, power staffs, etc, when they go hunting for gold and glory.


\subsection*{Casting when dry}
%-----------------------------
Trying to cast with no mana left? Try rolling against psy. At each attempt; the character must roll against psy modified with the total cumulative negative mana level the caster is trying to draw down to. If he fails the mana is not drawn and the spell cannot be cast. If he fails with -3 or worse he passes out. To regain consciousness he must attempt to roll the psy modified by actual negative mana level each round until successful.

E.g: Crimson Caster is at mana -1, tries to cast a cost 3 spell and must roll psy-4 to be able to cast. He has psy=7 and rolls an 9, which means failure=-5. He falls unconscious. Following rounds he rolls psy-1 once per round until he wakes up again.


\subsection*{Casting is exhausting}
%-----------------------------------
Drawing mana costs stamina. Each point of mana drawn also draws one point of stamina. If the Hero can't draw stamina the spellcasting will fail.

If the Hero tries to draw stamina when starting from 0 or negative stamina he must roll con modified by negative stamina once for the whole spell. The Hero can roll con before starting casting the spell instead of when he finishes casting it if he so wishes. The roll is valid as long as he doesn't loose stamina while casting.


\subsection*{Smooth casting}
%---------------------------
The caster can spend more xp to train a spell so that it will not draw stamina together with mana. Learning to cast a spell smoothly takes 50\% more xp than the regular cost of learning the spell. Any xp spent learning the spell in the usual fashion can be counted towards learning to cast the spell smoothly.

E.g: Tired Ture knows how to cast unground. He has spent 11xp to train unground 6. But he's noticed he's often out of breath when he need to cast, so he wants to learn how to cast it smoothly. Since unground is scf 0.33 the smooth cast unground is scf 0.50. He can take the 11xp he's already spent on unground, but that only gets him to lvl 4, for 8xp. So he spends another 1xp to afford smooth cast unground to lvl 5, for 12xp. Tadaaa. No more burning stamina when ungrounding his enemies.


\subsection*{Lacking psy}
%------------------------
Most spells have a minimum psy requirement to cast. If the character does not meet them the casting of that skill will take a mod equal to the missing psy.
E.g: Wizard Woosey wants to cast a psy 7 spell, he has psy 5, thus all casting of that spell will be at mod-2.

Taking psy mods for various reasons will sometimes make the spellcaster severely handicapped.


\subsection*{Lacking int}
%------------------------
Most spells have a minimum int requirement. For every lacking int the spell becomes twice as expensive to learn. $\mathrm{Cost} = \mathrm{scf} \cdot \mathrm{lvl}^2 \cdot 2^{\mathrm{diff}}$. This is a lot, and it will often be cheaper to just raise the intelligence.


\subsection*{Sacrificing permanent psy}
%--------------------------------------
A caster can sacrifice a permanent psy to get 10 mana points (can increase the total mana past the character's maximum) and a mod+3 to all spells cast until the "sacrificial" mana points are gone. And for all the munchkins: yes, he casts using the "extra mana" first.

Sacrificed psy ends up as a permanent mod to the original psy stat. Keep track of it separately if there is any way in the campaign where they might be recovered. Divine intervention, or similar...

This is a very taxing action and can only be done once for each adventure. The sacrificial action takes a full round.


\subsection*{Hands and mouth free}
%---------------------------------
Casting magic is hindered if the hands are occupied or bound and if the wizard is gagged or has an enclosed helmet.

Each occupied hand gives mod-1. Both hands bound give mod-3. A gag gives mod-3.
E.g: a bound and gagged wizard has mod-6 to cast spells.

Heavier armour types also make casting more difficult.

Use "trusty old X", see below, to be able to cast while holding objects in hand.


\subsection*{The wizard's staff}
%-------------------------------
The wizard can choose to have a "trusty old staff" in his hands without incurring penalties for not having his hands free. Some wizard has a sword instead of a staff. Fidgety Fredrick had a tower shield he loved so much he cast better with it than without.

The "trusty old x" is a specific weapon or other large item, which the wizard has invested time and xp in. Thus it takes time and xp to replace if lost, or change. This way a wizard can cast most spells without penalties even if he is carrying weapons. Making an item a "trusty old x" with mod=0 costs 5xp.
In some cases the wizard can invest more time and xp to have the item act as a security blanket, giving a mod+1 when casting. Making an item a "trusty old x, mod+1" costs another +10xp.

A wizard can have more than one item at the same time, but the max bonus is +1. Just don't loose them, because then the xp are gone.

\

It can be fun to limit the creation or replacement speed of trusty old items. Let's say the wizard can't just cough up 15xp directly, but has to spend 5xp first, then go on an adventure with the staff, then the other 10xp for the mod+1.

Perhaps even force him to play one adventure without the "trusty old" status and have mod-1 before he can spend the first 5xp to mod=0.


\subsection*{The paladin's armour}
%---------------------------------
Similar to the wizards staff, the paladin must be able to cast his magic while dressed in combat armour. Heavy armour generally give spellcasting penalties. The paladin can spend xp to cancel the mods. Each mod point costs 5xp to cancel. The armour is then "divine armour", just like a "trusty old" something for the wizard. By spending the extra +10xp set the paladin can push his armour to mod+1, just like the wizard's staff.
Or the paladin can get the tank skill to reduce or cancel the armour penalties that way.
Don't forget to also pay "trusty old" xp to cancel the mods from sword and shield...

E.g: A paladin in full plate with tank 4 has a spellcasting mod from his armour at -2. So he spends 2*5=10xp to get it to mod=0 and another 10xp for the mod+1. Then he also have mods for his sword and shield, so he spends another 5xp for the sword and 5xp for the shield. He has now spent a total of 30xp on his equipment to have a mod+1 when casting spells.











%-------------------------------------------------------------------------------
% P L A Y I N G   T H E   M O N S T E R S
%----------------------------------------

\phantomsection\addcontentsline{toc}{section}{howto monster}
\section*{Playing the monsters}
%------------------------------
It is necessary to simplify the workload for the GM who has to manage all the monsters, villains, goons, etc which the Heroes encounter. Monsters are not generally long lived. Many usually die or flee within a dozen or two rounds of appearing. Most monsters can be significantly simplified and their handling made quicker by ignoring many effects and rules.


\subsection*{Some simplifications}
%----------------------------------
Ignore stamina, but don't let monsters use a lot of stamina all the time. They can use some, but not several stamina continuously round after round. Some monsters, like orcs, tend to have massive stamina and can gladly spend a lot, but keep it reasonable.

Ignore some pain, a lot of monsters don't react the same way to pain as heroes. A lot of monsters will instead influence their decision making by how much damage or pain they have taken. For single monsters or humanoid monsters it might be interesting to apply pain when appropriate.

Ignore mods from significant damage. If it makes life easier and there are lots of monsters, especially short lived ones.

Let the monsters die at hp=0, instead of having them linger around rolling for con each round.

Don't let monsters have too many advanced skills. It is better to keep the skills that add a lot of complexity to the more special monsters. The bosses, the leaders, the sneaky bastards, etc. Most should just have basic attacks and movements, perhaps a special maneuver or so.

These simplifications are great for "cannon fodder" critters, and that saves time enough so that you can play the bosses and interesting antagonists with more depth.


\subsection*{Full complexity for boss monsters}
%----------------------------------------------
It's more interesting to have special NPCs and opposition use full complexity, with all the intricate details, specialities, and loop holes. Simply adding things like counter attack, combat group, Oy!, leader, synchronise, etc drastically changes the way the opposition feels. Use it to give spice and make some mobs extra interesting.


\subsection*{Things to think about}
%-----------------------------------
Most monsters are not suicidal. They will run away when they meet overpowering enemies, if they can. Some monsters might run headlong into certain death but it is not the norm.

Intelligent monsters will instead flee, re-group, sound the alarm, make ambushes and traps instead of direct open battle.














%-------------------------------------------------------------------------------
% M I S C E L L A N E O U S   G M   T I P S
%------------------------------------------

\phantomsection\addcontentsline{toc}{section}{misc tips}
\section*{Miscellaneous GM tips}
%-------------------------------

As you get used to the game and try different things you'll figure out what works well for you and your group, and what doesn't. Don't be afraid to experiment. Log what you're doing so you don't forget, a few years down the line it's all very fuzzy in the old noggin.


\subsection*{fun --- long term vs short term}
%--------------------------------------------
\todo avoid short term fun at the expense of long term fun


\subsection*{Nerfing}
%--------------------
Nerfing rules, skills, abilities, and equipment is no problem unless they are in play on a character. My suggestion is to just allow that character to keep the original version, but force all future acquisitions to use the new version. Player characters will have a reasonable turn over rate anyway, and it's a bit fun to have \emph{legacy} ancient weird stuff in play sometimes.

During campaign play I strive to not forcibly nerf, and always allow the character to re-trade his XP and gold for something else if I \emph{really} have to nerf something already in play. Again, consider the long term \emph{player enjoyment} vs the combat balance. Perhaps there are other ways to contextually limit something that has turned out to be overpowered, too complex, or broken.


\subsection*{Hero manual}
%------------------------
I recommend that the players copy the skills they choose for their character directly to a cheat sheet. It is important to note down what the original version of the skill actually said when the player decided to train it. Especially if you or your group is tweaking and fiddling with the skills and rules a lot during the game.

This also means that the players will effectively compile a small rule book specifically for their own character. The basic rules, notes, ideas, etc. Very useful. It offloads the GM since the player will have a good understanding of the rules affecting his character.


\subsection*{Finding stuff}
%--------------------------
Either roll find/per rolls hidden when needed so that the players won't notice, or ask the players to roll them, and then ask for rolls every now and then even when there is nothing to find. If asking for rolls you probably need to have at least a 2:1 ratio of asked vs real roll situations. This will keep the players anxious on a roll request, but not sure that there will be something to find.

I prefer asking for rolls in most situations, then roll hidden every now and then anyway.


\subsection*{Role Playing vs Tactical Optimisation}
%--------------------------------------------------
Generally: role playing limits tactically optimal planning and behaviour. Make sure you've discussed this with your group so you all know what level of rp vs tactics you're all happy with. This also involves what level of metagaming and information leakage you have across the table, battle map, or voice chat.

\

\noindent \emph{Know your players. Discuss in advance. Find common ground.}

\

\noindent
In the most severe option on one side you can use individual fog of war and vision for each Hero, and insist that all players share information of the battle map \emph{in character} and time what they say into actions and rounds for their Hero. Remember that it's possible to shout and murder at the same time. This is definitely doable with some tool setups and it can be a whole lot of fun, but in my experience not what most people prefer or have the most fun with.

The other extreme is the shared battle map and open collaborative reasoning between all players without time limits for every round and individual action. This can be a lot of fun and more social, but some players will zone out and participate less. It sometimes lead to situations where one or a small clique of players mastermind the whole party, essentially just giving suggestions or orders across the table or chat, and the rest of the players just executing.

Most groups quickly find common ground naturally, but be aware of the issue and discuss.


%\subsection*{Have the players narrate assisted actions}
%%------------------------------------------------------
%Fun if the players argue why and how they use their skills can roll for assist


\subsection*{Larger maps}
%------------------------
Gold and Glory tactical gameplay requires larger maps than normal because there is often a lot more movement and positioning going on during the fights than in most games. Most battle maps I build are at least 30x30sq, sometimes larger than 100x100sq.

Be careful when using smaller pre-made battle maps. It is best if they have firm walls or enclosure visible around the battle area, otherwise they have a tendency to feel artificially cramped when the players want to move around. Or simply expand the map by sketching surroundings, cut'n'paste, or similar. It won't be as pretty, but allows more movement.

I often add entry/exit boxes or borders. Cross those boundaries and the Hero has left the battle and cannot return.










%-------------------------------------------------------------------------------
% Y O U N G   K I D S
%--------------------

\phantomsection\addcontentsline{toc}{section}{young kids}
\section*{Playing with young kids}
%---------------------------------
\label{sec:youngkids}  % Bitter Candy
We've successfully played with kids as young as five years old, without any previous tabletop or role playing experience. It's enough that they can count to 10, add and subtract a bit. So, clever kids, but still.

\

\noindent \emph{Keep it simple. Keep it fun.}

\

\noindent
Use physical objects and markers instead of numbers on a paper. Improvise the battle map with everyday objects and use their toys as Hero minis. Any toy up to the size of a hand works great.


\subsection*{Bitter Candy}

I had a tiny adventure when I playtested with a trio of young girls. I had them build a small play farm on a coffee table, and choose one small toy each. Then told them that an evil bitter taste was spreading throughout the land, making all the candy taste like coffee. They must find the source of the horror and put a stop to it immediately.

Sparkly Pony and two old Rubber Trolls had to venture from their coffee table farmstead, across the Living Room Expanse and the Valley of Hallway, to reach the Forbidden Kitchen. There they had to climb the Chair Mountains up to the Oakwood Plateau, whence the miasma originated. Finally, on the high flatland, they faced Lord Grapefruit and all his purple and green grape minions. Along the way they had encountered a friendly sock monster who had taught them a useful spell of Stink Cloud, and fought a couple of evil car demons to get a set of SpeedWheel boots. They also had to solve the riddle of BrassHandle before the DoorMaster allowed them to enter the Forbidden Kitchen.

Bonus that they got to eat all the grapes they defeated, though they were kind enough to leave the carcass of Lord Grapefruit for their mother.

\

We used strips of paper for movement speed and a ruler for bow range. Smartphone for dice since the household were not gamers and didn't have anything except some old tiny yahtzee d6 which were too small and boring.
Simplified rule, skill, and equipment sets similar to the \hyperref[sec:basicenough]{\texttt{Ominous Crown}} adventure example, page~\pageref{sec:basicenough}.

I drew small paper cards of the weapons and equipment, with corresponding skill values written in the corner. We used coins as markers for hit points, small balls of green and red paper for action points and difficulty modification points.








%-------------------------------------------------------------------------------
% T A B L E T O P
%----------------

\phantomsection\addcontentsline{toc}{section}{tabletop}
\section*{TableTop}
%------------------
Not originally designed for actual physical tabletop gaming, we've run both simple and fairly complex versions of the game on physical tabletop, especially when playing with young children.
A few sheets of paper, pencil, eraser, and a small objects are all that's needed. Though much more impressive with cool terrain, props, and minis.


\subsection*{Markers and Minis}
%------------------------------
With markers it's quick and easy to keep track of the character's current stats. Suggest using small markers like coins, plastic bits, paper balls, whatever of various colour, size, shape to represent: mods, mp, ap, hp, stamina, mana, pain, stun, base mods, duration counters, etc.

Keep three separate areas:\\
per round accounting: mods, mp, ap\\
persistent status: hp, stamina, mana, pain, stun, etc\\
a marker bank, separate per character to have it close at hand.

Each round the players will fiddle with mp, ap, and mods, so that should be quick and clear without risk of touching and disrupting the heaps of hp, stamina, mana, etc. So, \emph{keep them separate}.

I suggest front loading hp, stamina, mana, so that the pools are filled with "nice markers" when they are whole and rested, and getting smaller and smaller as time goes by, while the piles of "nasty markers" like pain, grow. That way it's easy to see when someone is getting into risky situations: His nice heaps are getting uncomfortably small.


\subsection*{Simplified tabletop light and vision}
%-------------------------------------------------
In daylight people have their normal vision ranges and usually just see whatever is on the board. For large boards it might be relevant to measure approximate vision distances before placing minis, and using "blips" on the board to represent stuff in the distance that has not been clearly resolved. E.g: this blip is a group of a half dozen or so small humanoids, that one looks like a large monster, etc.

In low light conditions it's easier to approximate light and vision range based on light sources and the Heroes' low light vision. E.g: a candle lights a small room, a torch lights a large room. Use blips for things lurking in the dark.








%-------------------------------------------------------------------------------
% O N L I N E   G A M I N G
%--------------------------

\phantomsection\addcontentsline{toc}{section}{online}
\section*{Online gaming}
%-----------------------
Face to face, coffee and cake, friends and banter. Gathering around a table is more fun. But for many it's a rare treat. Life and logistics gets in the way. Children and work take priority and when your gaming friends all of a sudden live a continent away the Sunday game commute is not environmentally defensible any more.

For us, online play, once a week on a fixed weekday evening 2000--2200 has been shown to work well enough to actually be schedulable.


\subsection*{Tools: virtual tabletop and voice chat}
%---------------------------------------------------
The game is primarily designed for short sessions with fast gaming, and we play mainly online over voice and vtt. As of late 2019 we're still using maptool and mumble. They work well, even for the sometimes enormous battles of the Goblin Destiny campaign. I'm sure other VTTs work fine but maptool is the only one I've found that is self hosted, has vision blocking, fog of war, and individual token vision. It's also quick and easy to work with as a GM.

For voice chat I recommend mumble + murmur and make sure everyone has a decent headset. \vvsmall(early 2020)\normalsize.~ Don't use webex, skype, teams, etc as those are way too unstable and have horrible sound quality, lag, and jitter. Hangouts and Discord are better than skype but have much longer latency than mumble. Teamspeak and ventrilo were ok, haven't tested for a while, but mumble beats them as well. And since mumble can be self hosted you can have exceptionally good latency and jitter, especially between the GM and every single player.

\

Since most of my projects have longevity exceeding a decade I've come to rely primarily on open source tools. Proprietary stuff simply doesn't have the survival statistics. Economy, product strategy, and acquisitions kill tools too frequently for me to invest time in them.

% I'd like to find one that also works well on tablets and smartphones.

% some
% VTT: maptool, vassal, roll20, fantasy grounds, foundry, ...
% ? table top simulator on steam?





%-------------------------------------------------------------------------------
% M A P T O O L
%--------------

\phantomsection\addcontentsline{toc}{section}{maptool}
\section*{Maptool}
%-----------------
Maptool is the only Virtual TableTop system I've used for this game since the inception back in 2008. First prototype experiments we just used jarnal and doodled our way through a couple of mini test encounters.

Open source. Self hosted. You as GM or player have control and access to all the work you create, it's not locked away on a server somewhere, dependent on whether the hosting company survives / changes strategy / gets acquired in the coming years. The code is mainly java so it's fixable if the external community dies. I saw the likelihood of this little hack'n'slash game being long lived, so open source and self hosting were \textit{must have} when I scouted for tools.

Maptool had rudimentary support for fog of war and vision / vision blocking early on, and by now it's pretty good. This adds such a great feeling of exploration and immersion when playing, and lots of tactical depth. The experience of entering a dark dungeon, lighting a torch, and carefully sneaking around corners is wonderful, even in a simplified top down view. Just as having enemies hide in the long shadows cast by trees and bushes in the light from your camp fire. Great, and something most other tools were lacking. And many still lack today. \vvsmall(early 2020)\normalsize

Maptool scripting is passable for very simple stuff but horrible to work with. Despite this, I've implemented automation for most infrastructure in the game over the years. Almost all actions as GM is one or two clicks by now, and very little to keep in the noggin between rounds.

It's also reasonably quick to import maps and set vision blocking, as well as make improvised maps. Though not quick enough that I'm happy to do it live during a session.


\subsection*{Maptool light and vision}
%-------------------------------------
Set vision range and arc for each character. I've also built standard versions for the various typical npcs.

Vision / light condition multipliers should be set as follows:
\small\begin{verbatim}
normal    all light sources have the specified range: humans, halflings
dusk      1.5x range of light sources: dwarfs, orcs, goblins
night     2.0x range of light sources: elves
dark      3.0x range of light sources:
black     self light up to infra range regardless of light sources
\end{verbatim}\normalsize

This means an elf standing close to a camp fire has full daylight conditions, and a human with a torch is at most 1r dash away from a monster when he sees it. Hehehe.

Magical dark vision should have different multipliers, or set a personal self light to suitable radius.

Dwarven infra vision, goblin smellyvision, etc, can be set with a self light radius of a few squares.







