%===============================================================================

\documentclass[11pt, twoside, titlepage, a4paper]{report}

%% for xelatex and fontspec for searchable ligatures in pdf:
%\usepackage{fontspec}        % searchable ligatures in pdf
%% utf8 inputenc is ignored for xelatex utf8 based default

% for pdflatex, faster older, ligatures break search in pdf:   << fix with cmap
\usepackage{mmap}    % add letter sequences to pdf info for searchable ligatures
\usepackage[T1]{fontenc}    % ligatures break search in pdf    << fix with cmap
% set utf8 encoding, and set font encoding T1 to allow "|" ">" "<" etc
\usepackage[utf8]{inputenc}

\usepackage[a4paper,inner=40mm,outer=25mm,top=25mm,bottom=25mm,pdftex]{geometry}
% These page settings give images 1.0\linewidth around 135-140mm wide (ca 138mm)
% meaning a 300dpi image is around 1600 pixels wide
\usepackage{graphicx}   % For eps figures
\usepackage{epsfig}     % Alternative package
\usepackage[hang,small,bf]{caption}

\usepackage[british]{babel}       

\usepackage[yyyymmdd]{datetime}
\renewcommand{\dateseparator}{--}

\usepackage{fancyhdr}
\pagestyle{fancy}
% with this we ensure that the chapter and section
% headings are in lowercase.
\renewcommand{\chaptermark}[1]{\markboth{#1}{}}
\renewcommand{\sectionmark}[1]{\markright{\thesection\ #1}}
\fancyhf{} % delete current setting for header and footer
\fancyhead[LE,RO]{\bfseries\thepage}
\fancyhead[LO]{\bfseries\rightmark}
\fancyhead[RE]{\bfseries\leftmark}
\renewcommand{\headrulewidth}{0.5pt}
\renewcommand{\footrulewidth}{0pt}
\addtolength{\headheight}{0.5pt} % make space for the rule
\fancypagestyle{plain}{%
    \fancyhead{} % get rid of headers on plain pages
    \renewcommand{\headrulewidth}{0pt} % and the line
}


% remove forced implicit vertical whitespace before and after verbatim environment
\makeatletter
\preto{\@verbatim}{\topsep=0pt \partopsep=0pt }
\makeatother


% allow to force indentation of first line in section
% \indent is not working, so workaround \hspace{\parindent} works
\newcommand{\forceindent}{\hspace{\parindent}}


\newcommand{\degrees}{$^\circ$~}
\newcommand{\degree}{$^\circ$}
\newcommand{\ca}{$\approx$}

\newcommand{\vs}{$\backslash\ $}  % "versus" slash
\newcommand{\bs}{$\backslash\ $}  % just backslash


% want clear dash insert commands
\newcommand{\dash}{-}     % just a normal hyphen dash  "-"
\newcommand{\ndash}{--}   % n-dash "--"
\newcommand{\mdash}{---}  % m-dash "---"


%link new command names to the original font sizes,
%for easier to remember smaller font size
\newcommand{\vsmall}{\footnotesize}  % simpler to remember
\newcommand{\vvsmall}{\scriptsize}   %
%\newcommand{\vvvsmall}{\tiny}


\usepackage[colorlinks=true,linkcolor=black,urlcolor=blue]{hyperref}


\usepackage{ifthen}


% \needspace{5\baselineskip}      << reserves approximately 5 lines, leaves raggedbottom, more efficient
% \Needspace{5\baselineskip}      << reserves exactly 5 lines, leaves raggedbottom, less efficient
% \Needpsace*{5\baselineskip}     << leaves flushbottom if \flushbottom is in effect, otherwise ragged
\usepackage{needspace}



% \skill{blabla}
\newboolean{skillsaslist}
\setboolean{skillsaslist}{true}
\ifthenelse{\boolean{skillsaslist}}{\newcommand{\skill}[1]{\item[#1]}}{\newcommand{\skill}[1]{\subsubsection*{#1}}}
\ifthenelse{\boolean{skillsaslist}}{\newcommand{\openskillslist}{\begin{description}}}{\newcommand{\openskillslist}{}}
\ifthenelse{\boolean{skillsaslist}}{\newcommand{\closeskillslist}{\end{description}}}{\newcommand{\closeskillslist}{}}

% \action{blabla}
\newboolean{actionsaslist}
\setboolean{actionsaslist}{true}
\ifthenelse{\boolean{actionsaslist}}{\newcommand{\action}[1]{\item[#1]}}{\newcommand{\action}[1]{\subsubsection*{#1}}}
\ifthenelse{\boolean{actionsaslist}}{\newcommand{\openactionslist}{\begin{description}}}{\newcommand{\openactionslist}{}}
\ifthenelse{\boolean{actionsaslist}}{\newcommand{\closeactionslist}{\end{description}}}{\newcommand{\closeactionslist}{}}

% \eqitem{blabla}
\newboolean{itemsaslist}
\setboolean{itemsaslist}{true}
\ifthenelse{\boolean{itemsaslist}}{\newcommand{\eqitem}[1]{\item[#1]}}{\newcommand{\eqitem}[1]{\subsubsection*{#1}}}
\ifthenelse{\boolean{itemsaslist}}{\newcommand{\openitemslist}{\begin{description}}}{\newcommand{\openactionslist}{}}
\ifthenelse{\boolean{itemsaslist}}{\newcommand{\closeitemslist}{\end{description}}}{\newcommand{\closeactionslist}{}}


\newenvironment{readoutloud}%
{\begin{quote}\begin{itshape}}%
{\end{itshape}\end{quote}}%



% need a nice easily visible TODO marker
\newcommand{\todo}{\textbf{TODO:}~}
\newcommand{\TODO}{\LARGE\textbf{TODO:}\normalsize~}



%
\documentclass[11pt, twoside, titlepage, a4paper]{article}
% set utf8 encoding, and set font encoding T1 to allow "|" ">" "<" etc
\usepackage[utf8]{inputenc}
\usepackage[T1]{fontenc}
\usepackage[a4paper,inner=40mm,outer=25mm,top=25mm,bottom=25mm,pdftex]{geometry}
% These page settings give images 1.0\linewidth around 135-140mm wide (ca 138mm)
% meaning a 300dpi image is around 1600 pixels wide
\usepackage{graphicx}   % For eps figures
\usepackage{epsfig}     % Alternative package
\usepackage[hang,small,bf]{caption}

\usepackage[british]{babel}       

\usepackage[yyyymmdd]{datetime}
\renewcommand{\dateseparator}{--}

\usepackage{fancyhdr}
\pagestyle{fancy}
% with this we ensure that the chapter and section
% headings are in lowercase.
%\renewcommand{\chaptermark}[1]{\markboth{#1}{}}  % no "\chapter" in article doc type
\renewcommand{\sectionmark}[1]{\markright{\thesection\ #1}}
\fancyhf{} % delete current setting for header and footer
\fancyhead[LE,RO]{\bfseries\thepage}
\fancyhead[LO]{\bfseries\rightmark}
\fancyhead[RE]{\bfseries\leftmark}
\renewcommand{\headrulewidth}{0.5pt}
\renewcommand{\footrulewidth}{0pt}
\addtolength{\headheight}{0.5pt} % make space for the rule
\fancypagestyle{plain}{%
    \fancyhead{} % get rid of headers on plain pages
    \renewcommand{\headrulewidth}{0pt} % and the line
}


% remove forced implicit vertical whitespace before and after verbatim environment
\makeatletter
\preto{\@verbatim}{\topsep=0pt \partopsep=0pt }
\makeatother


% allow to force indentation of first line in section
% \indent is not working, so workaround \hspace{\parindent} works
\newcommand{\forceindent}{\hspace{\parindent}}


\newcommand{\degrees}{$^\circ$~}
\newcommand{\degree}{$^\circ$}
\newcommand{\ca}{$\approx$}

\newcommand{\vs}{$\backslash\ $}  % "versus" slash
\newcommand{\bs}{$\backslash\ $}  % just backslash


% want clear dash insert commands
\newcommand{\dash}{-}     % just a normal hyphen dash  "-"
\newcommand{\ndash}{--}   % n-dash "--"
\newcommand{\mdash}{---}  % m-dash "---"


%link new command names to the original font sizes,
%for easier to remember smaller font size
\newcommand{\vsmall}{\footnotesize}  % simpler to remember
\newcommand{\vvsmall}{\scriptsize}   %
%\newcommand{\vvvsmall}{\tiny}


\usepackage[colorlinks=true,linkcolor=black,urlcolor=blue]{hyperref}


\usepackage{ifthen}


% \needspace{5\baselineskip}      << reserves approximately 5 lines, leaves raggedbottom, more efficient
% \Needspace{5\baselineskip}      << reserves exactly 5 lines, leaves raggedbottom, less efficient
% \Needpsace*{5\baselineskip}     << leaves flushbottom if \flushbottom is in effect, otherwise ragged
\usepackage{needspace}



% \skill{blabla}
\newboolean{skillsaslist}
\setboolean{skillsaslist}{true}
\ifthenelse{\boolean{skillsaslist}}{\newcommand{\skill}[1]{\item[#1]}}{\newcommand{\skill}[1]{\subsubsection*{#1}}}
\ifthenelse{\boolean{skillsaslist}}{\newcommand{\openskillslist}{\begin{description}}}{\newcommand{\openskillslist}{}}
\ifthenelse{\boolean{skillsaslist}}{\newcommand{\closeskillslist}{\end{description}}}{\newcommand{\closeskillslist}{}}

% \action{blabla}
\newboolean{actionsaslist}
\setboolean{actionsaslist}{true}
\ifthenelse{\boolean{actionsaslist}}{\newcommand{\action}[1]{\item[#1]}}{\newcommand{\action}[1]{\subsubsection*{#1}}}
\ifthenelse{\boolean{actionsaslist}}{\newcommand{\openactionslist}{\begin{description}}}{\newcommand{\openactionslist}{}}
\ifthenelse{\boolean{actionsaslist}}{\newcommand{\closeactionslist}{\end{description}}}{\newcommand{\closeactionslist}{}}

% \eqitem{blabla}
\newboolean{itemsaslist}
\setboolean{itemsaslist}{true}
\ifthenelse{\boolean{itemsaslist}}{\newcommand{\eqitem}[1]{\item[#1]}}{\newcommand{\eqitem}[1]{\subsubsection*{#1}}}
\ifthenelse{\boolean{itemsaslist}}{\newcommand{\openitemslist}{\begin{description}}}{\newcommand{\openactionslist}{}}
\ifthenelse{\boolean{itemsaslist}}{\newcommand{\closeitemslist}{\end{description}}}{\newcommand{\closeactionslist}{}}


\newenvironment{readoutloud}%
{\begin{quote}\begin{itshape}}%
{\end{itshape}\end{quote}}%



% need a nice easily visible TODO marker
\newcommand{\todo}{\textbf{TODO:}~}
\newcommand{\TODO}{\LARGE\textbf{TODO:}\normalsize~}





\begin{document}
%-------------------------------------------------------------------------------


\section*{Dungeon of Testing}

This tiny dungeon adventure is meant to be the first introduction to the game that a new character (and player) has before taking on the real dangers out there. It is meant to be a solo experience, with only the one player and character, so as to get a good one on one introduction experience.

The monsters in the Dungeon of Testing will not kill a character, which is nice for a first experience. These are test monsters, not like \emph{real} monsters. The peculiar magic of self reflection and re-evaluation that accompanies the Dungeon of Testing is also very good for characters that want to change how they have spent their XP growing up, once they have found some glaring holes in their survival capacity.

A Dungeon of Testing is something most serious towns have, on franchise from Ottokar's Test Dungeons. It will scare some young fools into staying in the village to become farmers or blacksmiths instead of travelling the world as dragon slayers or explore the limited career options of compost. \emph{Compost, i.e. post-adventuring occupation.}

The idea of the Dungeon of Testing is to be slightly too difficult for most new characters so that the player can re-evaluate how the XP are spent before the character goes into real combat in adventures and campaigns.




\subsection*{Introduction}

The Dungeon of Testing is small and with only a few critters. It takes about an hour to play through. It is a bit too hard for most newbie 100xp adventurers, but that is the idea.

\begin{readoutloud}
Welcome young [insert name here]. This is the Dungeon of Testing, designed by Ottokar's Test Dungeons. The monsters here will not really kill you, not for real, unless by accident. They are also very well trained to feign death once their hp reaches 0, so please don't overkill them just to be sure. They will not stand up and attack you in the back once they have accepted "death".

When you have made it out alive, or been carried out as a near dead loser, you will be given the magical ability to change how you have spent your youthful XP growing up.

There is a very weak lantern you may borrow if you so wish, but we do advise that you purchase some decent spelunking equipment before setting out on your adventures: A light source, fire tools, rope, grappling hook, water skin, rations, loot bags, etc, will always come in handy sooner or later.

Any coins you find you will be allowed to keep, as a starting grant for your newly chosen career path. Weaponry and items you may find have to be returned if you make it out alive.

So, off you go. Good Luck and all that.
\end{readoutloud}





\subsection*{Dungeon mechanics}

Start the Hero by the Heroes' Standard. He may then choose to pick up the weak lantern if he wish. Assign ownership and let the player move the lantern with his character, keep them stacked on the same square, or he can hold it out to the next square if he so wish.

This is an exceptionally simple dungeon. There are no traps or other problems around, other than the monsters.

The weak lantern can be picked up and carried around, it takes one hand of course.

The brazier can be moved, 1sq/r, and the regular strength lanterns can be picked up from the wall hangers and carried.

A lantern can be placed on the ground to light an area. These training monsters will not destroy the lanterns to gain advantage by darkness. Note that this is a dangerous thing to do under \emph{real} adventuring conditions.




\subsection*{The opposition}

The opposition is meant to be a little too much for most newbie 100xp adventurers, so that the experience from the dungeon will allow the player to change how he has spent the XP of the character.

If you believe the monsters are too easy, then add another goblin fighter or two with clubs or bows, or change one or two fighter into warriors. In drastic cases add another orc. Extra monsters are found above the cave map.

More scholarly characters or support characters will usually be outclassed by a large degree. Then remove the goblin archer and/or the goblin warrior, and perhaps one of the goblin fighters with clubs. But keep the orc. That is useful for scaring the newbie out of the dungeon.

The first enemy is the goblin runt knifer. He is almost asleep and has a mod-3 to detect any intruder. Make sure to modify this accordingly if the Hero is carrying the lantern with him, or is back-lit by the lantern when he enters. The runt's job is to sound the alarm if he sees anyone come in. He will snap to immediately when he notices someone, but will not sound the alarm until the next round. His first free action in the next round will be to scream, which will slowly notify the goblins, but not wake the orc.
The goblin runt will try to flee instead of fight. He will first flee towards the goblins, but if they start falling fast he will try to flee towards the orc.


The goblins will react to the runt scream by grabbing their weapons and start walking down to check out the ruckus. If the alarm has not sounded they will be talking amongst themselves, playing dice, and have mod-3 to detect an intruder. They will put up a decent fight when engaged, but will flee once they get too hurt or have lost half their people. Both they and the runt will try to flee towards the orc if possible at this point.

Goblins are usually quite cautious and cowardly when fighting. They will try to take advantage of their numbers and surround the aspiring adventurer.


The orc is asleep, and has a 20\% chance of waking each round if there is fighting going on in the main room. He will then take one round to wake up, another to get his club, then start walking to join the fray. If he hears death screams from the goblins he will start to run. Any fleeing goblins will wake him and he will run to attack the intruders. The orc is a bold fighter and will not start parrying attacks until he is at or below half hp. He will instead usually push the advantage, charge if possible, and continue attacking as response to incoming attacks.

When the orc has engaged the Hero the goblins left standing will be sure of victory once again and return to fight offensively.

The stat sheets for the orcs and goblins are last in this document.




\subsection*{The Loot}

The original Dungeon of Testing is worth 10xp in monsters. A character who makes it out alive gets 5xp as a dungeon bonus if he either: \\
a) completes the dungeon, or \\
b) has lost at least 50\% of his hp at some point. \\
If he flees before loosing enough hp he is too much of a chicken shit and gets no bonus. If he dies he gets no bonus, but can keep the monster xp he has killed.

The goblins' hoard has 5+1d5 silver and 1d10 copper coins.
Each goblin fighter has 1d10 copper, and the orc has 1 silver an 2d10 copper.

\

\begin{samepage} \small \begin{verbatim}
XP values for the dungeon:
    make it out alive: 5xp (only if explored the dungeon or lost over 50% hp)
    explore all rooms: 1xp each
XP values for the critters:
    goblin fighter    1
    goblin warrior    2
    goblin runt       1 (should be 0.5 but hey, there is only one)
    orc fighter       3
\end{verbatim} \normalsize \end{samepage}



\subsection*{The test sessions}

The Dungeon of Testing has been played by every new player so far as an introduction to the game. Two characters have made it through with both loot and xp, most flee with barely an hp left to their name, then change a bit in their skill lists.

It usually takes about an hour to run through the dungeon, while explaining the rules and having the player get used to MapTool. 

\





\vfill
%-------------------------------------------------------------------------------
% appendix monsters
% -----------------

\raggedbottom


\subsection*{The Monsters}

Here are the monsters that populate the Dungeon of Testing, for convenience. They are more or less the same as their name sake in the monster list in the core book.

\

\goodbreak \begin{samepage} \small \begin{verbatim}
===================================
goblin runt knifer   (goblinrunt02)   set small token size
-----------------------------------
str 1     hp 3 abs 0
dex 5     m2 w3 r4 d6
int 1     vision 10 dusk
psy 1     ap 5
per 5     initiative 7
----------
knife     4 dam 3 abs 4
claw      7 dam 1, fast+1, double attack mod-3 3ap
bite      5 dam 2
avoid     6 yield+3
sneak     8
===================================
\end{verbatim} \normalsize \end{samepage}

\

\goodbreak \begin{samepage} \small \begin{verbatim}
===================================
goblin fighter           (goblin02,05)   set small token size
-----------------------------------
str  3    hp 5 abs 0
dex  6    m1 w3 r5 d7
int  2    vision 10 dusk
psy  3    ap 4
per  5    initiative 8
----------
sneak           5
charge          4
avoid           5 yield+4
-- carries either club or bow, not both --
small club      7 dam 4, abs 8
short bow       6 dam 4, range 12, short 6 mod+3, long 18 mod-3,
                rate of fire: quick 1r mod-3, normal 2r mod-0, aimed 3r mod+3
===================================
\end{verbatim} \normalsize \end{samepage}

\

\goodbreak \begin{samepage} \small \begin{verbatim}
===================================
goblin warrior           (goblin07)   set small token size
-----------------------------------
str  4    hp 6 abs 1
dex  7    m1 w3 r5 d7
int  3    vision 12 dusk
psy  3    ap 4
per  6    initiative 9
----------
avoid           6 yield+4
sneak           4
charge          6
veteran         1
mobile          1
light axe       7 dam 5, abs 6, parry-3, toparry-1, toavoid+1
small shield    8(6) abs 8, parry+2, ranged-1/-2
leather armour abs 1
===================================
\end{verbatim} \normalsize \end{samepage}

\

\goodbreak \begin{samepage} \small \begin{verbatim}
===================================
novice orc fighter       (orc01,06)
-----------------------------------
str  9    hp 16 abs 1
dex  6    m2 w4 r6 d8
int  4    vision 12 dusk
psy  4    ap 4
per  5    initiative 7
----------
large 2h club   5 dam 9, abs 18, parry-1, toavoid+1
bite           	5 dam 4
punch          	7 dam 3 fast+1 (2ap)
kick            6 dam 5 
avoid          	4 yield+3
charge         	5
tackle         	4
veteran        	3
leather armour abs 1
===================================
\end{verbatim} \normalsize \end{samepage}








%-------------------------------------------------------------------------------
\end{document}

