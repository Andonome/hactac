%--------|---------|---------|---------|---------|---------|---------|---------|
%       10        20        30        40        50        60        70        80
%-------------------------------------------------------------------------------

\documentclass[11pt, twoside, titlepage, a4paper]{article}
% set utf8 encoding, and set font encoding T1 to allow "|" ">" "<" etc
\usepackage[utf8]{inputenc}
\usepackage[T1]{fontenc}
\usepackage[a4paper,inner=40mm,outer=25mm,top=25mm,bottom=25mm,pdftex]{geometry}
% These page settings give images 1.0\linewidth around 135-140mm wide (ca 138mm)
% meaning a 300dpi image is around 1600 pixels wide
\usepackage{graphicx}   % For eps figures
\usepackage{epsfig}     % Alternative package
\usepackage[hang,small,bf]{caption}

\usepackage[british]{babel}       

\usepackage[yyyymmdd]{datetime}
\renewcommand{\dateseparator}{--}

\usepackage{fancyhdr}
\pagestyle{fancy}
% with this we ensure that the chapter and section
% headings are in lowercase.
%\renewcommand{\chaptermark}[1]{\markboth{#1}{}}  % no "\chapter" in article doc type
\renewcommand{\sectionmark}[1]{\markright{\thesection\ #1}}
\fancyhf{} % delete current setting for header and footer
\fancyhead[LE,RO]{\bfseries\thepage}
\fancyhead[LO]{\bfseries\rightmark}
\fancyhead[RE]{\bfseries\leftmark}
\renewcommand{\headrulewidth}{0.5pt}
\renewcommand{\footrulewidth}{0pt}
\addtolength{\headheight}{0.5pt} % make space for the rule
\fancypagestyle{plain}{%
    \fancyhead{} % get rid of headers on plain pages
    \renewcommand{\headrulewidth}{0pt} % and the line
}


% remove forced implicit vertical whitespace before and after verbatim environment
\makeatletter
\preto{\@verbatim}{\topsep=0pt \partopsep=0pt }
\makeatother


% allow to force indentation of first line in section
% \indent is not working, so workaround \hspace{\parindent} works
\newcommand{\forceindent}{\hspace{\parindent}}


\newcommand{\degrees}{$^\circ$~}
\newcommand{\degree}{$^\circ$}
\newcommand{\ca}{$\approx$}

\newcommand{\vs}{$\backslash\ $}  % "versus" slash
\newcommand{\bs}{$\backslash\ $}  % just backslash


% want clear dash insert commands
\newcommand{\dash}{-}     % just a normal hyphen dash  "-"
\newcommand{\ndash}{--}   % n-dash "--"
\newcommand{\mdash}{---}  % m-dash "---"


%link new command names to the original font sizes,
%for easier to remember smaller font size
\newcommand{\vsmall}{\footnotesize}  % simpler to remember
\newcommand{\vvsmall}{\scriptsize}   %
%\newcommand{\vvvsmall}{\tiny}


\usepackage[colorlinks=true,linkcolor=black,urlcolor=blue]{hyperref}


\usepackage{ifthen}


% \needspace{5\baselineskip}      << reserves approximately 5 lines, leaves raggedbottom, more efficient
% \Needspace{5\baselineskip}      << reserves exactly 5 lines, leaves raggedbottom, less efficient
% \Needpsace*{5\baselineskip}     << leaves flushbottom if \flushbottom is in effect, otherwise ragged
\usepackage{needspace}



% \skill{blabla}
\newboolean{skillsaslist}
\setboolean{skillsaslist}{true}
\ifthenelse{\boolean{skillsaslist}}{\newcommand{\skill}[1]{\item[#1]}}{\newcommand{\skill}[1]{\subsubsection*{#1}}}
\ifthenelse{\boolean{skillsaslist}}{\newcommand{\openskillslist}{\begin{description}}}{\newcommand{\openskillslist}{}}
\ifthenelse{\boolean{skillsaslist}}{\newcommand{\closeskillslist}{\end{description}}}{\newcommand{\closeskillslist}{}}

% \action{blabla}
\newboolean{actionsaslist}
\setboolean{actionsaslist}{true}
\ifthenelse{\boolean{actionsaslist}}{\newcommand{\action}[1]{\item[#1]}}{\newcommand{\action}[1]{\subsubsection*{#1}}}
\ifthenelse{\boolean{actionsaslist}}{\newcommand{\openactionslist}{\begin{description}}}{\newcommand{\openactionslist}{}}
\ifthenelse{\boolean{actionsaslist}}{\newcommand{\closeactionslist}{\end{description}}}{\newcommand{\closeactionslist}{}}

% \eqitem{blabla}
\newboolean{itemsaslist}
\setboolean{itemsaslist}{true}
\ifthenelse{\boolean{itemsaslist}}{\newcommand{\eqitem}[1]{\item[#1]}}{\newcommand{\eqitem}[1]{\subsubsection*{#1}}}
\ifthenelse{\boolean{itemsaslist}}{\newcommand{\openitemslist}{\begin{description}}}{\newcommand{\openactionslist}{}}
\ifthenelse{\boolean{itemsaslist}}{\newcommand{\closeitemslist}{\end{description}}}{\newcommand{\closeactionslist}{}}


\newenvironment{readoutloud}%
{\begin{quote}\begin{itshape}}%
{\end{itshape}\end{quote}}%



% need a nice easily visible TODO marker
\newcommand{\todo}{\textbf{TODO:}~}
\newcommand{\TODO}{\LARGE\textbf{TODO:}\normalsize~}




\begin{document}




%-------------------------------------------------------------------------------
% cover page for work in progress, remove when completed
%-------------------------------------------------------

\pagestyle{empty}
%\vspace{5cm}    -- nope, not honoured at beginning of page ?

\begin{center}
\huge
Goblin Destiny

\vspace{2cm}

\normalsize
ongoing campaign\\
work in progress, writing as we go\\

\vfill

\today

\end{center}






%-------------------------------------------------------------------------------
% begin main matter
%-------------------------------------------------------
\cleardoublepage
\pagestyle{fancy}

% will be marking both left and right pages with an abbreviated section title
% since this is most likely to be read on screens one page at a time instead of
% printed in a binder/book with left/right pages visible simultaneously.
%\markboth{lefttitle}{righttitle}


%\mainmatter  -- nope, not in article ?
\setcounter{page}{1}




%--------|---------|---------|---------|---------|---------|---------|---------|
%       10        20        30        40        50        60        70        80
%-------------------------------------------------------------------------------
\section*{Goblin Destiny}
\markboth{goblin destiny}{goblin destiny}

A silly and short\emph{-ish} campaign where the Players take on the role of Villainous Goblins.

\

\textbf{These misadventures} follow the GamGang Goblin Clan and their struggle to survive and carve out a slice of the world for themselves. The Heroes are mostly goblins, suitably inbred to all be "family".

What gets them going is the first piece of an old prophecy coming true. Their land and livelihood is being invaded by another group of bandits and their Fearless Leader \emph{Gamling} dies during a highway robbery.

\

\tableofcontents                           % sets page header "CONTENT"
\markboth{goblin destiny}{goblin destiny}  % restore the page headers

\vspace{4\baselineskip}

\needspace{15\baselineskip}

These are the adventures that make up the main campaign arc. Most are crucial, but some can be skipped or won't happen depending on the outcome of the previous ones.

\begin{itemize}

    \item \textbf{another day in the office} Just a short intro fight to get the players oriented with their characters. Just some farmers, guards, and a bit of loot.

    \item \textbf{robbery gone wrong} Another band of robbers show up during a routine highway robbery. Gamling dies.

    GammelTant recognises that this is the beginning of the Dread Prophecy of \emph{Death by GrimGnash}. She is now the Leader of GamGang and immediately calls for Revenge!

    \item \textbf{kill the bandits} The invaders must be killed. Their camp has been scouted. An attack in the middle of the night is the brilliant idea to revenge Gamling and reclaim their land!

    \item \textbf{defend hoom-hool} The Horrible Hoomans have banded together and hired some Adventuring Heroes to root out the goblin clan. They must defend their HomeHole caves.

    \item \textbf{flee and relocate} The Horrible Hoomans try again, and this time they succeed. Forcing the goblins to flee.

    \item \textbf{clean the new home} Every new home needs a proper cleaning up. Get rid of all those pesky spiders, worms, balrogs, and other monsters living there.

    \item \textbf{plunder} A couple of raids go well and the goblins gather some wealth.

    \item \textbf{grimgnash} GrimGnash comes and kills all who do not flee. This concludes the campaign and fulfils the prophecy. Perhaps a few characters flee and survive for future adventures, perhaps they all die.

\end{itemize}

\

Below are small intermission adventures that are not part of the main campaign and can usually be played at any time.

\begin{itemize}

    \item \textbf{patrols} When travelling the gang runs into patrols. One or two horsemen and a few foot soldiers. Discuss, bluff, bribe, or fight. The patrolmen are not automatically confrontational. If the gang can pass for traders, travellers, or other civilians they can even get some information, or an escort on their way.

    \item \textbf{iffygriff eggs} They have spotted iffygriffs flying to an area in The Moors. An old outpost has an iffygriff nest, a small bandit gang, and a haunted guard tower with a cursed sword.

    \item \textbf{mutamonsters} They find a magical place in the forest which turns goblins and critters into strange mutated monsters. They can fight their way to the magic source and get strange mutated abilities and magic-ified equipment.

    \item \textbf{torture dungeon} They have found a treasure map pointing to somewhere in the Duns. Ruins of a tower with an old torture dungeon below filled with undead, black bugs, a strange trap, and some loot.

    \item \textbf{burglary sleepy fishesh} They have heard that a Richy Rich is staying in the Sleeping with the Fishes tavern. Must be worth going there and plunder his room and stuff.

\end{itemize}

\

The players can also go on other adventures, extra raids, weekend shopping sprees, etc. As long as they have enough food to feed GamGang they are more or less free to do what they want. But when GammelTant's stomach starts growling it's dangerous to be around the cook pot...


\phantomsection\addcontentsline{toc}{section}{introduction}
\subsection*{Introduction}

This campaign should be fun, silly, and short. In the end most Heroes should die or flee. Regardless of which, the GamGang Clan is no more after GrimGnash is done.

Goblins are squishy but numerous. The idea here is that each player controls two goblins simultaneous. When one dies make a new goblin with (85+1D10)\% of the the xp of the dead one. The first two to five goblins should come from a list of 10, then each new from a new list of three to five.

For players that absolutely don't want to play goblins, have them choose a race and roll a small set of characters from that race (human 3, dwarf 2, elf 2, halfling 2, orc 2). Any elf here should probably have a mental disability to live in a goblin gang, or perhaps they are anthropologists studying the life of the native lesser races?


\subsection*{General on GamGang}

Gamling leads the Gang with an iron fist and empty head. GammelTant is smarter but will not be much in play until after Gamling dies.
At the start of the campaign the GamGang crowd is made up of approximately 20 goblins, including Heroes, some 10+ runts, some wolves, critters, etc. A few runts can be added between adventures but goblins or other hero level NPCs must be recruited. Replacement Heroes are assumed to walk in from the woods. They can be long lost cousins, expeditionary members, returning hunters or traders, etc.

The whole gang, including heroes, require ca 15 "day's food" each day. The gang can forage and hunt for ca 50\% of that but Gamling and GammelTant will get angry if they don't have enough to eat and will sooner or later start cooking runts and goblins. This can include any seriously failing or obnoxious RPs.
Simple food costs ca 1c per 5d food, and weighs ca 1enc per day's food. So, to keep Gamling and GammelTant happy costs ca 2-4c per day in food.

The more food they have in the hole, the more goblins and runts will be at home when the guards and soldiers come to attack. If they don't have much food then most of the NPC goblins will be away scouring the woods for something to eat.


\subsection*{Goblin Destiny\,\textendash\,\emph{Death by GrimGnash}}
% \, is a small non-breakable, non-disappearing space
% \textendash is an en-dash, smaller than em-dash

The prophecy says that when The Great Leader dies by The Blue Invader, the GamGang must kill GrimGnash or be themselves destroyed.

GrimGnash is a dragon and there is no way in hell the Goblins will survive that fight. Thus, the destiny of the goblins is to die! \emph{Mostly}, some might escape and appear in later campaigns.









\newpage
%--------|---------|---------|---------|---------|---------|---------|---------|
%       10        20        30        40        50        60        70        80
%-------------------------------------------------------------------------------
\phantomsection\addcontentsline{toc}{section}{00 just another day in the office}
\section*{00 just another day at the office}
\markboth{just another day}{just another day}

This is a simple intro fight to get the Heroes tested out. So it's business as usual. Go to the twisty bit of the road and wait for the wagons. Capture some loot and get rich.


\subsection*{synopsis}

Travel to the "bestest loot spot", set up an ambush, fight the wagon train. Probably get at least some loot, and probably some of the opposition will get away.


\subsection*{adventure}

Gamling sends the Heroes to do a robbery on their own. They cannot bring any other goblins, wolves, etc from the camp. The Heroes have to fight as they are rolled up and with their starting gear.

\begin{readoutloud}
\emph{[Gamling speaks:]}
Yoo faitas! Yoo go hunt food! We hungry! Me Bizzi! Go hunt loot at Bestest Lootspot.
\end{readoutloud}

The Heroes can pick up some minor equipment from the camp, but not much of value. They can't go buying anything new before this first "mission" either. If they want to bring something of value from the camp they have them argue with Gamling about it. Haggle, Leader, or any "fancy" intelligence based arguments won't work since Gamling will just smack them with superior strength and fighting skills. Instead have a psy-vs-psy nagging "argument" modified mod-1 for every 10c value they want to bring. If they fail-3 then Gamling will smack the arguing Hero until he shuts up.

The camp only have a few days of food, some torches, a few extra knives, some staffs and clubs, some shitty spears, rusty axes, goblin short bows, etc. A few hides, some rope, and other natural equipment is also available.

The wagons only move during the day so no use bringing night fighting stuff like torches. They can of course try to seek out the wagons during the night if they want. They need enough "track" or actually move the party onto the wagon night camp to find it. The wagons travel from Kleinshof to Sleepy Cove, which at 10sq/d takes two days. They camp for the night just north of the trail fork. \emph{No map included for this,} but randomise a light forest map and set up a wagon circle with tents, fire pit, lookout, etc.

Have the players set up the robbery as they please. They have plenty of time to prepare. If any of them pass int rolls you can helpfully suggest things like: "If you'd had anyone with the skill \emph{traps} you could have set traps by the road..."

When the players are finished: start the fight by having the wagon train move onto the map, wagon by wagon over a few rounds. Hjalmar Hjälte leads the train, walking just in front of the first wagon.


\subsection*{map mechanics}

The route along the road is almost 100sq long. Assume the wagons moves at a "safe" 5sq/r (slow run) after being attacked. That gives 20r for the GamGang to force them to stop before they leave the map.

One way to get more time is to barricade the road with a tree, stones, their donkey, angry goblins, etc. They can also try to break one of the wagons, or kill a donkey to have the wagon stop at a difficult to pass spot, or some such. But don't hint at them to barricade the road. More fun if they for example have a running fight, chasing the wagons through the map.

The wagons enter the map at speed 3, with 3-6sq empty space between them. Then accelerate to speed 5 when in a fight. The wagoneers and heroes have mod-3 to their perception rolls unless something indicates that there might be an attack incoming. Things like a tree felled over the road will indicate an attack and make them suspicious and careful. Same for goblins in plain sight.


\subsection*{opposition: wagons and Hjalmar Hjälte}

The wagons are led by Hjalmar Hjälte, a not too difficult hero, and are guarded by Swordsmen (150xp) and Archers (150xp).

Hjalmar Hjälte and one wagon should be enough to go against four goblins, then bring in another wagon for every two additional goblins. Recommend first two wagons have swordsmen, then one with an archer, then another archer or swordsman.

If the road is not blocked the wagons will drive at 3sq/r until attacked at which time they will go 5sq/r. The wagons can go faster, but risk breaking down. For every speed step above 5 a wagon suffers 10\% chance per round to loose a wheel, overturn, etc. E.g. at speed 8 it's a 30\% chance per round.

If the road is blocked they will go into defensive position bringing the wagons together, phalanx fighters with archer support, etc. If the Heroes wait to attack (must pass psy rolls) the farmers will finally act to remove a road block or impediment, but will be guarded by the swordsmen and Hjalmar Hjälte.

If the farmers are loosing and can flee they will dump some loot and no longer suffer the risk of breaking their wagons when they go fast. Without loot a wagon can safely travel 10sq/r.

Hjalmar Hjälte will retreat and quaff if hurt. If he still can't win he will flee when almost dead. He will make simple attempts to coordinate with the swordsmen guards, and possibly the farmers, but nothing super clever.

Story-wise the best outcome is if Hjalmar Hjälte manages to flee together with some wagons. Then it's natural that he starts trying to gather a posse to go after the goblins later in the campaign.


\subsection*{loot}

The wagons have mostly food, simple materials, etc. Assume each wagon has total loot for around 10s, but nothing fancy. Just look through the price/equipment lists and pick a few items. E.g: 50x days food, 2x 10l kegs of beer, 6x blankets.\\
Each farmer has 1d3 silver and 1d10 copper.\\
Swordsmen and archers have 1d5 silver and 1d20 copper.\\
Hjalmar Hjälte have 3+1d5 silver and 1d30 copper.\\
Also nice if one of the wagons have a hidden bag of 10-20 silver (perception roll when looting).

The wagon train donkeys are difficult to subdue since they don't like goblins. Requires a series of str vs str successes, or a smaller series of ride successes, or an animal command success per donkey. If they bring wagons it's much slower. The wagons only have half speed through forest, and require successful "ride" roll per day to avoid getting stuck or damaged by terrain if not on roads.

The loot cannot be sold in Kleinshof since that's where it comes from. The wagons or donkeys cannot be sold in Kleinshof, Sleepy Cove or Kegern. The villagers will recognise the donkeys or wagons and have Conq Solders arrest the Heroes for thievery! Interesting little side adventure if they try.


\subsection*{play through}

Four players, 2x goblins each, around 90xp freshlings. Highly varied builds.
Brought Hjalmar Hjälte and three wagons, two with swordsmen and one with an archer.

\

First session: \\
They started with felling a tree to block the road before the wagons show up, then spread out and hid. Waited until the wagon train arrived and two farmers got axes to chop up the tree and clear the road. Then they attacked at multiple fronts.

Hjalmar Hjälte was ahead and got seriously hurt before the farmers and swordsmen could arrive to help out. He retreated, hid in a tree, and quaffed to be able to get back in the fight. The swordsmen and farmers started fighting, and the archer was harassed by a backstabbing sneak.

Killed two farmers, disabled one goblin, hurt half the goblins to ca half hp.

\

Second session: \\
They killed one of the swordsmen before he had the time to quaff. Hjalmar Hjälte killed one goblin and a support critter. Then HH fled with the surviving archer, swordsman, and farmer.

All carts were captured along with one donkey, the other two escaped.

\

Loot: \\
Awarded 10xp each for adventure and ca 30xp to share for opposition:\\
\emph{HH:10xp, Swordsman:5xp*2, Archer 5xp, farmer 2xp*3}\\
Misc loot from the wagons and some weapons and armour from the dead humans.

\

Followup: \\
Hjalmar Hjälte survived together with a swordsman, an archer, and a farmer. Those characters travel back to Sleepy Cove and send word to the Lord of the region saying that the new goblin band is very dangerous.




\newpage
%--------|---------|---------|---------|---------|---------|---------|---------|
%       10        20        30        40        50        60        70        80
%-------------------------------------------------------------------------------
\phantomsection\addcontentsline{toc}{section}{01 robbery gone wrong}
\section*{01 robbery gone wrong}
\markboth{robbery gone wrong}{robbery gone wrong}

Business as usual. Go to the twisty bit of the road and wait for the wagons. Capture some loot and get rich. The problem is that the SturSkurk gang shows up and Gamling gets killed. This is the start of the \emph{Goblin Destiny: Death by GrimGnash}.


\subsection*{synopsis}

GamGang prepares a highway robbery. The wagon train enters and the fight starts. Soon the SturSkurk Gang enters the map. Gamling charges SturSkurk Boss and quickly dies.

The idea is to have the rest of GamGang soon flee the fight, taking perhaps some loot with them. It should be unlikely but not impossible to beat both the wagon train and SturSkurk Gang.


\subsection*{adventure}

The food from the first raid is gone, and the GamGang is starving again. Must have more food! More Loot! This time Gamling himself will lead the outing.

\begin{readoutloud}
\emph{[Gamling speaks:]}
Em faitas! We go hunt loot stuff! Grabba em spearsanswords! Hurtysticks for all! Waik em shootas. To e twisty bit we go. Tiny Looka has seen beeg wagon komma! Lotsa loot! Go! Go!
\end{readoutloud}

Since Gamling is with them they can also add some support goblins. A couple of disposable warriors or fighters. As long as they can all travel fast enough on the region map, the camp is soon hungry again!

Allow the players to position and prepare for the robbery. But it can't be too clever or Gamling will scream at them to do it differently.

The idea during the fighting is that GamGang should be forced to flee sometime after Gamling is slain. Perhaps they can pick up some loot along the way.


\subsection*{map mechanics}

Same as before, see the map mechanics section under the previous adventure.
But this time the wagon train has the Wagoneer in the end, which is different. His armoured-illers can clear the road to some extent.
As soon as the fight has started properly the SturSkurk Gang will arrive.


\subsection*{opposition}

With first the farmers and Wagoneer, then the SturSkurk Gang, it is unlikely that the Goblins will win this fight. Try to balance so that the players feel it really is \emph{unlikely but not impossible} to beat both the wagon train and SturSkurk Gang. It is possible to get the SturSkurk Gang to fight the wagon train, and carefully pick off some of the people on both sides until the rest of GamGang has a chance to win.


\subsection*{opposition: wagons}

The heavy Wagoneer  counts for 2x goblins. For every further 2x goblins add another wagon with guard. Use archer guards and get the farmers to escape as soon as possible.

The heavy Wagoneer comes last in the wagon train. If the road is blocked his armoured-illers can push aside other wagons, or perhaps even a tree if they have some time. The Wagoneer will fight if necessary, but will try to just escape.

The farmers will try to escape, including turn around if it seems to help. Archers can fire from wagons with mod-3.


\subsection*{opposition: SturSkurk}

The SturSkurk gang enters the map in staggered group. Main melee line phalanx, the witch in between, then archers group behind.
They will start to attach the goblins, but will also attack farmers and the Wagoneer if they can snag some loot.

SturSkurk will fight and kill Gamling first, since Gamling will charge him early in the fight. The melee phalanx line will protect their archers, and only push into the fight if their archers are not strong enough to win a shooting duel.

The Witch will shock bolt and cast slow on Heroes when possible to aid her melee line. She will also run up and use ward shield on the boss and orcs if necessary. Defensively she'll use fire storm.

The SturSkurk archer group will run around to stay out of reach and pepper arrows when possible to suppress Heroes' archers and casters.

If it's obvious that they will loose then Boss will order a retreat.


\subsection*{loot}

The regular farmer carts have similar loot as in "00 just another day in the office", ca 10s worth of food and simple items.\\
The heavy wagon has 100+1d100 days' food, ca 20s of random trade stuff, and a couple of high end items e.g. Quaff!. The wagon is a small camping house and also has a small stove, some cooking gear, a few days of food, minor trinkets, clothes, etc. In a \emph{hidden compartment} in the wagon is 20+1d50 silver and 50+1d100 copper (find roll required mod-0, can't find with perception roll).\\
SturSkurk people have just 1d3s + 1d10c each. Their gear is in ok condition.

Award ca 15xp each for the adventure, and shared xp for the defeated opposition. Don't deduct if the Heroes flee. They are supposed to run away here. Add another 5xp each if they force SturSkurk to flee.


\subsection*{play through}

Starting with felling a tree

\

Session 1:\\
Put up a robbery site in the NW part of the map, blocking the tight corner with a tree trunk they dragged there. Misc damages but no death. Cut loose a donkey to get hold of at least one wagon.\\
SturSkurk Gang enters in r20 and is discovered at end of r21.\\
\\
r1-16 quick, r17-21 combat

\

Session 2:\\
Just noticed SturSkurk Gang. Gamling is closest to them, mid way between SturSkurk and rest of GamGang. Gamling charges and dies in two rounds. The others retreat and regroup. They force the farmers and archers to flee, and kill a swordsman. They loot the wagon where they cut loose the donkey, and steal the other two wagons with donkeys when they flee.\\
Gamling is lost of course, and one more. A few are at very low health.\\
\\
r22-28 combat, r29-31 flee

\

Loot:\\
Award ca 15xp for the adventure: Ca 10xp base, 5xp for getting away with some loot, 5xp if they manage to stop and loot the Wagoneer, 5-10xp if they force SturSkurk to flee.\\
Award kill xp for everyone they kill or they force to flee. E.g: three wagons with farmers and escorts is worth around 24xp.\\
Don't deduct if the Heroes fled. They are supposed to flee here.

\

Followup:\\
SturSkurk gangs don't have any magical healing. Their wounds will heal at normal speed, so if the GamGang is quick to go on with "02 kill the bandits" then their opposition will still be wounded.


\section*{Gamling is Dead, all hail GammelTant!}

After Gamling has died GammelTant will be furious for a bit, call for revenge, and send Hemskelina to scout the SturSkurk gang. Soon she remembers the Old Prophecy of \emph{Death by GrimGnash}, and locks herself away with her crystal ball.

When Hemskelina comes back reporting of the SturSkurk bandit camp GammelTant immediately sends out the Heroes to kill them all and steal all their shit. And of course to bring back the corpse of the Blue Boss so she can cook a Victory Revenge Stew!






\newpage
%--------|---------|---------|---------|---------|---------|---------|---------|
%       10        20        30        40        50        60        70        80
%-------------------------------------------------------------------------------
\phantomsection\addcontentsline{toc}{section}{02 kill the bandits}
\section*{02 kill the bandits}
\markboth{kill the bandits}{kill the bandits}

The camp of the SturSkurk bandits has been found and scouted by Hemskelina. It's time to kill them all! And eat the good ones.


\subsection*{synopsis}

Kill the SturSkurk bandit gang by assaulting their camp at night.


\subsection*{adventure}

This time they have no leader with them. Hemskelina is coming along but will not fight as a lead figure. Her task is to kill Blue Boss if he tries to escape.
The Heroes can bring a couple of fighters, warriors, runts, and a wolf if they want.

\begin{readoutloud}
\emph{[GammelTant shouts:]}
Go Get Em Killt! Blue Boss bring back! Kook-em corpse in Victory Revenge Stew I will! Take-a em few warrjas and fajtas with-ya!
\end{readoutloud}

The Heroes can position and prepare for the camp assault but they cannot dig trap pits or do similar heavy work/noise activities in the area without getting spotted.

SturSkurk gang will have a competent scout on guard patrolling the camp. Hemskelina will track him but need help to kill him quickly.


\subsection*{dungeon mechanics}

bla bla bla

\small \begin{verbatim}
bla bla description etc
\end{verbatim} \normalsize


\subsection*{opposition}

bla bla bla


\subsection*{loot}

bla bla bla


\subsection*{playthrough}

bla bla bla







\newpage
%--------|---------|---------|---------|---------|---------|---------|---------|
%       10        20        30        40        50        60        70        80
%-------------------------------------------------------------------------------
\phantomsection\addcontentsline{toc}{section}{03 defend hoom-hool}
\section*{03 defend hoom-hool}
\markboth{defend hoom-hool}{defend hoom-hool}



\subsection*{synopsis}

The Burmak Guard Squad attacks Hoom-Hool. Should be balanced so the Heroes barely survive and some attackers flee. If the Heroes flee and are forced to relocate already after the first attack then skip the Lundqvist attack in "04 flee and relocate".


\subsection*{adventure}

This adventure is supposed to be difficult but the Heroes should win and force Burmak and the humans to retreat.
For Burmak and six spear men and Hjalmar Hjälte alive from before it's recommended that the Heroes have ca eight goblins at ca 130xp each, along with Hemskelina, GammelTant, some warriors and fighters, and some runts.

\begin{readoutloud}
\emph{[SomeDude speaks:]}
Go Get Shit Done!
\end{readoutloud}


\subsection*{dungeon mechanics}

With them are any survivors from the "00 just another day in the office", and perhaps a couple of experienced hirelings.
bla bla bla

\small \begin{verbatim}
bla bla description etc
\end{verbatim} \normalsize


\subsection*{opposition}

bla bla bla


\subsection*{opposition: Burmak Guard Squad}

Guard Captain Burmak and his squad of spear men have been tasked to root out the Goblin Menace. Burmak will keep his men in tight formation, using phalanx and spear reach to attack with a second line. The front line will declare more actions and be defensive, while the back line will declare fewer actions and be offensive over their front line brethren. The back line can also throw javelins.

Burmak will fight from the rear, using his leadership, high initiative, and Oy! to keep his troops fighting in synchronisation and high initiative.

Burmak will call retreat if he gets hurt below 8hp or looses half his men.


\subsection*{opposition: survivors}

The survivors from the "00 just another day in the office" robbery will come along. If Hjalmar Hjälte is with them they will probably lead the charge, and if not they will be behind Burmaks squad.


\subsection*{loot}

The soldiers and survivors don't have much money on them. Their equipment is in good condition though.


\subsection*{playthrough}

bla bla bla









\newpage
%--------|---------|---------|---------|---------|---------|---------|---------|
%       10        20        30        40        50        60        70        80
%-------------------------------------------------------------------------------
\phantomsection\addcontentsline{toc}{section}{04 flee and relocate}
\section*{04 flee and relocate}
\markboth{flee and relocate}{flee and relocate}


Another attack on Hoom-Hool. This time Lundqvist Soldier Squad will force the GamGang to flee and relocate.


\subsection*{synopsis}

The over powered Lundqvist squad will be too difficult for the GamGang goblins and they will have to flee.


\subsection*{adventure}

The people of Kleinshof and Sleepy Cove have finally managed to get a local lord (Conq?) to send out a proper military group to root out the Goblin Menace hiding in the woods.

\begin{readoutloud}
\emph{[SomeDude speaks:]}
Go Get Shit Done!
\end{readoutloud}


\subsection*{dungeon mechanics}

bla bla bla

\small \begin{verbatim}
bla bla description etc
\end{verbatim} \normalsize


\subsection*{opposition}

bla bla bla


\subsection*{loot}

bla bla bla


\subsection*{playthrough}

bla bla bla










\newpage
%--------|---------|---------|---------|---------|---------|---------|---------|
%       10        20        30        40        50        60        70        80
%-------------------------------------------------------------------------------
\phantomsection\addcontentsline{toc}{section}{XX patrols}
\section*{XX intermission: patrols}
\markboth{patrols}{patrols}


Patrols from the Castles or villages travel the lands to search for enemies, hunt bandits, and safeguard routes for travellers and merchants. The patrol soldiers are not automatically hostile. All depends on how well the goblins are known, what they do when meeting, and what they are carrying.


\subsection*{synopsis}

When travelling they run across patrols. Talk or fight. Look and behave like civilians or fight like bandits. Small low level patrols are easy to trick unless anything "bandity" is obvious. They are also not that tricky to fight.


\subsection*{adventure}

Patrols are most likely along the roads, but will be out in the field as well. They usually don't enter the forests unless necessary.

\begin{readoutloud}
\emph{[Friendly Patrol Soldier speaks:]}
Hello hello there! How's life this fine day? Where are you fellows travelling this day? Anything interesting in those wagons? My dear wife is looking for some fine cloth for a new dress. Can you give me good price?
\end{readoutloud}

The Heroes need to either talk their way through the patrols by pretending to be traders, or bribe them, or fight them. The important part is to not allow any patrolmen on horseback to escape, which will immediately bring more soldiers down on them, as soon as they can get there.

If the goblins can pass for peaceful traders or farmers this is not an issue. If they start fighting and the soldiers are loosing they will try to flee, and one will take off on horseback to get help. Any soldiers that manage to flee will try to follow the goblins at a safe distance (a few leagues behind), but only have limited track (3 or so).

Response groups can be called from Castle Pawa or Castle Conq and can travel up to 15 leagues per day. They will be 5-8 cavalry soldiers and have a scout (track 7) with them. They will likely be very quick to track down the goblins unless they can get into hiding quickly, or are very good at track or sneak to sufficiently hide their tracks.

Civilian militia response groups can be called from any village, but will generally only be farmers, hunters, and a couple of hirelings. They travel at 8 leagues per day, and will have a good hunter (track 9) with them. A civilian militia group will not necessarily attack when they find the goblins. It may be enough that they track and stay nearby, to join with a later real military group.

If a civilian militia is raised from Sleepy Cove and Hjalmar Hjälte is alive, then he will join it together with any other survivors from the previous battles who happen to be available in the village.


\subsection*{dungeon mechanics}

Just open field or plain. Add some rocks and trees, or old ruins, to make the terrain more interesting, but it should be relatively open battle ground.
Encounter will probably be during the day, unless the Heroes see a patrol camp, or the weary solders see the Heroes' camp.


\subsection*{opposition}

Just a small patrol group. One or two on horseback and three to five on foot. The soldiers fight with crossbows at range, one or two salvos, then switch to spear and shield phalanx in close quarters.

They will try to flee if the fight is not going their way, and they will immediately try to send someone on a horse to get help and report once they see that they can't handle the goblins themselves.


\subsection*{loot}

The soldiers' gear is in good order, and they have some coin. A horse or two is also worth quite a bit of money, or food. A civilian militia group has less gear and coin.


\subsection*{playthrough}

This encounter doesn't take much time. We played through one when the GamGang tried to travel to Sleepy Cove to buy food and sell off some loot. The Heroes were horrible at speaking Common so the initial talking was difficult, but then a couple of hot headed goblins started swinging clubs, aaand it was off to the races.

They managed to immediately kill the soldier on horseback first when he tried to make an attack run, and then quickly caught and dispatched the rest of the soldiers with magic (slow, fireball).

20 rounds: r1-10 distance, close, talk, r11-r19 battle










\newpage
%--------|---------|---------|---------|---------|---------|---------|---------|
%       10        20        30        40        50        60        70        80
%-------------------------------------------------------------------------------
\phantomsection\addcontentsline{toc}{section}{XX mutamonsters}
\section*{XX intermission: mutamonsters}
\markboth{mutamonsters}{mutamonsters}


\subsection*{synopsis}

GamGang discovers a magical stream in the forest. It can turn them into strange mutants and turn weapons magical. They can fight mutated monsters for access. If utilised some chars can get special powers or magical stuff.


\subsection*{adventure}

Either they stumble across this are when travelling through the forest, or they get alerted when two very strange runts show up in their camp. If just happening on the area just put them on the map without any significant intro.

One day two runts come back from a hunt, seriously mutated. One has wings and the other has three arms. They are also somewhat changed mentally. Also, the stone knife carried by one seems to have become magical.

\begin{readoutloud}
The forest is deep and dark, ancient and powerful. No sunlight make it down through the heavy canopy of enormous trees, the forest lies in deep darkness. In the distance you see soft blue-green lights slowly move among the trees. And there is a strange tingly feeling in the air.
\end{readoutloud}


\subsection*{map mechanics}

The map is always a low light area, even in the middle of the day this magical deep forest is always nightly dark.

This is an endless map. More critters will always keep coming, with periods of low population or empty map if the Heroes are fast enough to kill off everything. The surrounding forest spawns light bugs, air fish, moss lumps, and larva. The stream spawns air fish, moss lumps, and tentacle crabs. Stuff trickle in every few rounds, with larger batches every 10-20r, and full synchronised repopulation waves around every 50-100r over a few rounds. New monsters don't immediately start attacking the Heroes. They will meander around for a bit, but get aggressive when attacked, and start to draw others to the fight.

The monsters are aggressive towards non-mutated folk and animals. They ignore mutated people as long as they are not attacked.
Light is an issue. The light bugs are attracted to light sources, and will get aggressive when they find non-mutated folk near a light source or in their own feeble light (2sq or so). The lightbugs will however not fly into major fire, e.g. fire balls, etc.

It's ca 30sq to the magic island if they follow the road, so at least 10 rounds at normal pace, and 3-5r if someone manages to dash there to scout. It's about 20sq straight shoot through the trees, so the island can seen after the first tree, or glimpsed if they explore south a bit.

All the wild magic in the area is refreshing for the goblins. Any goblin heroes (not other races) regain 1 mana and 1 stamina each 10r.

The mutamonsters are also randomly magical. If attempting to drain them they will have 1d10 mana available and 1d10 psy for the resistance roll. After 5+1d10r this has "recharged" to a new random amount and protected by new random psy value. It is also possible to push the critters to sleep by draining them  to 0 temporary mana, and they wake up on psy rolls as usual.

\

The first successful expedition to the magical light will give full xp for the exploration, kill, and map clearing. If they return again to boost a new character or weapons the map will have repopulated, but will only give half xp for kill and map completion, and none for exploration.

All later returning expeditions should give 0xp for returning Heroes and a cap of 10xp per session and 30xp total for a new Hero who is there for the first time.


\subsection*{opposition}

Mutated monsters: Ranging from very large tentacle crab to tiny explosive light bugs. The monsters don't coordinate much, but the light bugs will probably drive people on the move during the fight, or require that a spellcaster keep bolting them. As they move they will stir up and attract more monsters as they go. Once in a fight the violence and smell of blood will quickly attract other mutamonsters.

The mutamonsters don't attack other mutated beings. So the already mutated runts can move freely in the area if they are present. Don't tell the players this though, and see if they notice. The mutated runts don't want to attack the mutamonsters either, so they must be forced to do so. This also applies to mutated Heroes who return to the map later. They won't get attacked until they get aggressive.

Start out with a tentacle crab half in the water near the magic island, and 1-2 larva roaming the forest, along with 5-8 moss humps, 4-6 airfish, and 10-12 light bugs spread out with another 3 near the magic island.
Before the fighting starts the light bugs and airfish lazily amble around 1-2 sq/r, the rest stay mainly still.

As long as the Heroes are in the area new monsters will come from the forest and the stream, random alone and in waves. Keep the players stressed. This is not just a clear out and rest place. It takes fighting and mobility to stay here any significant time. Short respites can be had if they are quick to kill off and clear the map.

If the Heroes flee the map it will rapidly repopulate before they can return.


\raggedbottom
\begin{description}

\item[Light bug:] small and slow bugs that buzz around and are attracted to the strongest light source they can find, but not to large fires (e.g. fire balls etc). They have r1 light source but are not attracted to each other's light. If they get close to the Heroes they will kamikaze attack (dash 6), but won't attack "mutated" Heroes. Explode on contact: r2 dam 4. The explosion cloud lights the area r4, but is removed at end of turn.
They also explode if hit. Since they are difficult to hit with ranged attacks and melee will be within explosion range the only simple and safe way is magic.
They fly: reach 1 is needed to hit them with melee weapons unless they are making a kamikaze run.
\goodbreak \begin{samepage} \small \begin{verbatim}
===================================
Mutamonster Light Bug       (token)
-----------------------------------
hp 1 abs 0, set size tiny
m2 w- r- d6
initiative 10
tiny target: ranged mod-6 melee mod-3
kamikaze attack 8 toavoid-2, explode on contact
                  set dash 6 speed when making a kamikaze attack run.
explode: r2 dam 4, explosion clouds are removed at end of turn.
-----------------------------------
\end{verbatim} \normalsize \end{samepage}


\item[Airfish:] glide serenely through the air, unleashing magical attacks on intruders. They fly: reach 1 is needed to hit them with melee weapons.
When avoiding they get Dash speed and 8mp extra, but can't zap that turn.
Zap attack resets the charge, and they start recharging again next round. Meaning they can zap each round, but only at charge 1.
\goodbreak \begin{samepage} \small \begin{verbatim}
===================================
Mutamonster Airfish         (token)
-----------------------------------
hp 6 abs 0, set size small
m4 w- r- d8
initiative 12
pain threshold 2
flying target: ranged mod-3, melee reach 1 necessary
avoid 7: fly high and dash away 8sq (max once per round)
zap 5 +1/abs: dam 3+charge stun 3*charge
       range 4*charge
       charges max 5 generate +1/r
       use multi counter 1-5, start charging again next round after fire
-----------------------------------
\end{verbatim} \normalsize \end{samepage}
A dead airfish gives 5enc worth of excellent sparkling sashimi.


\item[Moss lump:] a simple lump of ground vegetation has sprung to life. It has lots of hp and no pain. Attacks by smashing with arm growths. Guards the stream and other monsters. Will attack on sight, and coordinate in small groups.
\goodbreak \begin{samepage} \small \begin{verbatim}
===================================
Mutamonster Moss Lump       (token)
-----------------------------------
str 8     hp 25 abs 0, set medium size
dex 3     m3 w- r- d6
per 4     initiative 4
pain threshold 99
black knight 1
balance 3
smash: 7 dam 6 toavoid+1
       knockback 1
       2x per round (left,right)
Damage reduction (half damage, round down):
    Wet moss, takes half damage from fire and shock.
    Wet moss, takes half damage from piercing attacks (arrows, spears, etc).
Has no defence.

Moss lumps spontaneously regenerate over time if not specifically hacked to
pieces after they are "killed". Starts twitching slowly, then over a few rounds
they rise and start moving. Revive with 5hp, regen 10hp 1hp/r.
-----------------------------------
\end{verbatim} \normalsize \end{samepage}
There is no good eating on the moss lumps, even for goblins.


\item[Larva:] is a seriously angry and overgrown larva. The goblins will recognise it as a ThumbGrub. It is usually about the size of a goblin thumb, and very tasty. Now it's a couple of meters long, angry, and out for revenge!
It will charge with horn attack, then grab hold with barbed legs, hold and start chewing. The two front legs will be used for grab-hold attacks. Either at different targets or both on the same target. Treat each leg separately. Holding with one leg gives mod+3 for the attack with the second leg. Break free from each leg separately as well. The larva does not loose actions to resist attempts to break free from the barbed leg grips.\\
Specifically chopping off a leg requires a cutting weapon, is a mod-3 attack, and must do $\geq$8 damage after abs is deducted.
\goodbreak \begin{samepage} \small \begin{verbatim}
===================================
Mutamonster Larva           (token)
-----------------------------------
str 10    hp 50 abs 3, set large size (2x2)
dex  5    m2 w4 r6 d8
per  3    initiative 6
pain threshold 10
balance 6
charge 6 (set as mobile, so he can horn bump then grab hold in same round)
horn charge 8: dam 5 pen 2 trip or knockback
               trip: 10 vs dex of target to fell it to the ground for grab hold
               knockback: knock it 1 sq away and see if it falls there (dex-3)
leg grab hold 6: dam 1 penetrating, str vs 10 to break free 3ap
chew 7-abs: dam 3 penetrating, only when grabbed-held.
Has no defence.
-----------------------------------
\end{verbatim} \normalsize \end{samepage}
The larva body, inside the exoskeleton, is more of a sloshy liquid goo than real meat. It cannot be cut up and transported. Instead it must be stored in barrels, large skin sacks, or buckets. A larva holds ca 50enc worth of edible goo.


\item[Tentacle Crab:] is a swollen nightmarish mixture of snake nest and freshwater crab mollusc. It will try to grab hold of it's target and drag it to the mouth to eat it. Each tentacle is a separate unit that can stretch out to 2sq away from any part of the crab. It can have up to 6 tentacles at any time, and can grow a new one every round. Treat each tentacle separately.

The crab does not move far from the water. It will pursue an attacker for a bit then retreat back to the stream. If it gets $\leq$30hp it will try to retreat into the water and submerge. There it will heal 3hp/r.
\goodbreak \begin{samepage} \small \begin{verbatim}
===================================
Mutamonster Tentacle crab   (token)
-----------------------------------
str 20    hp 80 abs 0, set huge size (3x3)
dex  5    m2 w4 r6 d8
per  3    initiative 8
pain threshold 15
large target, range tohit+3
balance 6
chew 10-abs: dam 5 pen 2, only when grabbed and brought to mouth.
Regrows one tentacle each round.
-----------------------------------
Tentacle
str  4    hp 5 abs 0, set tiny size
dex  5    m6
per  3    initiative 12
each tentacle moves independently
each tentacle can stretch out 2sq from any point of the crab body.
small narrow target: range tohit-6, melee piercing tohit-3 (spear, etc)
avoid 4
slap 7: dam 3
grab 5: dam 1 pen 2
        each tentacle grab gives target mod-3 to all actions
        break free is str-vs-N*4 where N is number of tentacles
drag:   when grabbed the tentacles will cooperate to drag the victim to mouth
        N*5-vs-str, 0ap, one attempt per tentacle per round.
-----------------------------------
\end{verbatim} \normalsize \end{samepage}
The Tentacle Crab yields about 100enc of meat, and each tentacle another 3enc.

\end{description}
\flushbottom


\subsection*{loot}
%-----------------
Magical Mutationous Powers and Mutamonster Meat awaits those who survive long enough.

Miscellaneous body parts from the mutamonsters can also be sold in the villages at 50\% chance, and shipped to TradeTown for more gold if the Heroes can arrange for it.

If they manage to freeze, sleep, or otherwise capture lightbugs and bring them to the camp, Goblanda will be excited and start cooking up fun stuff. Including detonation grenades in 1d10+5 days, with each lightbug yielding two grenades.


\subsection*{loot: mutamonster meat}
%-----------------------------------
Mutamonster meat is edible, but with side effects. Eating mutamonster meat should have some fun effects like temporary stat changes, odd roll behaviours, perhaps very temporary abilities or curses. And of course: interesting cosmetic anomalies. Temporary in the beginning, but if the GamGang set up the forest stream as a significant long term food source the oddities should become permanent.

Aside from fun miscellaneous side effects, all food prepared from mutamonster meat gives temporary psy-3 for two days. For every week eating high mutamonster meat diet the Hero takes a permanent psy-1.


\subsection*{loot: magic mutations}
%----------------------------------
Going to the island and touching the magic light has some chance to give serious bodily changes, as well as affect equipment brought there. The longer the Hero stays there, and the more he touches the light, the more mutations he will get. At first it's generally good, then it will become worse and worse. Be careful to describe the effect and give the player a decent chance to figure out that it might be a bad idea to stick around too long. Each "tick", giving a new mutation seed, should take a few rounds to give the player time to think.

\

Example:
\begin{readoutloud}
Climbing onto the little island you reach out to touch the shimmering light. A sudden spark hits you with a quick stabbing pain \emph{(take 1dam 1pain)}.\\
Suddenly the skin around the strike starts to turn \emph{blue/pink/purple/orange} with discoloured tendrils rapidly growing up your arm onto your body.\\
\emph{E.g: growing a wing:} You feel muscle and bone starts shifting in your back and a small lump growth starts pushing against your armour.\\
\emph{E.g: extra arm:} An itchy crawling feeling in your left armpit... a tiny runt arm is crawling out of an angry red oozing welt.\\
Your mind is reeling from the strangeness \emph{(take psy-1 permanent)}.
\end{readoutloud}

The first mutation should be something small and cosmetic: big eyes, glowing snot, purple skin, extra big flapping ears, smoking nose, tail, long curly hair, extra fingers, snake tongue (speaking-2, per+1), etc. This first hit gives a psy-1 permanent mod. Successive mutations do not have \emph{this} automatic penalty.

If the Hero stays longer, or touches the light again for a second time, he will get a serious ability. Roll from the list of serious mutations. Initially it will just manifest as a weird lump or similar but will, over time and with xp, grow into whatever is suitable for the ability. An extra arm, smelly pustules, brain nodules, etc.

Remaining in the light longer and repeatedly touching the light will get worse and worse with more detrimental mutations, draining stats, etc. Effects:\\
1st: roll 1x Cosmetic weirdness and 1x Minor boost\\
2nd: roll 1x Serious mutation with suitable cosmetic component\\
3rd: roll 1x Major boost, 1x minor drain\\
4rd: roll 1x Dubious mutation with suitable cosmetic component, 1x minor drain\\
5th: roll 1x Cosmetic weirdness, 1x minor boost, 1x minor drain\\
6th: roll 1x minor boost, 1x minor drain\\
all further rolls give 1x minor drain and int-1 or psy-1 (50/50)

\

The mutation seeds will develop over time. If the Heroes don't spend xp on them, let them start stealing some of the xp from each session and start developing on their own anyway.


\raggedbottom

\goodbreak \small \begin{samepage} \begin{verbatim}
Cosmetic weirdness, 1d16:
 1-3   big eyes
 4-5   glowing snot
 6-8   colourful skin: purple/pink/orange/yellow/red
 9-10  polka dots
  11   moving moustache
  12   clattering teeth
13-14  useless tail
15-16  smoking nose
17-18  squeaking knees, sneak-1
\end{verbatim} \end{samepage} \normalsize

\goodbreak \small \begin{samepage} \begin{verbatim}
Minor boosts, 1d13:
 1-3   str +1
 3-6   con +1
 7-8   psy +1
 9-10  hp +1d2
  11   stamina +1
12-13  mana +1d3
\end{verbatim} \end{samepage} \normalsize

\goodbreak \small \begin{samepage} \begin{verbatim}
Major boosts, 1d24
 1-3   str +1d3
  4    dex +1
 5-7   con +1d3
  8    int +1
 9-10  psy +1d3
 11    per +1d3
 12    cha +1
13-16  mana +1d5
17-18  hp +1d5
 19    abs +1
20-22  stamina +1
       vision +1d5 range +1d
23-24  increased size, goblin becomes normal size, str+2, con+2, hp+3
       must eat full rations each day
       re-roll for already full size Heroes
\end{verbatim} \end{samepage} \normalsize

\goodbreak \small \begin{samepage} \begin{verbatim}
Serious mutations, positive, 1d26:
 1-5  Extra Extremity, roll for which below
 6-8  Tasty
 9-11 Foul
12-13 Reactive Skin
   14 Regeneration
   15 Grapple
   16 Incorporeal Double
   17 Teleport
   18 Reach
   19 Blink
20-22 Cockroach
23-24 Black Knight
25-26 Painless
\end{verbatim} \end{samepage} \normalsize

\goodbreak \small \begin{samepage} \begin{verbatim}
Which Extra Extremity, roll 1d10:
1-5  arm: as normal by the ability
6-7  leg: balance+3, r+1, d+2,
          brawl kick: dam+1
8-10 tentacle: climb+3, wrestle+3, block+3
          moves independently up to 2sq from body
          tentacle has 3hp, ranged tohit-6
                            melee pierce tohit-3
          brawl slap: dam 3
          brawl grab: dam 1 pen 2,
                hold: str-3 vs str
                      break free attempts do not cost
                      ap for the tentacled holder
          grows back: on successful con check, 1x/day
\end{verbatim} \end{samepage} \normalsize

\goodbreak \small \begin{samepage} \begin{verbatim}
Dubious mutations, 1d23:
1-3   reduced size (change token size), runts cannot shrink further
      str-1, dex+1, con-1, hp-2, r-1, d-2
4-5   increased size (size +2 steps), goblin becomes 2x2 size,
      2x str,con, 3x hp, r+1, d+2
6     increased size (size +3 steps), goblin becomes 3x3 size,
      3x str, 2x con, 5x hp, w+1, r+2, d+3
7-9   Bulbous Behind: Ever seen a Centaur? This is more of a Dachshund.
      The Hero grows a large derriere, complete with extra legs.
      The "behind" is an extra token that should always be
      placed straight behind the Hero's main token along path
      of movement, hero can still "face" +/- 45deg without
      moving the rear. Moving the rear costs no extra mp.
      r+1, d+2, hp+5, balance+3
10-12 Stone Skin: abs+1, dex-2, cha-3, speaking-3
13-15 Metabolism: stamina+2, must eat 4x full rations each day
16-18 Evil Mini Twin, conjoint: A nasty little malformed mini with an
      evil mind of it's own grows out from the stomach/back/side of
      the Hero. It will sometimes cooperate
      cha-3, hp+2d3, counts as 3enc, has 1d2 tiny arms (str 1d3)
19-23 Glowing: light source range 2d3 -1/abs of armour,
      will attract monsters like "tasty"
\end{verbatim} \end{samepage} \normalsize

\flushbottom


\subsection*{loot: magifying gear}
%---------------------------------
Bringing items and leaving them on the island for a few days will make them magical, for better or worse. It's a 50/50 chance the items will just disappear, or get destroyed. If they remain, then they accrue some fun minor behavioural hiccup, boost, defect, or cosmetic change.

Examples: colour change, glowing, smoking, hot, cold, burning, dripping water, attack+1, parry+1, reach+1, knockback+1, abs+/-50\% or even +100\%, or "indestructible", wielder stat changes like str+1 or int-1, or giving special ability like "Black Knight 2" or "Combat Advantage", or simply tagged "magical" meaning it can harm some incorporeal things and special monsters.
Perhaps the weapon seems to be cool with a mod+1, but actually gives 50\% of failing all hits that are not success+3 or better? A sneaky curse.


\subsection*{playthrough}
%------------------------
The players decide to go hunting for a new Hoom Hool after they hear from a patrol that Hjalmar Hjälte is searching for people to form a posse to go root out the Evil Goblin menace he thinks have settled in the old Goblin Cave in the west of SouthWoods. GammelTant is against the idea since they have defeated Hjalmar Hjälte before, but since she has enough food for a little while she doesn't care much.

Opposition balance: The players initially brought 8 goblins around 130-140xp.\\ Approximated critters: initial : 12 bugs, 4 fish, 6 lumps, 2 larva, 1 crab, spread out and not attacking in coordinated fashion.\\
Goblin reinforcements for following sessions: 3x more goblins\\
Critter spawns: 8 bugs, 3 fish, 4 lumps, staggered spread.

Fighting for 39 rounds. Almost killed 3 goblins, wounded 2 more, all got down to zero mana, most to near 0 stamina.\\
Played the critters in an uncoordinated fashion, with no tactical deviousness. The players could quickly see the behaviour patterns and start planning and coordinating their side of the fight.

\

Session 1: rounds 1-3 explore, 4-13 fight, 2h30m\\
They travel towards BlackPeak Mountains, and have trained a few points of "track", so they actually notice when they pass close to the forest stream map marker.\\
On map they start exploring, run into exploding bugs, quickly realise the problem and change tactics. Start bolting bugs. Get stuck hammering away at moss lumps, fish and more lumps are attracted with staggered arrival. Goblins manage to control positioning and have the control of the fight until one of the larva joins in and wrecks their strategy.\\
One goblin gets seriously wounded and has to flee the fight.

\

Session 2: rounds 14-23 fight, 2h45m\\
The bring in one very quick melee rapier goblin to the map.
The larva is tricky until they can re-position their fighters for better efficiency against the heavy armoured grub and bring spear wielders to the right positions. Once they have figured out the details they immediately handle the arrival of the second larva in a quicker more efficient way, but still have problem with positioning since the heavy larva can tackle and force them out of the way easily.\\
In the end their spell casters have near zero mana, and most fighters have very low stamina.

\

Session 3: rounds 24-28 fight, 29-31 cleanup, 32-39 crab, 2h55m\\
They bring in 2x new heavy 2h club melee goblins to the map.
They kill off the second larva and the staggered reinforcements very efficiently now that they know the monster behaviour and weaknesses. They mop up the last critters, almost clearing the map, and can rest for a couple of rounds as they advance on the crab. The crab fight is relatively quick since they can bring in the heavy melee club wielders in the forefront, though one of them almost dies and the other gets seriously wounded.









\newpage
%--------|---------|---------|---------|---------|---------|---------|---------|
%       10        20        30        40        50        60        70        80
%-------------------------------------------------------------------------------
\phantomsection\addcontentsline{toc}{section}{XX iffygriff eggs}
\section*{XX intermission: iffygriff eggs}
\markboth{iffygriff eggs}{iffygriff eggs}


Just a small side adventure, should come somewhat early in the campaign. Save it for an evening when only a few players are available.


\subsection*{synopsis}

Iffygriffs have been spotted flying to and from The Moors. IffyGriff Eggs are a delicacy for the goblins...


\subsection*{adventure}

They have seen IffyGriffs and go scouting for the nest. They want to get the eggs which are a delicacy to goblins. The eggs are also worth quite a bit of money if they can be brought quickly to a larger town.

\begin{readoutloud}
\emph{[GammelTant roars:]}
Go Get Me Em Eggs! Or I Will Make Goblin Cook Stew For Next Dinner!
\end{readoutloud}

Travel south to the moors. They'll need some track skill to easily find the iffygriff nest site.

The IffyGriffs live in the ruined tower. The other tower has somewhat intact roof and walls and it is possible to hide in there, but the roof and walls are very unsafe and might collapse if an adult iffygriff attacks.

If the goblins arrive in the day at least one adult iffygriff is away hunting. At night both adults are present.

The small bandit gang is present on a 50/50 basis. Adapt to suit the level and amount of Heroes. Can also switch out goblins to heavier orcs. The bandits have smeared iffygriff poo over themselves to not get attacked by the birds. \emph{This works, but will eventually eat away at their armour, then clothes and skin, highly toxic.}


\subsection*{dungeon mechanics}

The IffyGriffs are housing their nest in an old crumbled tower ruin. There is no roof and a sturdy tree growing in the old ruin.

The ghost activates when someone enters the tower. It only speaks Ancient. It will ask all who enter (in ancient):
\begin{readoutloud}
\emph{[ghostly whisper:]}
Have you come to relieve me and take up my post?
\end{readoutloud}
If they can't understand, or don't answer (in ancient) that they will take up his post he will scream \emph{Intruders!} (in ancient) and attack. The scream is a fear attack with: fear mod-3, on fail: flee and drain 3stam 3mana, fail-3 pass out.
If they answer "yes" (in ancient) the ghost will hand them the key to the tower and the magical sword "Ifring". If they take the key and sword they are now bound to the tower for eternity, or until they can pass on the post to someone else. If they don't take the key and sword he will scream \emph{Infiltrators!} (in ancient) and attack, same as before.

The Key binds the Hero to the tower for eternity or until the post is passed on to someone else. The Sword is cursed and eventually forces them to travel to the tower and take up the key (the guard post). The holder of the sword must pass daily psy roll or start travelling to the guard tower to take up the post.


\subsection*{opposition}

The adult iffygriffs will defend their young until death. If the young are not around they will flee when below half hp. The young will flee and regroup when below half hp. They can all leave the eggs behind if too seriously wounded.

The bandits take shelter in the ruin. They don't trust the "intact" tower and are afraid of the ghost.

The ghost must be avoided or defeated with magic. It won't pursue past 10sq from the tower, and will simply wait a bit then go back in and de-materialise soon after. The ghost will however pursue anyone who holds the sword but not the key!


\subsection*{loot}

The bandits have a few silver and copper on their persons, and a bit more in the hidden loot stash. The stash must be searched for with "find", not perception. For Heroes with "find" it's mod+3. Heroes without "find" cannot find it.

The sword "Ifring" is \emph{very} valuable. In the small villages it will sell for 10g with successful haggle. In trade town at least 20g. But: the sword is cursed. The owner must pass a psy roll each day or start travelling to the tower to take up the guard post...


\subsection*{playthrough}

bla bla bla







\newpage
%--------|---------|---------|---------|---------|---------|---------|---------|
%       10        20        30        40        50        60        70        80
%-------------------------------------------------------------------------------
\phantomsection\addcontentsline{toc}{section}{XX torture dungeon}
\section*{XX intermission: torture dungeon}
\markboth{torture dungeon}{torture dungeon}

Evil Undead Torturer Wizard, bla bla bla. Hack through the dungeon for xp and loot, or run away while still alive.


\subsection*{synopsis}

Follow the treasure map into the Dun Hills. Find the ruined tower. Fight the crazy farmer (servant minion) or perhaps capture and question him. Black Bugs leaking out from the hidden entrance. Enter the dungeon and fight the undead and the bugs. Tricky trap by the entrance and a hidden loot stash at the end.


\subsection*{adventure}

Some evil dude long ago was experimenting in blood magic. Tortured folks in his hideaway evil lair torture dungeon. Now the Heroes have found a treasure map, indicating a location in the dun hills.

%map: 7 leagues south of the old guard outpost, 8 leagues sw of Kegern, 3 leagues nw from the lake. The old wizards tower.
The treasure map lists the "distances" as regular human days' travel. I.e. for con 5 it's easy 2 days (7 leagues) from the abandoned outpost, as well as from Kegern, and one day (3 leagues) from the lake.
When they get to the right area they will find the tower ruins in the woods.

The Heroes can prepare for this adventure by trying gossip or histography in the towns and castles. A successful gossip will tell an old story of an evil wizard in a tower in the Duns. And people still disappear around those parts even today. With histography they can find out the location of the tower, and that the original owner was an evil wizard. With success+3 they find that there was a torture dungeon below. With a success+6 they can find a map schematic of the tower in an old archive. With success+9 they also find the ancient construction plans for the torture dungeon, incl trap pit but not trap function. Costs 1d3 silver to have a copy made. Kegern is close by and has mod+3 to both gossip and histography for this purpose. Merna is far away and has mod-3.


\subsection*{dungeon mechanics}

%TODO: hints for the pit trap exit and for finding the hidden loot stash
\begin{description}
\item[pit trap:] The pit trap by the entrance is set to trigger on human size trespassers. Goblins and halflings are too small and require two concurrent occupants to trigger. When triggered a Hero has a per-6 roll to react, and a jump roll mod-3 to get to safety. On fail the Heroes fall to the floor far below and take 1d3 damage penetrating. Then a soft buzzing starts. In 1d3+1 rounds they will be sucked away (direction SW) and will be de-materialise off map. They can be re-materialised at the SW room double pillars.
The walls are super slick, climb-9. A 10m rope is long enough to help fish out anyone who is trapped, if they are quick enough.

A perception roll mod-6 can detect that there is something strange with the corridor just inside the entrance. A find roll (mod+3 if already detected by perception) can see that it's probably a trap. Dungeoneering or locks \& traps can help figure out the pit trap behaviour; it will trigger on heavier people but allow goblins to pass.

\item[trap exit:] In the SW corner room there is a strange control panel with a big breaker, and two large metallic pillars on the floor. Closing the breaker will activate the angry magic arc lights and re-materialise all who has fallen into the pit trap and been de-materialised, one per round.

The angry arc lights will force a continuous stun 9 (not cumulative) on the targets. They disappear after the breaker is switched off and the stun will dissipate by con each round as usual. It is possible to break the stun if con $>$9. To break free, roll con-9 once per round.

\item[black bugs:] The dungeon is crawling with black bugs. The small black bugs are normally uncoordinated, but the big black bug can call them back and control them if it gets attacked. If attacked, the Big Black Bug will screech loudly and all black bugs in the dungeon will take flight (dash) to come to aid asap. They will attack any Hero they encounter on the way.

\item[skeletons:] Skeletons take only 1dam from piercing attacks regardless of what damage was rolled. They take full damage from other types of attacks. Spears are mod-1 and only do half damage when used for slashing/bashing attacks.

\item[wraiths:] Wraiths take no damage from physical attacks and half damage from elemental attacks. They take full damage from magical attacks including elemental magic.
\end{description}


\subsection*{opposition: black bugs}

The Black Bugs infest the whole dungeon area, but can be cleared out in small groups since they don't normally coordinate their movement and attacks. But once the Heroes have killed some in melee it's more likely the next ones attack immediately when in smell range.

The black bugs ignore the skeletons. There is nothing to eat on them. But it's possible to get them to attack by smearing the skeletons with black bug gore. The Big Black Bug has not specifically bred the small black bugs to ignore the undead. BBB will not be fooled by this trick however, and any BB under her direct control will focus on the Heroes instead.

\begin{description}

\item[Black Bug:] They attack on "sight"(smell), ca 6sq. There is a 50\% chance that they will become hostile and attack each round when a potential target is in their smell range. Mostly they stand around or mill around slowly. Normally they are independent and uncoordinated. They will however attack any non-bug who has bug blood/gore on them, since it signals an enemy.\\
Take the stats from the rule book.

\item[Big Black Bug:] The Big Black Bug is the hive mommy. It can can direct the normal smaller black bugs to attack in a coordinated fashion. The Big Black Bug also starts spawning more black bug critters if attacked. The Big Black Bug seems trapped in the cage, but have a 50\% chance of breaking out each round if attacked.\\
Take the stats from the rule book.

\end{description}


\subsection*{opposition: undead}

The undead are mostly intelligent and coordinated, and they are safe from the black bugs unless some clever Hero starts smearing them with black bug gore. Monsterology is good to have!

\begin{description}

\item[Skeleton Grunt:] barely functional undead. This lowly and barely intelligent bag of bones follows simple instructions without individual thought. There is not much undead blood magic left in them after all these years. They are tasked with patrolling and defending the dungeon. Wake the others if something happens, and bring any intruders to the stun pillars if possible.
\goodbreak \begin{samepage} \small \begin{verbatim}
===================================
skeleton grunt         (skeleton02)
-----------------------------------
str  6    hp 10 abs 0
dex  5    m1 w3 r- d-
psy  3    vision 15
per  3    initiative 6, ap 4
----------
immune to pain
no low hp mods (Black Knight 3)
takes half hp from thrust/pierce attacks (arrows, spears, etc)
----------
spear     5 dam 5/6, pen 1/2, abs 8, reach 1 mod-3
sword     5 dam 6, abs 10
shield    8 abs 10, parry+3
-----------------------------------
healthy: >5hp : 3ap M, NO parry, just attack
damaged: will parry
===================================
aka: skeleton grunt, skeleton soldier
\end{verbatim} \normalsize \end{samepage}


\item[Skeleton Guard:] determined efficient undead. They will think for themselves, as to how they can best fulfil their orders. They will protect the dungeon, the equipment, and most of all the Undead Master.
\goodbreak \begin{samepage} \small \begin{verbatim}
===================================
skeleton guard         (?)
-----------------------------------
str  9    hp 15 abs 2
dex  5    m2 w4 r6 d8
psy  3    vision 15
per  9   initiative 8, 8ap
----------
immune to pain
no low hp mods (Black Knight 3)
takes half hp from thrust/pierce attacks (arrows, spears, etc)
----------
glave      8 dam 9, pen 2, abs 8, parry-3, slow-1, toavoid+1
             reach 0 mod-1, reach 1 mod-0, reach 2 mod-6,
avoid      7 yield+3
-----------------------------------
healthy >5hp : 8ap M, NO parry, just attack  :  2x attack glave
damaged: 8ap will avoid  :  1/2x avoid / 1x attack glave
glave attack, avoid defence
===================================
\end{verbatim} \normalsize \end{samepage}


\item[Undead Master:] The undead remains of the Old Evil Dude who built this whole shebang. He still has some memories of his previous life, but not much is left. He simply follows the slowly evaporating drive to capture people and drain their life to keep himself going. No new learning or research comes out of this horrible place any more.
\goodbreak \begin{samepage} \small \begin{verbatim}
Undead Master
===================================
scary skeleton          (?)
-----------------------------------
str  8    hp 20 abs 0
dex  7    m2 w4 r7 d10
int  8    vision 15
psy  8    initiative 10, ap 9
per  5
----------
immune to pain
takes half damage from thrust/pierce attacks
no low hp mods (Black Knight 3)
----------
mobile 1
charge 3
avoid 8
sword 11, duelling, poke, swing
double 6
counter attack x
poke x
swing x
-----------------------------------
2x exquisite     master crafted: pen 1  abs +50%
small sword      dam 5(4), pen 1(0), abs 12(8), finesse-7(6)
(duelling spc)   str 2 (max +1 str bonus, max +1 str pen bonus),
                 fast+1: str 5 dex 7
                 poke: mod-1 dam-1 pen+1
                 swing: slow-1 mod-1 dam+1 todefend+1
===================================
\end{verbatim} \normalsize \end{samepage}


\item[Wraith:] very scary ghost. Cold shivers run down your spine just by looking at it, and when it screams you just run away. These are summoned when needed and are there to simply drain any intruders and put them to sleep for easy capture and blood sacrifice. Very difficult to fight unless the Heroes are equipped with some magical weapons, shields, and armour.
\goodbreak \begin{samepage} \small \begin{verbatim}
===================================
wraith                 (?)
-----------------------------------
str  -    hp 12 abs 0
dex  -    m2 w4 r5 d6
int  4    vision 15 night (set elf poor)
psy  8    initiative 8, ap 3
per  7
----------
can move through walls/ground
pain threshold 5
immune to non-magical physical attacks
half damage from non-magical elemental damage
full damage from magical weapons/elemental/damage
----------
avoid 5 yield+3
touch 7 dam 1 pen 99

scary 3: aura 3,
    move closer (into/in aura): psy+3 vs 3
    attack melee: psy vs 3
    passing a move test gives mod+3 to future move tests
    passing an attack test gives mod+3 to future attack tests

touch attack: dam 1 penetrating. roll 8 vs psy on successful attack for drain
    parried only with magical weapons, avoid is only other defence
    drain: 1d4 stamina, 1d4 mana on touch attack, heals 1hp per drain attack
    attacks until target passed out / dead
    magical armour protects completely: attack mod = -abs, and no penetration

scream defence: vs: 8 vs psy
    area range 5 psy attack to scare people away
    run away dash/run each round until not scared
    scared until passes psy roll mod+1 cumulative per round
    Targets who recover have mod+3 against future scream attacks
    Failing a scream defence resets the aura bonuses.
===================================
\end{verbatim} \normalsize \end{samepage}

\end{description}


\subsection*{loot}

The loot stash is hidden at the bottom of the Big Black Bug spawning pool (old latrine). It contains 12g 109s 320c (ca 25g total).

The weapons and armour of the skeletons are of excellent quality, but ancient and worn (abs-1). Collectors will pay extra, fighters will pay less.


\subsection*{playthrough}

bla bla bla







\newpage
%--------|---------|---------|---------|---------|---------|---------|---------|
%       10        20        30        40        50        60        70        80
%-------------------------------------------------------------------------------
\phantomsection\addcontentsline{toc}{section}{XX burglary with sleepy fishesh}
\section*{XX intermission: burglary with sleepy fishesh}
\markboth{burglary}{burglary}


\subsection*{synopsis}

bla bla bla


\subsection*{adventure}

bla bla bla

\begin{readoutloud}
\emph{[SomeDude speaks:]}
Go Get Shit Done!
\end{readoutloud}


\subsection*{dungeon mechanics}

bla bla bla

\small \begin{verbatim}
bla bla description etc
\end{verbatim} \normalsize


\subsection*{opposition}

bla bla bla


\subsection*{loot}

bla bla bla


\subsection*{playthrough}

bla bla bla






























































\newpage
%--------|---------|---------|---------|---------|---------|---------|---------|
%       10        20        30        40        50        60        70        80
%-------------------------------------------------------------------------------
\phantomsection\addcontentsline{toc}{section}{XX bla bla}
\section*{XX bla bla}
\markboth{bla bla}{bla bla}


\subsection*{synopsis}

bla bla bla


\subsection*{adventure}

bla bla bla

\begin{readoutloud}
\emph{[SomeDude speaks:]}
Go Get Shit Done!
\end{readoutloud}


\subsection*{dungeon mechanics}

bla bla bla

\small \begin{verbatim}
bla bla description etc
\end{verbatim} \normalsize


\subsection*{opposition}

bla bla bla


\subsection*{loot}

bla bla bla


\subsection*{playthrough}

bla bla bla





























































\newpage
%--------|---------|---------|---------|---------|---------|---------|---------|
%       10        20        30        40        50        60        70        80
%-------------------------------------------------------------------------------
\phantomsection\addcontentsline{toc}{section}{appendix: GamGang}
\section*{appendix: GamGang}
\markboth{GamGang}{GamGang}

The main characters of GamGang are Gamling, GammelTant, and Hemskelina. The rest can be treated as more or less standard units, but with some extra xp if they survive fights. Later on the Kraaazy Runts come in and they be weird non-standard as well.

GamGang is a total of ca 20 npc characters and around 2x goblins per player. During the playthrough that meant around 30 goblins and runts for the beginning of the campaign. 30 gobbos and runts require around 10-20 day's rations of food each day, assuming some are out hunting and foraging in the forest. The more gobbos that are out hunting the fewer will be available to defend Hoom-Hool when the Horrible Hoomans attack.

When the food is out the gang starves. Day 1-3: each day gives mod-1. Day 4-9 each 2 days gives mod-1. Day 10 GammelTant will kill and cook a goblin for sure, and there will be much rejoicing. Each day starving there is a cumulative +10\% chance that GammelTant will kill and cook someone. I.e: day 3 has 30\%, day 7 has 70\%, roll each day.

\pagebreak[2]
\small
\begin{verbatim}
Gamling                   (overgrown goblin)
================== goblin ==================
str  9(6)                 hp 13 abs 2
dex  6                    m1 w3 r5 d7(8)
con  4                    stamina 11(6)
int  1                    vision 25 dusk 257 -- goblin good
psy  6                    mana 0
per  5                    action points 5
cha  1                    xp 8 (310)
--------------------------------------------
yield bonus 2
sneak bonus +2
off balance
--------------------------------------------
Common 2
Svartlingo 1
disengage 3
bite 2
scratch 3
sneak 3 (1,+2)
throw 2
mobile 2         3.0   12
enduring 5       1.0   25
axe 12           0.9  129
avoid 8(5)       0.8   31
slugger 2        ---   20
charge 5         1.0   25
veteran 4        1.0   16
rapid 2          1.5    6
accurate 2       ---   20
strong 3         2.0   18
--------------------------------------------
swing            ---    5
Hardened         ---   10  (pain threshold +1 =4)
--------------------------------------------
\end{verbatim} \goodbreak \begin{verbatim}
money: 1 silver, 1 copper, 3 teeth, 3 stones, 2 feathers, 2 glass beads
goblins can live on half rations, and can eat spoiled food
bite 2 dam 3 3ap (2ap if both hands free) scf 0.5
scratch 3 dam 1 2ap (1ap if both hands free) scf 0.5

2h axe    12 (str 9, slugger 2) dam 13,
    dam 11, abs 11, parry-3, toparry-1, toavoid+1, finesse-3
    str 6
    swing: slow-1 mod-1 dam+3 toavoid+2

brigandine    abs 2 ring, scale, brigantine, etc
    str 3 (str penalties affect all actions)
    dex-1,
    dash-1,
    per-1
    acrobatics mod-3
    martial arts mod-1
    spellcasting mod-1
    sneak mod-3
    takes 3 rounds to put on or take off
staff      5 dam 5/6, abs 8, parry+1/+2, reach 1 mod-3
spells:
    heroism
    slow
    heal




\end{verbatim} \pagebreak[2] \begin{verbatim}
GammelTant      (goblin with some orc blood)
================== goblin ==================
str  6                    hp 19 abs 0
dex  5                    m2 w4 r5 d6
con  4                    stamina 15
int  5                    vision 14 dusk 179
psy  6                    mana 7
per  8                    action points 4
cha  4                    xp 0 (120)
--------------------------------------------
yield bonus 2
veteran bonus +2
brawling bonus +3
--------------------------------------------
Common 3
Svartlingo 5
bite 1
brawling 8 (0+3)        0.6  15
veteran 2 (0+2)
staff 6                 0.8  28
consistent              ---  15
--------------------------------------------
yield
off balance
opportunity
--------------------------------------------
\end{verbatim} \goodbreak \begin{verbatim}
money: 5 silver, 1 copper, and 4 large teeth/claws
double con against poisons
Orcs without any war trophies suffer psy-1 mod, until honour reclaimed.
bite 1 dam 4 slow-1 scf 0.5
brawl fist 8 dam 2 fast+1
brawl kick 8 dam 4
staff      6 dam 5/6, abs 8, parry+1/+2, reach 1 mod-3, finesse-6/-9
1h/2h        str 5/4 (max +1/+2 str bonus)
             2h: fast+1 if str 6 and dex 6
--------------------------------------------
\end{verbatim} \goodbreak \begin{verbatim}
magic      3   1.0    9

slow       8   0.33  21
           cast 1r 1m, +1target/2m, range 15 +5/m, duration 5r +2/m
           Movement costs double movement points.
           All actions are slow-2 (takes +2 extra ap)
           Continue slow on psy vs psy +2/m each round (mana paid once
           on cast) up to the max duration. First round auto success.
           Once slow is lost on one target that target is free until
           the end of the spell duration.
           int 4, psy 3

heal       6   0.33  11
           cast 2r 1m, heal 3 +2/m, time 2hp/r, range contact
           int 5, psy 4

heroism    8   0.33  21
           cast 2r 1m, range 10 +5/m, duration 10r +5/m
           select up to 5 +3/m targets which get that heroic feeling
           they receive +6 psy to all fear tests
           they receive +1 to all offensive actions
           The effect vanishes if the targets get outside the range of the
           spell, which is centred on the caster for the duration.
           A target looses the effect if he does not attack in a round
           when he has the opportunity to do so, unless attacking is clearly
           detrimental to his goal.
           int 5, psy 5




\end{verbatim} \pagebreak[2] \begin{verbatim}
Hemskelina
================== goblin ==================
str  5                    hp 3(10) abs 1
dex  8                    m2 w4 r7(6) d10(8)
con  4                    stamina 8(6)
int  4                    vision 25 dusk 200 - goblin good
psy  4                    mana 8
per  5                    action points 8(6)
cha  4                    xp 5 (335)
--------------------------------------------
yield bonus 5
sneak bonus +2
off balance
--------------------------------------------
Common      3
Svartlingo  3
bite        2
scratch     2

quick       2         4.0   16
mobile      2         3.0   12
fast        2         1.0    4
resilient   3         1.0    9
enduring    2         1.0    4
veteran     3         1.0    9

balance     3         1.0    9
jump        4         0.5    8
swim        4         0.5    8
climb       5         0.5   12
find        4         0.5    8
quick draw  6         0.5   18

sneak       9 (0,+2)  0.5   24
back stab   7         0.5   24
throw       8 (2)     0.8   48
knife       9         0.7   56
avoid       9 (4)     0.8   57
disengage   3 (3)     1.0    0
accurate    1         ---   10

lucky       2         1.0    4
gut feeling 1         4.0    4
--------------------------------------------
\end{verbatim} \goodbreak \begin{verbatim}
money: 1 silver, 2 copper, 1 teeth, 1 stones, 2 feathers, 1 glass beads
goblins can live on half rations, and can eat spoiled food
bite 2 dam 3 3ap (2ap if both hands free) scf 0.5
scratch 2 dam 1 2ap (1ap if both hands free) scf 0.5

2x dagger      9  dam 3, abs 4, parry-1, toparry-2, toavoid-1, finesse-9
                    str 2 (no str bonus)
                    fast+1 if str 3 and dex 5
                    first two attacks don't require stamina
                    poke: mod-1 dam-1 pen+1

4x throwing knife  8  dam 2  fast+1
                  range 8, short 4 mod+3, long 12 mod-3, vlong 16 mod-6 dam-1,
                  str 1 (no str bonus)
                  fast+1 if str 2 and dex 4
                  first three attacks do not require stamina

leather armour  abs 1




\end{verbatim} \pagebreak[2] \begin{verbatim}
Other Dude
================== human ===================
str  4                    hp 11 abs 0
dex  8                    m1 w4 r6 d8
con  6                    stamina 6
int  5                    vision 22 day 223
psy  6                    mana 6
per  4                    action points 3
cha 10                    xp 108
--------------------------------------------
yield bonus 3
--------------------------------------------
Common 3
avoid 4
throw 3
--------------------------------------------
yield
off balance
--------------------------------------------
\end{verbatim} \goodbreak \begin{verbatim}
money: 3 gold, 1 silver, 10 copper

============================================

\end{verbatim}
\normalsize


























\newpage
%--------|---------|---------|---------|---------|---------|---------|---------|
%       10        20        30        40        50        60        70        80
%-------------------------------------------------------------------------------
\phantomsection\addcontentsline{toc}{section}{appendix: SturSkurk Gang}
\section*{appendix: SturSkurk Gang}
\markboth{SturSkurk}{SturSkurk}

The SturSkurk Boss is a dangerous opponent but the witch is also surprisingly touch in a melee fight. The Boss always fights with orc phalanx support. The Witch supports with ward shields, slow, and shock bolts. The archers are protected by the spear men. The scouts are quick mobile support.

For the 01 Robbery Gone Wrong there is probably just the boss, 2 orcs, witch, 3 archers. For the 02 Kill The Bandits the rest of the SturSkurk Gang is available, minus those that were killed in the previous episode. The full gang has boss, witch, 3x orcs, 4x archers, 2x spear men, 2x scouts, 4x goblin fighters, 4x camp babes.


\pagebreak[2]
\small
\begin{verbatim}
SturSkurk Boss

6ap-0, 9ap-3, W-1
fancy-4 OR counter   , poke & swing
intercept
phalanx

2x Quaff!  quick draw 4 + häfva 8 + dead drunk 5
black cat 2

mobile 2

=============== SturSkurk ==================
str  9                    hp 18 abs 3
dex  6(7)                 m2 w4 r7 d9(10)
con  9                    stamina 12(8)
int  6(4)                 vision 16 day 221
psy  7                    mana 6
per  3(4)                 action points 6(4)
cha  8                    xp 9 (420)
--------------------------------------------
yield bonus 2
--------------------------------------------
\end{verbatim} \goodbreak \begin{verbatim}
Common 6
avoid 3
throw 2
quick 2          4.0   16
mobile 2         3.0   12
rapid 2          2.0    8
veteran 4        1.0   16
smart 2          2.0    8
resilient 4      1.0   16
enduring 4       1.0   16
axe 13           0.8  135
shield 9         0.7   56
fancy attacks 4  0.5    8
tank 2           spc   20
consistent 2     spc   30
quick draw 4     0.5    8
häfva 8          0.25  16
dead drunk 5     0.5   12
phalanx          spc   10
counter attack   spc   10
battle axe              5xp
poke                    5xp
swing                   5xp
black cat 2      1.0    4
intercept               5xp
--------------------------------------------
yield
off balance
--------------------------------------------
\end{verbatim} \goodbreak \begin{verbatim}
money: 3 gold, 3 silver, 19 copper

battle axe 1h  13
        dam 8, abs 10, parry-2, toparry-2, finesse-4
        str 5
        swing: slow-1 dam+2 toavoid+2
        poke: mod-2 dam-3 pen+2

tower shield 14(9)
        abs 14, parry+5,
        str 8
        or slow-1 str 5
        Ranged attacks mod-4 when in the way.
        Hiding behind it (3ap) ranged mod-8
        tackle mod+3

plate   tank 2: mods as chain mail
        abs 3
        str 5 (str penalties affect all actions)
        -- tank 2 --
        dex-1,
        dash-1,
        per-1
        acrobatics mod-3
        martial arts mod-1
        spellcasting mod-1
        sneak mod-3
        takes 3 rounds to put on or take off



\end{verbatim} \pagebreak[2] \begin{verbatim}
SturSkurk Witch

4ap-0, 6ap-2, W-2
spells 1r: fire storm 8, ward shield 8, shock bolt 7, slow 7

6x quaffs in belt, quick draw 8, to help others

================== human 1 ==================
str  6                    hp 20 abs 0
dex  6                    m1 w3 r6 d11
con  2                    stamina 7
int  8                    vision 19 day 207
psy  8(7)                 mana 20
per  8                    action points 4(3)
cha  5                    xp 1 (230)
--------------------------------------------
yield bonus 3
--------------------------------------------
\end{verbatim} \goodbreak \begin{verbatim}
Common 5
avoid 2
throw 3

literate 4            0.5  12
counting 3            0.5   4
ancient 3             0.5   4
histography 3         0.5   4

quick 1               4.0   4
mobile 1              3.0   3
determined 1          2.0   2
veteran 3             1.0   9
quick draw  8         0.5  32
staff 8               0.8  51
accurate 1            spc  10

magic  4              1.0  16
power casting  3      0.5   4
fire storm  8         0.33 21
ward shield  8        0.33 21
shock bolt  7         0.33 16
slow  7               0.33 16
--------------------------------------------
yield
off balance
--------------------------------------------
\end{verbatim} \goodbreak \begin{verbatim}
money: 4 gold, 8 silver, 10 copper

staff 8 2h fast+1
        dam 4/5, abs 8, parry+1/+2, reach 1 mod-3, finesse-6/-9
        str 5/4 (max +1/+2 str bonus)
        2h: fast+1 if str 6 and dex 6

fire storm  8
    cast 1r 1m, dam 5 +2/m, range self, radius 1 +1/m,
    duration 3r +3/m, damage each round,
    caster is immune to fire for the duration
    The storm follows the caster at most up to walking speed,
    if the caster moves faster the storm will disappear.
    The max damage of the storm each round is reduced by one for
    each movement speed declared for the round.
    int 6, psy 8

ward shield  8
    cast 1r 2m, range touch, duration 5r +3r/m
    charge 4 +2/m (pay on cast)
    Absorbs incoming damage before it strikes the character
    or his equipment. Each absorbed damage point costs
    one charge from the spell.
    Ignores penetrating, i.e. penX just passes through the spell.

shock bolt  7
    cast 1r 1m, dam 5 +1/m, penetrating,
    range 15 +10/m, long -3dam,
    int 5, psy 5

slow  7
    cast 1r 1m, +1target/2m, range 15 +5/m, duration 5r +2/m
    Movement costs double movement points.
    All actions are slow-2 (takes +2 extra ap)
    Full/multi-round actions take +1r
    Continue slow on psy vs psy +2/m each round (mana paid once
    on cast) up to the max duration. First round auto success.
    Once slow is lost on one target that target is free from
    the spell.
    int 4, psy 3



\end{verbatim} \pagebreak[2] \begin{verbatim}
SturSkurk Orc

6ap-1, 9ap-4, W-1
phalanx
opportunity

================== orc 0 ==================
str  7                    hp 18 abs 1
dex  5                    m2 w4 r6 d8
con  8                    stamina 10
int  4                    vision 11 dusk 175
psy  3                    mana 0
per  3                    action points 5
cha  1                    xp 147
--------------------------------------------
yield bonus 2
veteran bonus +3
brawling bonus +2
--------------------------------------------
\end{verbatim} \goodbreak \begin{verbatim}
Common 5
Svartlingo 7
bite 4
brawling (incl bonus) 4
veteran 4(0,+3)          1.0    1
quick 2                  4.0   16
mobile 2                 3.0   12
spear 9                  0.8   64
shield 7                 0.7   34
accurate 1               spc   10
phalanx x                spc   10
--------------------------------------------
yield
off balance
opportunity
--------------------------------------------
\end{verbatim} \goodbreak \begin{verbatim}
money: 4 silver, 4 copper, and 4 large teeth/claws
double con against poisons
Orcs without any war trophies suffer psy-1 mod, until honour reclaimed.
bite 4 dam 4 slow-1 scf 0.5
brawl fist 4 dam 2 fast+1
brawl kick 4 dam 4

heavy spear 1h/2h   9
        dam 7/8, pen 1/2, abs 8, parry-1/+1, reach 1 mod-3
        str 6 (max +3 str damage bonus, rest is penetrating)
        narrow tip spears can have all str bonus as penetrating
        broad tip spears are dam+1 mod-1
        finesse-4/-6

large shield   11(7)
        abs 12, parry+4,
        str 6
        or slow-1 str 3
        Ranged attacks mod-3 when in the way.
        Hiding behind it (3ap) ranged mod-6
        tackle mod+2

leather armour abs 1
        acrobatics mod-1
        takes 2 rounds to put on or take off



\end{verbatim} \pagebreak[2] \begin{verbatim}
SturSkurk Goblin Archer

4AP-0 6AP-2  W-3

bow 1r 8  7-14, 24-30-35

================== goblin 0 ==================
str  4                    hp 8 abs 0
dex 11                    m1 w3 r5 d8
con  4                    stamina 6
int  4                    vision 17 dusk 193
psy  4                    mana 4
per 10                    action points 4
cha  2                    xp 8 (140)
--------------------------------------------
yield bonus 3
sneak bonus +1
disengage bonus +1
--------------------------------------------
\end{verbatim} \goodbreak \begin{verbatim}
Common 3
Svartlingo 3
avoid 8 (5)         0.8   31
bite 3
disengage 3 (2,+1)
scratch 3
sneak 3 (2,+1)
throw 3
veteran 2           1.0    4
strong 1            2.0    2
bow 8               1.2   76
quick shot 3        1.5   13
accurate 1          spc   10
--------------------------------------------
yield
off balance
--------------------------------------------
\end{verbatim} \goodbreak \begin{verbatim}
money: 1 silver, 6 copper, 4 teeth, 2 stones, 1 feathers, 2 glass beads
goblins can live on half rations, and can eat spoiled food
bite 3 dam 2 3ap (2ap if both hands free) scf 0.5
scratch 3 dam 1 2ap (1ap if both hands free) scf 0.5

bow     8
        dam 4, penetrating 1
        range 14, short 7 mod+3, long 24 mod-3,
        vlong 30 mod-6 dam-1, extreme 35 mod-9 dam-2,
        str 5 (no str bonus)



\end{verbatim} \pagebreak[2] \begin{verbatim}
SturSkurk Goblin Spear man

5AP-0 6AP-1  W-1
phalanx

================== goblin 0 ==================
str  5                    hp 9 abs 0
dex 10                    m1 w3 r5 d7
con  4                    stamina 6
int  4                    vision 17 dusk 193
psy  4                    mana 4
per  8                    action points 5(4)
cha  2                    xp 13 (140)
--------------------------------------------
yield bonus 3
sneak bonus +1
disengage bonus +1
--------------------------------------------
\end{verbatim} \goodbreak \begin{verbatim}
Common 3
Svartlingo 3
avoid 5 (5)         0.8    0
bite 3
disengage 3 (2,+1)
scratch 3
sneak 3 (2,+1)
throw 3
quick 1             4.0    4
mobile 2            3.0   12
veteran 2           1.0    4
strong 1            2.0    2
spear 8             0.8   51
shield 7            0.7   34
phalanx X           spc   10
accurate 1          spc   10
--------------------------------------------
yield
off balance
--------------------------------------------
\end{verbatim} \goodbreak \begin{verbatim}
money: 1 silver, 6 copper, 4 teeth, 2 stones, 1 feathers, 2 glass beads
goblins can live on half rations, and can eat spoiled food
bite 3 dam 2 3ap (2ap if both hands free) scf 0.5
scratch 3 dam 1 2ap (1ap if both hands free) scf 0.5

leather armour   abs 1

spear   8   dam5 pen1
1h/2h   dam 5/6, pen 1/2, abs 6, parry-1/+1, reach 1 mod-3
        str 4 (max +2 str damage bonus, max+2 penetrating bonus)
        narrow tip spears can have all str bonus as penetrating
        broad tip spears are dam+1 mod-1
        finesse-4/-6

shield  10(7)
        abs 10, parry+3,
        str 4
        Ranged attacks mod-2 when in the way.
        Hiding behind it (3ap) ranged mod-4
        tackle mod+1



\end{verbatim} \pagebreak[2] \begin{verbatim}
SturSkurk Goblin Scout

5AP-0 6AP-1  W-1

sneak 8

staff 9  2ap

================== goblin 0 ==================
str  6(4)                 hp 7 abs 0
dex 12                    m2 w4 r7(6) d10(8)
con  4                    stamina 6
int  4                    vision 17 dusk 193
psy  4                    mana 4
per 10                    action points 5(4)
cha  2                    xp 9 (140)
--------------------------------------------
yield bonus 3
sneak bonus +1
disengage bonus +1
--------------------------------------------
\end{verbatim} \goodbreak \begin{verbatim}
Common 3
Svartlingo 3
avoid 5 (5)         0.8    0
bite 3
disengage 3 (2,+1)
scratch 3
sneak 8 (2,+1)      0.5   22
throw 3
quick 1             4.0    4
strong 2            2.0    8
mobile 2            3.0   12
fast 2              2.0    8
veteran 2           1.0    4
staff 9             0.8   64
accurate 1          spc   10
--------------------------------------------
yield
off balance
--------------------------------------------
\end{verbatim} \goodbreak \begin{verbatim}
money: 1 silver, 6 copper, 4 teeth, 2 stones, 1 feathers, 2 glass beads
goblins can live on half rations, and can eat spoiled food
bite 3 dam 2 3ap (2ap if both hands free) scf 0.5
scratch 3 dam 1 2ap (1ap if both hands free) scf 0.5

leather armour   abs 1

staff   9 dam5 parry+2
1h/2h   dam 4/5, abs 8, parry+1/+2, reach 1 mod-3, finesse-6/-9
        str 5/4 (max +1/+2 str bonus)
        2h: fast+1 if str 6 and dex 6

\end{verbatim}
\normalsize






































































\newpage
%--------|---------|---------|---------|---------|---------|---------|---------|
%       10        20        30        40        50        60        70        80
%-------------------------------------------------------------------------------
\phantomsection\addcontentsline{toc}{section}{appendix: monsters}
\section*{appendix: monsters}
\markboth{monsters}{monsters}

bla bla bla


\raggedbottom

\goodbreak \small \begin{samepage} \begin{verbatim}
Some Dude
================== human ===================
str  4                    hp 11 abs 0
dex  8                    m1 w4 r6 d8
con  6                    stamina 6
int  5                    vision 22 day 223
psy  6                    mana 6
per  4                    action points 3
cha 10                    xp 108
--------------------------------------------
yield bonus 3
--------------------------------------------
Common 3
avoid 4
throw 3
--------------------------------------------
yield
off balance
--------------------------------------------
money: 3 gold, 1 silver, 10 copper

============================================
\end{verbatim} \end{samepage} \normalsize

\

\goodbreak \small \begin{samepage} \begin{verbatim}
Other Dude
================== human ===================
str  4                    hp 11 abs 0
dex  8                    m1 w4 r6 d8
con  6                    stamina 6
int  5                    vision 22 day 223
psy  6                    mana 6
per  4                    action points 3
cha 10                    xp 108
--------------------------------------------
yield bonus 3
--------------------------------------------
Common 3
avoid 4
throw 3
--------------------------------------------
yield
off balance
--------------------------------------------
money: 3 gold, 1 silver, 10 copper

============================================
\end{verbatim} \end{samepage} \normalsize

\flushbottom





























\newpage
%--------|---------|---------|---------|---------|---------|---------|---------|
%       10        20        30        40        50        60        70        80
%-------------------------------------------------------------------------------
\phantomsection\addcontentsline{toc}{section}{appendix: playthrough}
\section*{appendix: playthrough}
\markboth{playthrough}{playthrough}

% 171126,1203
Sessions 1-2: Just Another Day in the Office

% 171217, 180107
Sessions 3-4: Robbery Gone Wrong

% 180114,
Session 5: Patrol - Grassland

% 180121, 18, 0204
Sessions 6-8: Mutamonster












%-------------------------------------------------------------------------------
\end{document}
